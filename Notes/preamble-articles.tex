% for type setting urls
\usepackage[hyphens]{url} % This package has to be loaded *before* hyperref
\usepackage[pagebackref,colorlinks,citecolor=darkgreen,linkcolor=darkgreen,unicode]{hyperref}
\usepackage[english]{babel}

%%% Because Germans have umlauts and Slavs have even stranger ways of mangling letters
\usepackage[utf8]{inputenc}

%%% Multi-Columns for long lists of names
\usepackage{multicol}

%%% Set the fonts
\usepackage{mathpazo}
\usepackage[scaled=0.95]{helvet}
\usepackage{courier}
\linespread{1.05} % Palatino looks better with this

\usepackage{graphicx}
\usepackage{comment}

%\usepackage{wallpaper} % For the background image on the cover page
%\usepackage{geometry} % For the cover page
\usepackage{fancyhdr} % To set headers and footers

\usepackage{ifthen}
\usepackage{amssymb,amsmath,amsthm,stmaryrd,mathrsfs,wasysym}
\usepackage{enumitem,mathtools,xspace,xcolor}
\definecolor{darkgreen}{rgb}{0,0.45,0}
\usepackage{aliascnt}
\usepackage[capitalize]{cleveref}
%\usepackage[all,2cell]{xy}
%\UseAllTwocells
% \usepackage{natbib}
\usepackage{braket} % used for \setof{ ... } macro
\usepackage{tikz-cd}
\usepackage{tikz}
\usetikzlibrary{decorations.pathmorphing}
\usepackage[inference]{semantic}
\usepackage{booktabs}

%%%%%%%%%%%%%%%%%%%%%%%%%%%%%%%%%%%%%%%%%%%%%%%%%%%%%%%%%%%%%%%%%%%%%%%%%%%%%%%%
%% To include references in TOC we should use this package rather than a hack.
\usepackage{tocbibind}
%\usepackage{etoolbox}           % get \apptocmd
%\apptocmd{\thebibliography}{\addcontentsline{toc}{section}{References}}{}{} % tell bibliography to get itself into the table of contents


\begin{comment}
%%%% Header and footers
\pagestyle{fancyplain}
\setlength{\headheight}{15pt}
\renewcommand{\chaptermark}[1]{\markboth{\textsc{Chapter \thechapter. #1}}{}}
\renewcommand{\sectionmark}[1]{\markright{\textsc{\thesection\ #1}}}
\end{comment}

% TOC depth
\setcounter{tocdepth}{2}

\lhead[\fancyplain{}{{\thepage}}]%
      {\fancyplain{}{\nouppercase{\rightmark}}}
\rhead[\fancyplain{}{\nouppercase{\leftmark}}]%
      {\fancyplain{}{\thepage}}
\cfoot{\textsc{\footnotesize [Draft of \today]}}
\lfoot[]{}
\rfoot[]{}

%%%%%%%%%%%%%%%%%%%%%%%%%%%%%%%%%%%%%%%%%%%%%%%%%%%%%%%%%%%%%%%%%%%%%%%%%%%%%%%%
%%%% We mostly use the macros of the book, to keep notations
%%%% and conventions the same. Recall that when the macros file
%%%% is updated, we need to comment the lines containing the
%%%% string `[chapter]` since our article is not a book.
%%%%
%%%% Instructions for updating the macros.tex file:
%%%% - fetch the latest macros.tex file from the HoTT/book git repository.
%%%% - comment all lines containing "[chapter]" because this is not a book.
%%%% - comment the definition of pbcorner because the xypic package is not used.
%%%%
\input{macros}

\newcommand{\idsymbin}{=}

%%%%%%%%%%%%%%%%%%%%%%%%%%%%%%%%%%%%%%%%%%%%%%%%%%%%%%%%%%%%%%%%%%%%%%%%%%%%%%%%
%%%% Our commands which are not part of the macros.tex file.
%%%% We should keep these commands separate, because we will
%%%% update the macros.tex following the updates of the book.

%%%% First we redefine the \id, \eqv and \ct commands so that they accept an
%%%% arbitrary number of arguments. This is useful when writing longer strings
%%%% of equalities or equivalences.

\makeatletter

\renewcommand{\id}[3][]{
  \@ifnextchar\bgroup
    {#2 \mathbin{\idsym_{#1}} \id[#1]{#3}}
    {#2 \mathbin{\idsym_{#1}} #3}
  }

\renewcommand{\eqv}[2]{
  \@ifnextchar\bgroup
    {#1 \eqvsym \eqv{#2}}
    {#1 \eqvsym #2}
  }

\newcommand{\ctsym}{%
  \mathchoice{\mathbin{\raisebox{0.5ex}{$\displaystyle\centerdot$}}}%
             {\mathbin{\raisebox{0.5ex}{$\centerdot$}}}%
             {\mathbin{\raisebox{0.25ex}{$\scriptstyle\,\centerdot\,$}}}%
             {\mathbin{\raisebox{0.1ex}{$\scriptscriptstyle\,\centerdot\,$}}}
  }

\renewcommand{\ct}[3][]{
  \@ifnextchar\bgroup
    {#2 \mathbin{\ctsym_{#1}} \ct[#1]{#3}}
    {#2 \mathbin{\ctsym_{#1}} #3}
  }

\makeatother

%%%% We always use textstyle products and sums...
%\renewcommand{\prd}{\tprd}
%\renewcommand{\sm}{\tsm}
\makeatletter
\renewcommand{\@dprd}{\@tprd}
\renewcommand{\@dsm}{\@tsm}
\renewcommand{\@dprd@noparens}{\@tprd}
\renewcommand{\@dsm@noparens}{\@tsm}

%%%% ...with a bit more spacing
\renewcommand{\@tprd}[1]{\mathchoice{{\textstyle\prod_{(#1)}\,}}{\prod_{(#1)}\,}{\prod_{(#1)}\,}{\prod_{(#1)}\,}}
\renewcommand{\@tsm}[1]{\mathchoice{{\textstyle\sum_{(#1)}\,}}{\sum_{(#1)}\,}{\sum_{(#1)}\,}{\sum_{(#1)}\,}}

%%%%%%%%%%%%%%%%%%%%%%%%%%%%%%%%%%%%%%%%%%%%%%%%%%%%%%%%%%%%%%%%%%%%%%%%%%%%%%%%
%%%% We adjust the \prd command so that implicit arguments become possible.
%%%%
%%%% First, we have the following switch. Set it to true if implicit arguments
%%%% are desired, or to false if not. Note turning off implicit arguments
%%%% might render some parts of the text harder to comprehend, since in the
%%%% text might appear $f(x)$ where we would have $f(i,x)$ without implicit
%%%% arguments.

\newcommand{\implicitargumentson}{\boolean{true}}

%%%% If one wants to use implicit arguments in the notation for product types,
%%%% a * has to be put before the argument that has to be implicit.
%%%% For example: in $\prd{x:A}*{y:B}{u:P(y)}Q(x,y,u)$, the argument y is
%%%% implicit. Any of the arguments can be made implicit this way.

%%%% First of all, we should make the command \prd search not only for a
%%%% brace, but also for a star. We introduce an auxiliary command that
%%%% determines whether the next character is a star or brace.
\newcommand{\@ifnextchar@starorbrace}[2]
%  {\@ifnextcharamong{#1}{#2}{*}{\bgroup};}
  {\@ifnextchar*{#1}{\@ifnextchar\bgroup{#1}{#2}}}
  
%%%% When encountering the \prd command, latex should determine whether it
%%%% should print implicit argument brackets or not. So the first branching
%%%% happens right here.
\renewcommand{\prd}{\@ifnextchar*{\@iprd}{\@prd}}

\newcommand{\@prd}[1]
  {\@ifnextchar@starorbrace
    {\prd@parens{#1}}
    {\@ifnextchar\sm{\prd@parens{#1}\@eatsm}{\prd@noparens{#1}}}}
\newcommand{\@prd@parens}{\@ifnextchar*{\@iprd}{\prd@parens}}
\renewcommand{\prd@parens}[1]
  {\@ifnextchar@starorbrace
    {\@theprd{#1}\@prd@parens}
    {\@ifnextchar\sm{\@theprd{#1}\@eatsm}{\@theprd{#1}}}}
\newcommand{\@theprd}[1]
  {\mathchoice{\@dprd{#1}}{\@tprd{#1}}{\@tprd{#1}}{\@tprd{#1}}}
\renewcommand{\dprd}[1]{\@dprd{#1}\@ifnextchar@starorbrace{\dprd}{}}
\renewcommand{\tprd}[1]{\@tprd{#1}\@ifnextchar@starorbrace{\tprd}{}}

%%%% Here we tell the actual symbols to be printed.
\newcommand{\@theiprd}[1]{\mathchoice{\@diprd{#1}}{\@tiprd{#1}}{\@tiprd{#1}}{\@tiprd{#1}}}
\newcommand{\@iprd}[2]{\@ifnextchar@starorbrace%
  {\@theiprd{#2}\@prd@parens}%
  {\@ifnextchar\sm%
    {\@theiprd{#2}\@eatsm}%
    {\@theiprd{#2}}}}
\def\@tiprd#1{
  \ifthenelse{\implicitargumentson}
    {\@@tiprd{#1}\@ifnextchar\bgroup{\@tiprd}{}}
    {\@tprd{#1}}}
\def\@@tiprd#1{\mathchoice{{\textstyle\prod_{\{#1\}}\,}}{\prod_{\{#1\}}\,}{\prod_{\{#1\}}\,}{\prod_{\{#1\}}\,}}
\def\@diprd{
  \ifthenelse{\implicitargumentson}
    {\@tiprd}
    {\@tprd}}
    

%%%% And finally we need to redefine \@eatprd so that implicit arguments also
%%%% works in the scope of a dependent sum.    
\def\@eatprd\prd{\@prd@parens}

\makeatother

%%%%%%%%%%%%%%%%%%%%%%%%%%%%%%%%%%%%%%%%%%%%%%%%%%%%%%%%%%%%%%%%%%%%%%%%%%%%%%%%
%%%% Redefining the quantifiers, so that some of the longer 
%%%% formulas appear one a single line without problems

%%% Dependent products written with \forall, in the same style
\makeatletter
\def\tfall#1{\forall_{(#1)}\@ifnextchar\bgroup{\,\tfall}{\,}}
\renewcommand{\fall}{\tfall}

%%% Existential quantifier %%%
\def\texis#1{\exists_{(#1)}\@ifnextchar\bgroup{\,\texis}{\,}}
\renewcommand{\exis}{\texis}

%%% Unique existence %%%
\def\uexis#1{\exists!_{(#1)}\@ifnextchar\bgroup{\,\uexis}{\,}}
\makeatother
%%%%%%%%%%%%%%%%%%%%%%%%%%%%%%%%%%%%%%%%%%%%%%%%%%%%%%%%%%%%%%%%%%%%%%%%%%%%%%%%

%%%% Introducing logical usage of fonts.
\newcommand{\modelfont}{\mathit} % use 'mf' in command to indicate model font
\newcommand{\typefont}{\mathsf} % use 'tf' in command to indicate type font
\newcommand{\catfont}{\mathrm} % use 'cf' in command to indicate cat font

%%%%%%%%%%%%%%%%%%%%%%%%%%%%%%%%%%%%%%%%%%%%%%%%%%%%%%%%%%%%%%%%%%%%%%%%%%%%%%%%
%%%% Some macros of the book are redefined.

\renewcommand{\UU}{\typefont{U}}
\renewcommand{\isequiv}{\typefont{isEquiv}}
\renewcommand{\happly}{\typefont{hApply}}
\renewcommand{\pairr}[1]{{\mathopen{}\langle #1\rangle\mathclose{}}}
\renewcommand{\type}{\typefont{Type}}
\renewcommand{\op}[1]{{{#1}^\typefont{op}}}
\renewcommand{\susp}{\typefont{\Sigma}}

%%%%%%%%%%%%%%%%%%%%%%%%%%%%%%%%%%%%%%%%%%%%%%%%%%%%%%%%%%%%%%%%%%%%%%%%%%%%%%%%
%%%% The following is a big unorganized list of new macros that we use in the
%%%% notes. 

\newcommand{\mfM}{\modelfont{M}}
\newcommand{\mfN}{\modelfont{N}}
\newcommand{\tfctx}{\typefont{ctx}}
\newcommand{\mftypfunc}[1]{{\modelfont{typ}^{#1}}}
\newcommand{\mftyp}[2]{{\mftypfunc{#1}(#2)}}
\newcommand{\tftypfunc}[1]{{\typefont{typ}^{#1}}}
\newcommand{\tftyp}[2]{{\tftypfunc{#1}(#2)}}
\newcommand{\hfibfunc}[1]{\typefont{fib}_{#1}}
\newcommand{\mappingcone}[1]{\mathcal{C}_{#1}}
\newcommand{\equifib}{\typefont{equiFib}}
\newcommand{\tfcolim}{\typefont{colim}}
\newcommand{\tflim}{\typefont{lim}}
\newcommand{\tfdiag}{\typefont{diag}}
\newcommand{\tfGraph}{\typefont{Graph}}
\newcommand{\mfGraph}{\modelfont{Graph}}
\newcommand{\unitGraph}{\unit^\mfGraph}
\newcommand{\UUGraph}{\UU^\mfGraph}
\newcommand{\tfrGraph}{\typefont{rGraph}}
\newcommand{\mfrGraph}{\modelfont{rGraph}}
\newcommand{\isfunction}{\typefont{isFunction}}
\newcommand{\tfconst}{\typefont{const}}
\newcommand{\conemap}{\typefont{coneMap}}
\newcommand{\coconemap}{\typefont{coconeMap}}
\newcommand{\tflimits}{\typefont{limits}}
\newcommand{\tfcolimits}{\typefont{colimits}}
\newcommand{\islimiting}{\typefont{isLimiting}}
\newcommand{\iscolimiting}{\typefont{isColimiting}}
\newcommand{\islimit}{\typefont{isLimit}}
\newcommand{\iscolimit}{\typefont{iscolimit}}
\newcommand{\pbcone}{\typefont{cone_{pb}}}
\newcommand{\tfinj}{\typefont{inj}}
\newcommand{\tfsurj}{\typefont{surj}}
\newcommand{\tfepi}{\typefont{epi}}
\newcommand{\tftop}{\typefont{top}}
\newcommand{\sbrck}[1]{\Vert #1\Vert}
\newcommand{\strunc}[2]{\Vert #2\Vert_{#1}}
\newcommand{\gobjclass}{{\typefont{U}^\mfGraph}}
\newcommand{\gcharmap}{\typefont{fib}}
\newcommand{\diagclass}{\typefont{T}}
\newcommand{\opdiagclass}{\op{\diagclass}}
\newcommand{\equifibclass}{\diagclass^{\eqv{}{}}}
\newcommand{\universe}{\typefont{U}}
\newcommand{\catid}[1]{{\catfont{id}_{#1}}}
\newcommand{\isleftfib}{\typefont{isLeftFib}}
\newcommand{\isrightfib}{\typefont{isRightFib}}
\newcommand{\leftLiftings}{\typefont{leftLiftings}}
\newcommand{\rightLiftings}{\typefont{rightLiftings}}
\newcommand{\psh}{\typefont{Psh}}
\newcommand{\rgclass}{\typefont{\Omega^{RG}}}
\newcommand{\terms}[2][]{\lfloor #2 \rfloor^{#1}}
\newcommand{\grconstr}[2]
             {\mathchoice % max size is textstyle size.
             {{\textstyle \int_{#1}}#2}% 
             {\int_{#1}#2}%
             {\int_{#1}#2}%
             {\int_{#1}#2}}
\newcommand{\ctxhom}[3][]{\typefont{hom}_{#1}(#2,#3)}
\newcommand{\graphcharmapfunc}[1]{\gcharmap_{#1}}
\newcommand{\graphcharmap}[2][]{\graphcharmapfunc{#1}(#2)}
\newcommand{\tfexp}[1]{\typefont{exp}_{#1}}
\newcommand{\tffamfunc}{\typefont{fam}}
\newcommand{\tffam}[1]{\tffamfunc(#1)}
\newcommand{\tfev}{\typefont{ev}}
\newcommand{\tfcomp}{\typefont{comp}}
\newcommand{\isDec}[1]{\typefont{isDecidable}(#1)}
\newcommand{\smal}{\mathcal{S}}
\renewcommand{\modal}{{\ensuremath{\ocircle}}}
\newcommand{\eqrel}{\typefont{EqRel}}
\newcommand{\piw}{\ensuremath{\Pi\typefont{W}}} %% to be used in conjunction with -pretopos.
\renewcommand{\sslash}{/\!\!/}
\newcommand{\mprd}[2]{\Pi(#1,#2)}
\newcommand{\msm}[2]{\Sigma(#1,#2)}
\newcommand{\midt}[1]{\idvartype_#1}
\newcommand{\reflf}[1]{\typefont{refl}^{#1}}
\newcommand{\tfJ}{\typefont{J}}
\newcommand{\tftrans}{\typefont{trans}}

\newcommand{\tfT}{\typefont{T}}
\newcommand{\reflsym}{{\mathsf{refl}}}
\newcommand{\strans}[2]{\ensuremath{{#1}_{*}({#2})}}

%%%%%%%%%%%%%%%%%%%%%%%%%%%%%%%%%%%%%%%%%%%%%%%%%%%%%%%%%%%%%%%%%%%%%%%%%%%%%%%%
%%%% JUDGMENTS
%%%%
%%%% Below we define several commands for the judgments of type theory. There
%%%% are commands
%%%% * \jctx for the judgment that something is a context.
%%%% * \jctxeq for the judgment that two contexts are the same
%%%% * \jtype for the judgment that something is a type in a context
%%%% * \jtypeeq for the judgment that two types in the same context are the same
%%%% * \jterm for the judgment that something is a term of a type in a context
%%%% * \jtermeq for the judgment that two terms of the same type are the same

\makeatletter
\newcommand{\jctx}{\@ifnextchar*{\@jctxAlignTrue}{\@jctxAlignFalse}}
\newcommand{\@jctxAlignTrue}[2]{& \vdash #2~ctx}
\newcommand{\@jctxAlignFalse}[1]{\vdash #1~ctx}

\newcommand{\jtype}{\@ifnextchar*{\@jtypeAlignTrue}{\@jtypeAlignFalse}}
\newcommand{\@jtypeAlignFalse}[2]{#1\vdash #2~type}
\newcommand{\@jtypeAlignTrue}[3]{#2 & \vdash #3~type}

\newcommand{\jtermc}{\@ifnextchar*{\@jtermcAlignTrue}{\@jtermcAlignFalse}}
\newcommand{\@jtermcAlignTrue}[3]{ & \vdash #3:#2}
\newcommand{\@jtermcAlignFalse}[2]{\vdash #2:#1}

\newcommand{\jtermt}{\@ifnextchar*{\@jtermtAlignTrue}{\@jtermtAlignFalse}}
\newcommand{\@jtermtAlignTrue}[4]{#2 & \vdash #4:#3}
\newcommand{\@jtermtAlignFalse}[3]{#1 \vdash #3:#2}

\newcommand{\jctxeq}{\@ifnextchar*{\@jctxeqAlignTrue}{\@jctxeqAlignFalse}}
\newcommand{\@jctxeqAlignTrue}[3]{& \vdash #2\jdeq #3~ctx}
\newcommand{\@jctxeqAlignFalse}[2]{\vdash #1\jdeq #2~ctx}

\newcommand{\jtypeeq}{\@ifnextchar*{\@jtypeeqAlignTrue}{\@jtypeeqAlignFalse}}
\newcommand{\@jtypeeqAlignTrue}[4]{#2 & \vdash #3\jdeq #4~type}
\newcommand{\@jtypeeqAlignFalse}[3]{#1\vdash #2\jdeq #3~type}

\newcommand{\jtermceq}{\@ifnextchar*{\@jtermceqAlignTrue}{\@jtermceqAlignFalse}}
\newcommand{\@jtermceqAlignTrue}[4]{& \vdash #3\jdeq #4:#2}
\newcommand{\@jtermceqAlignFalse}[3]{\vdash #2\jdeq #3:#1}

\newcommand{\jtermteq}{\@ifnextchar*{\@jtermteqAlignTrue}{\@jtermteqAlignFalse}}
\newcommand{\@jtermteqAlignTrue}[5]{#2 & \vdash #4\jdeq #5:#3}
\newcommand{\@jtermteqAlignFalse}[4]{#1\vdash #3\jdeq #4:#2}
\makeatother

%%%%%%%%%%%%%%%%%%%%%%%%%%%%%%%%%%%%%%%%%%%%%%%%%%%%%%%%%%%%%%%%%%%%%%%%%%%%%%%%
%%%% Often we shall need to display lists of inference rules. This environment
%%%% adjusts the array environment so that there is enough vertical space
%%%% between two inference rules
%%%%
%%%% bug: there's two much space above the array.

\newenvironment{infarray}[1]{\begingroup\renewcommand*{\arraystretch}{3}
\begin{equation*}
\begin{array}{#1}}{
\end{array}
\end{equation*}
\endgroup}

%%%%%%%%%%%%%%%%%%%%%%%%%%%%%%%%%%%%%%%%%%%%%%%%%%%%%%%%%%%%%%%%%%%%%%%%%%%%%%%%
%%%% CONTEXT EXTENSION 
%%%%
%%%% explicit context extension notation which we will use only rarely

\newcommand{\tfext}{\typefont{ext}}

%%%% The context extension command.
%%%%
%%%% To get a feeling of how the command works, here are a few examples.
%%%% \ctxext{A}{B} will print A.B
%%%% \ctxext{{A}{B}}{C} will print (A.B).C
%%%% \ctxext{{{A}{B}}{C}}{{D}{E}} will print ((A.B).C).(D.E)

\makeatletter
\newcommand{\ctxext}[2]{\@ctxext@ctx #1.\@ctxext@type #2}
\newcommand{\@ctxext}{\@ifnextchar\bgroup{\@@ctxext}{}}
\newcommand{\@ctxext@ctx}{\@ifnextchar\ctxext{\@ctxext@nested}{\@ifnextchar\ctxwk{\@ctxwk@nested}{\@ctxext}}}
\newcommand{\@ctxext@type}{\@ifnextchar\ctxext{\@ctxext@nested}{\@ifnextchar\subst{\@subst@nested}{\@ctxext}}}
\newcommand{\@@ctxext}[1]{\@ifnextchar\bgroup{\@ctxext@parens{#1}}{#1}}
\newcommand{\@ctxext@parens}[2]{(\ctxext{#1}{#2})}
\newcommand{\@ctxext@nested}[3]{\@ctxext@parens{#2}{#3}}
\makeatother

%%%%%%%%%%%%%%%%%%%%%%%%%%%%%%%%%%%%%%%%%%%%%%%%%%%%%%%%%%%%%%%%%%%%%%%%%%%%%%%%
%%%% SUBSTITUTION

\newcommand{\tfsubst}{\typefont{subst}}

%%%% The substitution command will act the following way
%%%%
%%%% \subst{x}{P} will print P[x]
%%%% \subst{x}{{f}{Q}} will print Q[f][x]
%%%% \subst{{x}{f}}{{x}{Q}} will print Q[x][f[x]]

\makeatletter
\newcommand{\subst}[2]{\@subst@type #2[\@subst@term #1]}
\newcommand{\@subst}{\@ifnextchar\bgroup{\@@subst}{}}
\newcommand{\@@subst}[1]{\@ifnextchar\bgroup{\subst{#1}}{#1}}
\newcommand{\@subst@term}{\@subst}
\newcommand{\@subst@type}{\@ifnextchar\ctxext{\@ctxext@nested}{\@ifnextchar\ctxwk{\@ctxwk@nested}{\@subst}}}
\newcommand{\@subst@nested}[3]{\@subst@parens{#2}{#3}}
\newcommand{\@subst@parens}[2]{(\subst{#1}{#2})}
\makeatother

%%%%%%%%%%%%%%%%%%%%%%%%%%%%%%%%%%%%%%%%%%%%%%%%%%%%%%%%%%%%%%%%%%%%%%%%%%%%%%%%
%%%% WEAKENING

\newcommand{\tfwk}{\typefont{wk}}

%%%% The weakening command is very much like the substitution command.

\makeatletter
\newcommand{\ctxwk}[2]{\langle\@ctxwk@act #1\rangle\@ctxwk@pass #2}
\newcommand{\@ctxwk}{\@ifnextchar\bgroup{\@@ctxwk}{}}
\newcommand{\@@ctxwk}[1]{\@ifnextchar\bgroup{\ctxwk{#1}}{#1}}
\newcommand{\@ctxwk@act}{\@ctxwk}
\newcommand{\@ctxwk@pass}{\@ifnextchar\ctxext{\@ctxext@nested}{\@ifnextchar\subst{\@subst@nested}{\@ctxwk}}}
\newcommand{\@ctxwk@parens}[2]{(\ctxwk{#1}{#2})}
\newcommand{\@ctxwk@nested}[3]{\@ctxwk@parens{#2}{#3}}
\makeatother

%%%%%%%%%%%%%%%%%%%%%%%%%%%%%%%%%%%%%%%%%%%%%%%%%%%%%%%%%%%%%%%%%%%%%%%%%%%%%%%%
%%%% When investigation pointed structures we use the \pt macro.

\makeatletter
\newcommand{\pt}[1][]{*_{
  \@ifnextchar\undergraph{\@undergraph@nested}
    {\@ifnextchar\underovergraph{\@underovergraph@nested}{}}#1}}
\makeatother

%%%%%%%%%%%%%%%%%%%%%%%%%%%%%%%%%%%%%%%%%%%%%%%%%%%%%%%%%%%%%%%%%%%%%%%%%%%%%%%%
%%%% OPERATIONS ON GRAPHS
%%%%
%%%% First of all, each graph has a type of vertices and a type of edges. The
%%%% type of vertices of a graph $\Gamma$ is denoted by $\pts{\Gamma}$;
%%%% and likewise for the type of edges.

\makeatletter
\newcommand{\pts}[1]{{\@graphop@nested{#1}}_{0}}
\newcommand{\edg}[1]{{\@graphop@nested{#1}}_{1}}
\newcommand{\@graphop@nested}[1]
  {\@ifnextchar\ctxext{\@ctxext@nested}
      {\@ifnextchar\undergraph{\@undergraph@nested}
         {\@ifnextchar\underovergraph{\@underovergraph@nested}{}}}
    #1}
\makeatother

%%%% The following operations of \undergraph and \underovergraph are used to
%%%% define the free category and the free groupoid of a graph, respectively

\makeatletter
\newcommand{\@undergraphtest}[2]{\@ifnextchar({#1}{#2}}
\newcommand{\undergraph}[2]{\@undergraphtest{\@undergraph@parens{#1}{#2}}{\@undergraph{#1}{#2}}}
\newcommand{\@undergraph}[2]{{#2/#1}}
\newcommand{\@undergraph@nested}[3]{\@undergraph@parens{#2}{#3}}
\newcommand{\@undergraph@parens}[2]{(\@undergraph{#1}{#2})}
\makeatother

\makeatletter
\newcommand{\underovergraph}[2]{\@underovergraphtest{\@underovergraph@parens{#1}{#2}}{\@underovergraph{#1}{#2}}}
\newcommand{\@underovergraph}[2]{{#2}\,{\parallel}\,{#1}}
\newcommand{\@underovergraphtest}{\@undergraphtest}
\newcommand{\@underovergraph@parens}[2]{(\@underovergraph{#1}{#2})}
\newcommand{\@underovergraph@nested}[3]{\@underovergraph@parens{#2}{#3}}
\makeatother

\newcommand{\graphid}[1]{\mathrm{id}_{#1}}
\newcommand{\freecat}[1]{\mathcal{C}(#1)}
\newcommand{\freegrpd}[1]{\mathcal{G}(#1)}

%%%%%%%%%%%%%%%%%%%%%%%%%%%%%%%%%%%%%%%%%%%%%%%%%%%%%%%%%%%%%%%%%%%%%%%%%%%%%%%%
%% Some tikz macros to typeset diagrams uniformly.

\tikzset{patharrow/.style={double,double equal sign distance,-,font=\scriptsize}}
\tikzset{description/.style={fill=white,inner sep=2pt}}

%% Used for extra wide diagrams, e.g. when the label is too large otherwise.
\tikzset{commutative diagrams/column sep/Huge/.initial=18ex}

%%%%%%%%%%%%%%%%%%%%%%%%%%%%%%%%%%%%%%%%%%%%%%%%%%%%%%%%%%%%%%%%%%%%%%%%%%%%%%%%
%%%% New theorem environment for conjectures.

\defthm{conj}{Conjecture}{Conjectures}

%%%%%%%%%%%%%%%%%%%%%%%%%%%%%%%%%%%%%%%%%%%%%%%%%%%%%%%%%%%%%%%%%%%%%%%%%%%%%%%%
%%%% The following environment for desiderata should not be there. It is better
%%%% to use the issue tracker for desiderata.

\newenvironment{desiderata}{\begingroup\color{blue}\textbf{Desiderata.}}
{\endgroup}
