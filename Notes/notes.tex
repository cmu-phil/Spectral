\documentclass{article}

% for type setting urls
\usepackage[hyphens]{url} % This package has to be loaded *before* hyperref
\usepackage[pagebackref,colorlinks,citecolor=darkgreen,linkcolor=darkgreen,unicode]{hyperref}
\usepackage[english]{babel}

%%% Because Germans have umlauts and Slavs have even stranger ways of mangling letters
\usepackage[utf8]{inputenc}

%%% Multi-Columns for long lists of names
\usepackage{multicol}

%%% Set the fonts
\usepackage{mathpazo}
\usepackage[scaled=0.95]{helvet}
\usepackage{courier}
\linespread{1.05} % Palatino looks better with this

\usepackage{graphicx}
\usepackage{comment}

%\usepackage{wallpaper} % For the background image on the cover page
%\usepackage{geometry} % For the cover page
\usepackage{fancyhdr} % To set headers and footers

\usepackage{ifthen}
\usepackage{amssymb,amsmath,amsthm,stmaryrd,mathrsfs,wasysym}
\usepackage{enumitem,mathtools,xspace,xcolor}
\definecolor{darkgreen}{rgb}{0,0.45,0}
\usepackage{aliascnt}
\usepackage[capitalize]{cleveref}
%\usepackage[all,2cell]{xy}
%\UseAllTwocells
% \usepackage{natbib}
\usepackage{braket} % used for \setof{ ... } macro
\usepackage{tikz-cd}
\usepackage{tikz}
\usetikzlibrary{decorations.pathmorphing}
\usepackage[inference]{semantic}
\usepackage{booktabs}

%%%%%%%%%%%%%%%%%%%%%%%%%%%%%%%%%%%%%%%%%%%%%%%%%%%%%%%%%%%%%%%%%%%%%%%%%%%%%%%%
%% To include references in TOC we should use this package rather than a hack.
\usepackage{tocbibind}
%\usepackage{etoolbox}           % get \apptocmd
%\apptocmd{\thebibliography}{\addcontentsline{toc}{section}{References}}{}{} % tell bibliography to get itself into the table of contents


\begin{comment}
%%%% Header and footers
\pagestyle{fancyplain}
\setlength{\headheight}{15pt}
\renewcommand{\chaptermark}[1]{\markboth{\textsc{Chapter \thechapter. #1}}{}}
\renewcommand{\sectionmark}[1]{\markright{\textsc{\thesection\ #1}}}
\end{comment}

% TOC depth
\setcounter{tocdepth}{2}

\lhead[\fancyplain{}{{\thepage}}]%
      {\fancyplain{}{\nouppercase{\rightmark}}}
\rhead[\fancyplain{}{\nouppercase{\leftmark}}]%
      {\fancyplain{}{\thepage}}
\cfoot{\textsc{\footnotesize [Draft of \today]}}
\lfoot[]{}
\rfoot[]{}

%%%%%%%%%%%%%%%%%%%%%%%%%%%%%%%%%%%%%%%%%%%%%%%%%%%%%%%%%%%%%%%%%%%%%%%%%%%%%%%%
%%%% We mostly use the macros of the book, to keep notations
%%%% and conventions the same. Recall that when the macros file
%%%% is updated, we need to comment the lines containing the
%%%% string `[chapter]` since our article is not a book.
%%%%
%%%% Instructions for updating the macros.tex file:
%%%% - fetch the latest macros.tex file from the HoTT/book git repository.
%%%% - comment all lines containing "[chapter]" because this is not a book.
%%%% - comment the definition of pbcorner because the xypic package is not used.
%%%%
\input{macros}

\newcommand{\idsymbin}{=}

%%%%%%%%%%%%%%%%%%%%%%%%%%%%%%%%%%%%%%%%%%%%%%%%%%%%%%%%%%%%%%%%%%%%%%%%%%%%%%%%
%%%% Our commands which are not part of the macros.tex file.
%%%% We should keep these commands separate, because we will
%%%% update the macros.tex following the updates of the book.

%%%% First we redefine the \id, \eqv and \ct commands so that they accept an
%%%% arbitrary number of arguments. This is useful when writing longer strings
%%%% of equalities or equivalences.

\makeatletter

\renewcommand{\id}[3][]{
  \@ifnextchar\bgroup
    {#2 \mathbin{\idsym_{#1}} \id[#1]{#3}}
    {#2 \mathbin{\idsym_{#1}} #3}
  }

\renewcommand{\eqv}[2]{
  \@ifnextchar\bgroup
    {#1 \eqvsym \eqv{#2}}
    {#1 \eqvsym #2}
  }

\newcommand{\ctsym}{%
  \mathchoice{\mathbin{\raisebox{0.5ex}{$\displaystyle\centerdot$}}}%
             {\mathbin{\raisebox{0.5ex}{$\centerdot$}}}%
             {\mathbin{\raisebox{0.25ex}{$\scriptstyle\,\centerdot\,$}}}%
             {\mathbin{\raisebox{0.1ex}{$\scriptscriptstyle\,\centerdot\,$}}}
  }

\renewcommand{\ct}[3][]{
  \@ifnextchar\bgroup
    {#2 \mathbin{\ctsym_{#1}} \ct[#1]{#3}}
    {#2 \mathbin{\ctsym_{#1}} #3}
  }

\makeatother

%%%% We always use textstyle products and sums...
%\renewcommand{\prd}{\tprd}
%\renewcommand{\sm}{\tsm}
\makeatletter
\renewcommand{\@dprd}{\@tprd}
\renewcommand{\@dsm}{\@tsm}
\renewcommand{\@dprd@noparens}{\@tprd}
\renewcommand{\@dsm@noparens}{\@tsm}

%%%% ...with a bit more spacing
\renewcommand{\@tprd}[1]{\mathchoice{{\textstyle\prod_{(#1)}\,}}{\prod_{(#1)}\,}{\prod_{(#1)}\,}{\prod_{(#1)}\,}}
\renewcommand{\@tsm}[1]{\mathchoice{{\textstyle\sum_{(#1)}\,}}{\sum_{(#1)}\,}{\sum_{(#1)}\,}{\sum_{(#1)}\,}}

%%%%%%%%%%%%%%%%%%%%%%%%%%%%%%%%%%%%%%%%%%%%%%%%%%%%%%%%%%%%%%%%%%%%%%%%%%%%%%%%
%%%% We adjust the \prd command so that implicit arguments become possible.
%%%%
%%%% First, we have the following switch. Set it to true if implicit arguments
%%%% are desired, or to false if not. Note turning off implicit arguments
%%%% might render some parts of the text harder to comprehend, since in the
%%%% text might appear $f(x)$ where we would have $f(i,x)$ without implicit
%%%% arguments.

\newcommand{\implicitargumentson}{\boolean{true}}

%%%% If one wants to use implicit arguments in the notation for product types,
%%%% a * has to be put before the argument that has to be implicit.
%%%% For example: in $\prd{x:A}*{y:B}{u:P(y)}Q(x,y,u)$, the argument y is
%%%% implicit. Any of the arguments can be made implicit this way.

%%%% First of all, we should make the command \prd search not only for a
%%%% brace, but also for a star. We introduce an auxiliary command that
%%%% determines whether the next character is a star or brace.
\newcommand{\@ifnextchar@starorbrace}[2]
%  {\@ifnextcharamong{#1}{#2}{*}{\bgroup};}
  {\@ifnextchar*{#1}{\@ifnextchar\bgroup{#1}{#2}}}
  
%%%% When encountering the \prd command, latex should determine whether it
%%%% should print implicit argument brackets or not. So the first branching
%%%% happens right here.
\renewcommand{\prd}{\@ifnextchar*{\@iprd}{\@prd}}

\newcommand{\@prd}[1]
  {\@ifnextchar@starorbrace
    {\prd@parens{#1}}
    {\@ifnextchar\sm{\prd@parens{#1}\@eatsm}{\prd@noparens{#1}}}}
\newcommand{\@prd@parens}{\@ifnextchar*{\@iprd}{\prd@parens}}
\renewcommand{\prd@parens}[1]
  {\@ifnextchar@starorbrace
    {\@theprd{#1}\@prd@parens}
    {\@ifnextchar\sm{\@theprd{#1}\@eatsm}{\@theprd{#1}}}}
\newcommand{\@theprd}[1]
  {\mathchoice{\@dprd{#1}}{\@tprd{#1}}{\@tprd{#1}}{\@tprd{#1}}}
\renewcommand{\dprd}[1]{\@dprd{#1}\@ifnextchar@starorbrace{\dprd}{}}
\renewcommand{\tprd}[1]{\@tprd{#1}\@ifnextchar@starorbrace{\tprd}{}}

%%%% Here we tell the actual symbols to be printed.
\newcommand{\@theiprd}[1]{\mathchoice{\@diprd{#1}}{\@tiprd{#1}}{\@tiprd{#1}}{\@tiprd{#1}}}
\newcommand{\@iprd}[2]{\@ifnextchar@starorbrace%
  {\@theiprd{#2}\@prd@parens}%
  {\@ifnextchar\sm%
    {\@theiprd{#2}\@eatsm}%
    {\@theiprd{#2}}}}
\def\@tiprd#1{
  \ifthenelse{\implicitargumentson}
    {\@@tiprd{#1}\@ifnextchar\bgroup{\@tiprd}{}}
    {\@tprd{#1}}}
\def\@@tiprd#1{\mathchoice{{\textstyle\prod_{\{#1\}}\,}}{\prod_{\{#1\}}\,}{\prod_{\{#1\}}\,}{\prod_{\{#1\}}\,}}
\def\@diprd{
  \ifthenelse{\implicitargumentson}
    {\@tiprd}
    {\@tprd}}
    

%%%% And finally we need to redefine \@eatprd so that implicit arguments also
%%%% works in the scope of a dependent sum.    
\def\@eatprd\prd{\@prd@parens}

\makeatother

%%%%%%%%%%%%%%%%%%%%%%%%%%%%%%%%%%%%%%%%%%%%%%%%%%%%%%%%%%%%%%%%%%%%%%%%%%%%%%%%
%%%% Redefining the quantifiers, so that some of the longer 
%%%% formulas appear one a single line without problems

%%% Dependent products written with \forall, in the same style
\makeatletter
\def\tfall#1{\forall_{(#1)}\@ifnextchar\bgroup{\,\tfall}{\,}}
\renewcommand{\fall}{\tfall}

%%% Existential quantifier %%%
\def\texis#1{\exists_{(#1)}\@ifnextchar\bgroup{\,\texis}{\,}}
\renewcommand{\exis}{\texis}

%%% Unique existence %%%
\def\uexis#1{\exists!_{(#1)}\@ifnextchar\bgroup{\,\uexis}{\,}}
\makeatother
%%%%%%%%%%%%%%%%%%%%%%%%%%%%%%%%%%%%%%%%%%%%%%%%%%%%%%%%%%%%%%%%%%%%%%%%%%%%%%%%

%%%% Introducing logical usage of fonts.
\newcommand{\modelfont}{\mathit} % use 'mf' in command to indicate model font
\newcommand{\typefont}{\mathsf} % use 'tf' in command to indicate type font
\newcommand{\catfont}{\mathrm} % use 'cf' in command to indicate cat font

%%%%%%%%%%%%%%%%%%%%%%%%%%%%%%%%%%%%%%%%%%%%%%%%%%%%%%%%%%%%%%%%%%%%%%%%%%%%%%%%
%%%% Some macros of the book are redefined.

\renewcommand{\UU}{\typefont{U}}
\renewcommand{\isequiv}{\typefont{isEquiv}}
\renewcommand{\happly}{\typefont{hApply}}
\renewcommand{\pairr}[1]{{\mathopen{}\langle #1\rangle\mathclose{}}}
\renewcommand{\type}{\typefont{Type}}
\renewcommand{\op}[1]{{{#1}^\typefont{op}}}
\renewcommand{\susp}{\typefont{\Sigma}}

%%%%%%%%%%%%%%%%%%%%%%%%%%%%%%%%%%%%%%%%%%%%%%%%%%%%%%%%%%%%%%%%%%%%%%%%%%%%%%%%
%%%% The following is a big unorganized list of new macros that we use in the
%%%% notes. 

\newcommand{\mfM}{\modelfont{M}}
\newcommand{\mfN}{\modelfont{N}}
\newcommand{\tfctx}{\typefont{ctx}}
\newcommand{\mftypfunc}[1]{{\modelfont{typ}^{#1}}}
\newcommand{\mftyp}[2]{{\mftypfunc{#1}(#2)}}
\newcommand{\tftypfunc}[1]{{\typefont{typ}^{#1}}}
\newcommand{\tftyp}[2]{{\tftypfunc{#1}(#2)}}
\newcommand{\hfibfunc}[1]{\typefont{fib}_{#1}}
\newcommand{\mappingcone}[1]{\mathcal{C}_{#1}}
\newcommand{\equifib}{\typefont{equiFib}}
\newcommand{\tfcolim}{\typefont{colim}}
\newcommand{\tflim}{\typefont{lim}}
\newcommand{\tfdiag}{\typefont{diag}}
\newcommand{\tfGraph}{\typefont{Graph}}
\newcommand{\mfGraph}{\modelfont{Graph}}
\newcommand{\unitGraph}{\unit^\mfGraph}
\newcommand{\UUGraph}{\UU^\mfGraph}
\newcommand{\tfrGraph}{\typefont{rGraph}}
\newcommand{\mfrGraph}{\modelfont{rGraph}}
\newcommand{\isfunction}{\typefont{isFunction}}
\newcommand{\tfconst}{\typefont{const}}
\newcommand{\conemap}{\typefont{coneMap}}
\newcommand{\coconemap}{\typefont{coconeMap}}
\newcommand{\tflimits}{\typefont{limits}}
\newcommand{\tfcolimits}{\typefont{colimits}}
\newcommand{\islimiting}{\typefont{isLimiting}}
\newcommand{\iscolimiting}{\typefont{isColimiting}}
\newcommand{\islimit}{\typefont{isLimit}}
\newcommand{\iscolimit}{\typefont{iscolimit}}
\newcommand{\pbcone}{\typefont{cone_{pb}}}
\newcommand{\tfinj}{\typefont{inj}}
\newcommand{\tfsurj}{\typefont{surj}}
\newcommand{\tfepi}{\typefont{epi}}
\newcommand{\tftop}{\typefont{top}}
\newcommand{\sbrck}[1]{\Vert #1\Vert}
\newcommand{\strunc}[2]{\Vert #2\Vert_{#1}}
\newcommand{\gobjclass}{{\typefont{U}^\mfGraph}}
\newcommand{\gcharmap}{\typefont{fib}}
\newcommand{\diagclass}{\typefont{T}}
\newcommand{\opdiagclass}{\op{\diagclass}}
\newcommand{\equifibclass}{\diagclass^{\eqv{}{}}}
\newcommand{\universe}{\typefont{U}}
\newcommand{\catid}[1]{{\catfont{id}_{#1}}}
\newcommand{\isleftfib}{\typefont{isLeftFib}}
\newcommand{\isrightfib}{\typefont{isRightFib}}
\newcommand{\leftLiftings}{\typefont{leftLiftings}}
\newcommand{\rightLiftings}{\typefont{rightLiftings}}
\newcommand{\psh}{\typefont{Psh}}
\newcommand{\rgclass}{\typefont{\Omega^{RG}}}
\newcommand{\terms}[2][]{\lfloor #2 \rfloor^{#1}}
\newcommand{\grconstr}[2]
             {\mathchoice % max size is textstyle size.
             {{\textstyle \int_{#1}}#2}% 
             {\int_{#1}#2}%
             {\int_{#1}#2}%
             {\int_{#1}#2}}
\newcommand{\ctxhom}[3][]{\typefont{hom}_{#1}(#2,#3)}
\newcommand{\graphcharmapfunc}[1]{\gcharmap_{#1}}
\newcommand{\graphcharmap}[2][]{\graphcharmapfunc{#1}(#2)}
\newcommand{\tfexp}[1]{\typefont{exp}_{#1}}
\newcommand{\tffamfunc}{\typefont{fam}}
\newcommand{\tffam}[1]{\tffamfunc(#1)}
\newcommand{\tfev}{\typefont{ev}}
\newcommand{\tfcomp}{\typefont{comp}}
\newcommand{\isDec}[1]{\typefont{isDecidable}(#1)}
\newcommand{\smal}{\mathcal{S}}
\renewcommand{\modal}{{\ensuremath{\ocircle}}}
\newcommand{\eqrel}{\typefont{EqRel}}
\newcommand{\piw}{\ensuremath{\Pi\typefont{W}}} %% to be used in conjunction with -pretopos.
\renewcommand{\sslash}{/\!\!/}
\newcommand{\mprd}[2]{\Pi(#1,#2)}
\newcommand{\msm}[2]{\Sigma(#1,#2)}
\newcommand{\midt}[1]{\idvartype_#1}
\newcommand{\reflf}[1]{\typefont{refl}^{#1}}
\newcommand{\tfJ}{\typefont{J}}
\newcommand{\tftrans}{\typefont{trans}}

\newcommand{\tfT}{\typefont{T}}
\newcommand{\reflsym}{{\mathsf{refl}}}
\newcommand{\strans}[2]{\ensuremath{{#1}_{*}({#2})}}

%%%%%%%%%%%%%%%%%%%%%%%%%%%%%%%%%%%%%%%%%%%%%%%%%%%%%%%%%%%%%%%%%%%%%%%%%%%%%%%%
%%%% JUDGMENTS
%%%%
%%%% Below we define several commands for the judgments of type theory. There
%%%% are commands
%%%% * \jctx for the judgment that something is a context.
%%%% * \jctxeq for the judgment that two contexts are the same
%%%% * \jtype for the judgment that something is a type in a context
%%%% * \jtypeeq for the judgment that two types in the same context are the same
%%%% * \jterm for the judgment that something is a term of a type in a context
%%%% * \jtermeq for the judgment that two terms of the same type are the same

\makeatletter
\newcommand{\jctx}{\@ifnextchar*{\@jctxAlignTrue}{\@jctxAlignFalse}}
\newcommand{\@jctxAlignTrue}[2]{& \vdash #2~ctx}
\newcommand{\@jctxAlignFalse}[1]{\vdash #1~ctx}

\newcommand{\jtype}{\@ifnextchar*{\@jtypeAlignTrue}{\@jtypeAlignFalse}}
\newcommand{\@jtypeAlignFalse}[2]{#1\vdash #2~type}
\newcommand{\@jtypeAlignTrue}[3]{#2 & \vdash #3~type}

\newcommand{\jtermc}{\@ifnextchar*{\@jtermcAlignTrue}{\@jtermcAlignFalse}}
\newcommand{\@jtermcAlignTrue}[3]{ & \vdash #3:#2}
\newcommand{\@jtermcAlignFalse}[2]{\vdash #2:#1}

\newcommand{\jtermt}{\@ifnextchar*{\@jtermtAlignTrue}{\@jtermtAlignFalse}}
\newcommand{\@jtermtAlignTrue}[4]{#2 & \vdash #4:#3}
\newcommand{\@jtermtAlignFalse}[3]{#1 \vdash #3:#2}

\newcommand{\jctxeq}{\@ifnextchar*{\@jctxeqAlignTrue}{\@jctxeqAlignFalse}}
\newcommand{\@jctxeqAlignTrue}[3]{& \vdash #2\jdeq #3~ctx}
\newcommand{\@jctxeqAlignFalse}[2]{\vdash #1\jdeq #2~ctx}

\newcommand{\jtypeeq}{\@ifnextchar*{\@jtypeeqAlignTrue}{\@jtypeeqAlignFalse}}
\newcommand{\@jtypeeqAlignTrue}[4]{#2 & \vdash #3\jdeq #4~type}
\newcommand{\@jtypeeqAlignFalse}[3]{#1\vdash #2\jdeq #3~type}

\newcommand{\jtermceq}{\@ifnextchar*{\@jtermceqAlignTrue}{\@jtermceqAlignFalse}}
\newcommand{\@jtermceqAlignTrue}[4]{& \vdash #3\jdeq #4:#2}
\newcommand{\@jtermceqAlignFalse}[3]{\vdash #2\jdeq #3:#1}

\newcommand{\jtermteq}{\@ifnextchar*{\@jtermteqAlignTrue}{\@jtermteqAlignFalse}}
\newcommand{\@jtermteqAlignTrue}[5]{#2 & \vdash #4\jdeq #5:#3}
\newcommand{\@jtermteqAlignFalse}[4]{#1\vdash #3\jdeq #4:#2}
\makeatother

%%%%%%%%%%%%%%%%%%%%%%%%%%%%%%%%%%%%%%%%%%%%%%%%%%%%%%%%%%%%%%%%%%%%%%%%%%%%%%%%
%%%% Often we shall need to display lists of inference rules. This environment
%%%% adjusts the array environment so that there is enough vertical space
%%%% between two inference rules
%%%%
%%%% bug: there's two much space above the array.

\newenvironment{infarray}[1]{\begingroup\renewcommand*{\arraystretch}{3}
\begin{equation*}
\begin{array}{#1}}{
\end{array}
\end{equation*}
\endgroup}

%%%%%%%%%%%%%%%%%%%%%%%%%%%%%%%%%%%%%%%%%%%%%%%%%%%%%%%%%%%%%%%%%%%%%%%%%%%%%%%%
%%%% CONTEXT EXTENSION 
%%%%
%%%% explicit context extension notation which we will use only rarely

\newcommand{\tfext}{\typefont{ext}}

%%%% The context extension command.
%%%%
%%%% To get a feeling of how the command works, here are a few examples.
%%%% \ctxext{A}{B} will print A.B
%%%% \ctxext{{A}{B}}{C} will print (A.B).C
%%%% \ctxext{{{A}{B}}{C}}{{D}{E}} will print ((A.B).C).(D.E)

\makeatletter
\newcommand{\ctxext}[2]{\@ctxext@ctx #1.\@ctxext@type #2}
\newcommand{\@ctxext}{\@ifnextchar\bgroup{\@@ctxext}{}}
\newcommand{\@ctxext@ctx}{\@ifnextchar\ctxext{\@ctxext@nested}{\@ifnextchar\ctxwk{\@ctxwk@nested}{\@ctxext}}}
\newcommand{\@ctxext@type}{\@ifnextchar\ctxext{\@ctxext@nested}{\@ifnextchar\subst{\@subst@nested}{\@ctxext}}}
\newcommand{\@@ctxext}[1]{\@ifnextchar\bgroup{\@ctxext@parens{#1}}{#1}}
\newcommand{\@ctxext@parens}[2]{(\ctxext{#1}{#2})}
\newcommand{\@ctxext@nested}[3]{\@ctxext@parens{#2}{#3}}
\makeatother

%%%%%%%%%%%%%%%%%%%%%%%%%%%%%%%%%%%%%%%%%%%%%%%%%%%%%%%%%%%%%%%%%%%%%%%%%%%%%%%%
%%%% SUBSTITUTION

\newcommand{\tfsubst}{\typefont{subst}}

%%%% The substitution command will act the following way
%%%%
%%%% \subst{x}{P} will print P[x]
%%%% \subst{x}{{f}{Q}} will print Q[f][x]
%%%% \subst{{x}{f}}{{x}{Q}} will print Q[x][f[x]]

\makeatletter
\newcommand{\subst}[2]{\@subst@type #2[\@subst@term #1]}
\newcommand{\@subst}{\@ifnextchar\bgroup{\@@subst}{}}
\newcommand{\@@subst}[1]{\@ifnextchar\bgroup{\subst{#1}}{#1}}
\newcommand{\@subst@term}{\@subst}
\newcommand{\@subst@type}{\@ifnextchar\ctxext{\@ctxext@nested}{\@ifnextchar\ctxwk{\@ctxwk@nested}{\@subst}}}
\newcommand{\@subst@nested}[3]{\@subst@parens{#2}{#3}}
\newcommand{\@subst@parens}[2]{(\subst{#1}{#2})}
\makeatother

%%%%%%%%%%%%%%%%%%%%%%%%%%%%%%%%%%%%%%%%%%%%%%%%%%%%%%%%%%%%%%%%%%%%%%%%%%%%%%%%
%%%% WEAKENING

\newcommand{\tfwk}{\typefont{wk}}

%%%% The weakening command is very much like the substitution command.

\makeatletter
\newcommand{\ctxwk}[2]{\langle\@ctxwk@act #1\rangle\@ctxwk@pass #2}
\newcommand{\@ctxwk}{\@ifnextchar\bgroup{\@@ctxwk}{}}
\newcommand{\@@ctxwk}[1]{\@ifnextchar\bgroup{\ctxwk{#1}}{#1}}
\newcommand{\@ctxwk@act}{\@ctxwk}
\newcommand{\@ctxwk@pass}{\@ifnextchar\ctxext{\@ctxext@nested}{\@ifnextchar\subst{\@subst@nested}{\@ctxwk}}}
\newcommand{\@ctxwk@parens}[2]{(\ctxwk{#1}{#2})}
\newcommand{\@ctxwk@nested}[3]{\@ctxwk@parens{#2}{#3}}
\makeatother

%%%%%%%%%%%%%%%%%%%%%%%%%%%%%%%%%%%%%%%%%%%%%%%%%%%%%%%%%%%%%%%%%%%%%%%%%%%%%%%%
%%%% When investigation pointed structures we use the \pt macro.

\makeatletter
\newcommand{\pt}[1][]{*_{
  \@ifnextchar\undergraph{\@undergraph@nested}
    {\@ifnextchar\underovergraph{\@underovergraph@nested}{}}#1}}
\makeatother

%%%%%%%%%%%%%%%%%%%%%%%%%%%%%%%%%%%%%%%%%%%%%%%%%%%%%%%%%%%%%%%%%%%%%%%%%%%%%%%%
%%%% OPERATIONS ON GRAPHS
%%%%
%%%% First of all, each graph has a type of vertices and a type of edges. The
%%%% type of vertices of a graph $\Gamma$ is denoted by $\pts{\Gamma}$;
%%%% and likewise for the type of edges.

\makeatletter
\newcommand{\pts}[1]{{\@graphop@nested{#1}}_{0}}
\newcommand{\edg}[1]{{\@graphop@nested{#1}}_{1}}
\newcommand{\@graphop@nested}[1]
  {\@ifnextchar\ctxext{\@ctxext@nested}
      {\@ifnextchar\undergraph{\@undergraph@nested}
         {\@ifnextchar\underovergraph{\@underovergraph@nested}{}}}
    #1}
\makeatother

%%%% The following operations of \undergraph and \underovergraph are used to
%%%% define the free category and the free groupoid of a graph, respectively

\makeatletter
\newcommand{\@undergraphtest}[2]{\@ifnextchar({#1}{#2}}
\newcommand{\undergraph}[2]{\@undergraphtest{\@undergraph@parens{#1}{#2}}{\@undergraph{#1}{#2}}}
\newcommand{\@undergraph}[2]{{#2/#1}}
\newcommand{\@undergraph@nested}[3]{\@undergraph@parens{#2}{#3}}
\newcommand{\@undergraph@parens}[2]{(\@undergraph{#1}{#2})}
\makeatother

\makeatletter
\newcommand{\underovergraph}[2]{\@underovergraphtest{\@underovergraph@parens{#1}{#2}}{\@underovergraph{#1}{#2}}}
\newcommand{\@underovergraph}[2]{{#2}\,{\parallel}\,{#1}}
\newcommand{\@underovergraphtest}{\@undergraphtest}
\newcommand{\@underovergraph@parens}[2]{(\@underovergraph{#1}{#2})}
\newcommand{\@underovergraph@nested}[3]{\@underovergraph@parens{#2}{#3}}
\makeatother

\newcommand{\graphid}[1]{\mathrm{id}_{#1}}
\newcommand{\freecat}[1]{\mathcal{C}(#1)}
\newcommand{\freegrpd}[1]{\mathcal{G}(#1)}

%%%%%%%%%%%%%%%%%%%%%%%%%%%%%%%%%%%%%%%%%%%%%%%%%%%%%%%%%%%%%%%%%%%%%%%%%%%%%%%%
%% Some tikz macros to typeset diagrams uniformly.

\tikzset{patharrow/.style={double,double equal sign distance,-,font=\scriptsize}}
\tikzset{description/.style={fill=white,inner sep=2pt}}

%% Used for extra wide diagrams, e.g. when the label is too large otherwise.
\tikzset{commutative diagrams/column sep/Huge/.initial=18ex}

%%%%%%%%%%%%%%%%%%%%%%%%%%%%%%%%%%%%%%%%%%%%%%%%%%%%%%%%%%%%%%%%%%%%%%%%%%%%%%%%
%%%% New theorem environment for conjectures.

\defthm{conj}{Conjecture}{Conjectures}

%%%%%%%%%%%%%%%%%%%%%%%%%%%%%%%%%%%%%%%%%%%%%%%%%%%%%%%%%%%%%%%%%%%%%%%%%%%%%%%%
%%%% The following environment for desiderata should not be there. It is better
%%%% to use the issue tracker for desiderata.

\newenvironment{desiderata}{\begingroup\color{blue}\textbf{Desiderata.}}
{\endgroup}


%%%%%%%%%%%%%%%%%%%%%%%%%%%%%%%%%%%%%%%%%%%%%%%%%%%%%%%%%%%%%%%%%%%%%%%%%%%%%%%%
%%%% We define a command \@ifnextcharamong accepting an arbitrary number of
%%%% arguments. The first is what it should do if a match is found, the second
%%%% contains what it should do when no match is found; all the other arguments
%%%% are the things it tries to find as the next character.
%%%%
%%%% For example \@ifnextcharamong{#1}{#2}{*}{\bgroup} expands #1 if the next
%%%% character is a * or a \bgroup and it expands #2 otherwise.

\makeatletter
\newcommand{\@ifnextcharamong}[2]
  {\@ifnextchar\bgroup{\@@ifnextchar{#1}{\@@ifnextcharamong{#1}{#2}}}{#2}}
\newcommand{\@@ifnextchar}[3]{\@ifnextchar{#3}{#1}{#2}}
\newcommand{\@@ifnextcharamong}[3]{\@ifnextcharamong{#1}{#2}}
\makeatother

\newcommand{\ucomp}[1]{\hat{#1}}
\newcommand{\finset}[1]{{[#1]}}

\makeatletter
\newcommand{\higherequifibsf}{\mathcal}
\newcommand{\higherequifib}[2]{\higherequifibsf{#1}(#2)}
\newcommand{\underlyinggraph}[1]{U(#1)}
\newcommand{\theequifib}[1]{{\def\higherequifibsf{}#1}}
\makeatother

\newcommand{\loopspace}[2][]{\typefont{\Omega}^{#1}(#2)}
\newcommand{\join}[2]{{#1}*{#2}}

%%%%%%%%%%%%%%%%%%%%%%%%%%%%%%%%%%%%%%%%%%%%%%%%%%%%%%%%%%%%%%%%%%%%%%%%%%%%%%%%
\title{Notes on algebraic topology}
\date{\today}

\begin{document}

\maketitle

\tableofcontents

\part{Spectral sequences}
\section{Background}
\begin{defn}
A graded $R$-module $M$ is an $R$-module which decomposes as a direct
sum
\begin{equation*}
\bigoplus_{p\in\Z} F_p M
\end{equation*}
of $R$-modules. A graded $R$-homomorphism $h:M\to N$ is an $R$-homomorphism which
decomposes into $h_p:F_pM\to F_pN$. 
\end{defn}

\begin{lem}
Suppose $M$ and $N$ are graded $R$-modules. Then $M\otimes N$ is a graded
$R$-module by
\begin{equation*}
(M\otimes_R N)_i\defeq \bigoplus_{p+q=i} F_pM\otimes_R F_qN.
\end{equation*}
\end{lem}

\begin{defn}
A graded algebra is a graded $R$-module $M$ for which there are linear mappings
$\varphi_{p,q}:F_pM\otimes_R F_qM\to F_{p+q}M$, i.e.~a graded $R$-homomorphism
$\varphi:M\otimes M\to M$, which is associative in the sense
that the diagram
\begin{equation*}
\begin{tikzcd}
M\otimes M\otimes M \arrow[r,"\varphi\otimes 1"] \arrow[d,swap,"1\otimes\varphi"] &
M\otimes M \arrow[d,"\varphi"] \\ M\otimes M \arrow[r,swap,"\varphi"] & M
\end{tikzcd}
\end{equation*}
commutes.
\end{defn}

\begin{eg}
Polynomials with coefficients in $R$ forms a graded algebra. Moreover, in the
polynomial ring $R[X]$, we find that $G_pR[X]\defeq F_pR[X]/F_{p-1}R[X]\cong R$.
Since those are free modules, we have that $R[X]\cong \bigoplus_p G_pR[X]$. 
\end{eg}

\section{Spectral sequences}

\subsection{Motivation from the long exact sequence of a pair}

Recall that a pair of spaces $(X,A)$ induces a long exact sequence of homology
groups
\begin{equation*}
\begin{tikzcd}
\cdots \arrow[r,"\partial_{n+1}"]
& H_n(A) \arrow[r,"i_n"]
& H_n(X) \arrow[r,"j_n"]
& H_n(X,A) \arrow[r,"\partial_n"]
& H_{n-1}(A) \arrow[r,"i_{n-1}"]
& \cdots
\end{tikzcd}
\end{equation*}
from the short exact sequence
\begin{equation*}
\begin{tikzcd}
0 \arrow[r] & C_\ast(A) \arrow[r] & C_\ast(X) \arrow[r] & C_\ast(X,A) \arrow[r] & 0
\end{tikzcd}
\end{equation*}
of chain complexes, by means of the snake lemma. This long exact sequence helps
us to compute $H_n(X)$ in terms of $H_n(A)$ and $H_n(X,A)$, which may be easier
to determine. For instance, from the long exact sequence we obtain the short
exact sequence
\begin{equation*}
\begin{tikzcd}
0 \arrow[r] & \mathrm{coker}(\partial_{n+1}) \arrow[r] & H_n(X) \arrow[r] & \mathrm{ker}(\partial_n) \arrow[r] & 0
\end{tikzcd}
\end{equation*}
and hence we have determined that $H_n(X)$ can be obtained as some element of the
group $\mathrm{Ext}(\mathrm{coker}(\partial_{n+1}),\mathrm{ker}(\partial_n))$.
In other words, $H_n(X)$ is a particular solution to an extension problem.

Note also that the long exact sequence of relative homology groups can be
presented as an exact triangle of graded $R$-homomorphisms:
\begin{equation*}
\begin{tikzcd}[column sep=0em]
\bigoplus_n H_n(C_\ast(A)) 
\arrow[rr,"i"] & & \bigoplus_n H_n(C_\ast(X)) \arrow[dl,"j"] \\
& \bigoplus_n H_n(C_\ast(X,A)) \arrow[ul,"\partial"]
\end{tikzcd}
\end{equation*}

The first idea of spectral sequences is to generalize the long exact sequence
of homology obtained from a pair of spaces, to an algebraic gadget obtained from
a filtration on a space, and mimic the derivation of determining the homology
group as a solution to an extension problem.

\begin{defn}
A filtration of a space X consists of a sequence 
\begin{equation*}
\cdots\subseteq X_p\subseteq X_{p+1}\subseteq\cdots
\end{equation*}
such that $X=\bigcup_p X_p$ and $\bigcap_p X_p=\varnothing$. A filtration of $X$ is said to be bounded, if
$X_p=\varnothing$ for $p$ sufficiently small, and $X_p=X$ for $X$ sufficiently
large.
\end{defn}

An important class of filtered spaces is that of CW-complexes, where the filtration
$X_p$ of $X$ is given by the $p$-skeleton of $X$. Another case is where
$X_p\defeq\varnothing$ for $p<0$, $X_0\defeq A$ and $X_p\defeq X$ for $p>0$; here
we recover the old theory of the topological pair.

\begin{defn}
Given a space $X$ with a filtration, we can form the staircase diagram
\begin{footnotesize}
\begin{equation*}
\begin{tikzcd}
& \vdots \arrow[d] & & \vdots \arrow[d] \\ 
\cdots \arrow[r]
& H_{n+1}(X_p) \arrow[r] \arrow[d]
& H_{n+1}(X_p,X_{p-1}) \arrow[r]
& H_n(X_{p-1}) \arrow[r] \arrow[d]
& H_n(X_{p-1},X_{p-2}) \arrow[r]
& \cdots \\
\cdots \arrow[r]
& H_{n+1}(X_{p+1}) \arrow[r] \arrow[d]
& H_{n+1}(X_{p+1},X_{p}) \arrow[r]
& H_n(X_{p}) \arrow[r] \arrow[d]
& H_n(X_{p},X_{p-1}) \arrow[r]
& \cdots \\
& \vdots & & \vdots
\end{tikzcd}
\end{equation*}%
\end{footnotesize}%
in which the familiar long exact sequence of the pairs $(X_p,X_{p-1})$ run
down like a staircase.
\end{defn}

\begin{defn}
Let $X$ be a space with a filtration. Then we obtain the exact couple
\begin{equation*}
\begin{tikzcd}
A \arrow[rr,"i"] & & A \arrow[dl,"j"] \\
& E \arrow[ul,"\partial"]
\end{tikzcd}
\end{equation*}
in which $A\defeq\bigoplus_{p,n} H_n(X_p)$, and $E\defeq\bigoplus_{p,n}H_n(X_p,X_{p-1})$. 
\end{defn}

We can come to such an exact couple from any filtered chain complex, which is
what we turn our attention to before continuing.

\subsection{The spectral sequence of a filtered complex}

\begin{defn}
A filtration of an $R$-module $M$ consists of a sequence
\begin{equation*}
\cdots\subseteq F_pM\subseteq F_{p+1}M\subseteq\cdots
\end{equation*}
of $R$-submodules of $M$, such that $M=\bigcup_p F_pM$ and $\bigcap_p F_pM=0$. 
A filtration of $R$ is said to be bounded if $F_pM=0$ for $p$ sufficiently
small and $F_pM=M$ for $p$ sufficiently large.
\end{defn}

\begin{defn}
Let $\{M,F_pM\}$ be a graded $R$-module. The associated graded module is defined
by $G_p M\defeq F_pM/F_{p-1}M$. We obtain a short exact sequence
\begin{equation*}
\begin{tikzcd}
0 \arrow[r] & F_{p-1}M \arrow[r] & F_pM \arrow[r] & G_pM \arrow[r] & 0.
\end{tikzcd}
\end{equation*}
\end{defn}

\begin{rmk}
It would be nice if $F_pM\cong F_{p-1}M\oplus G_pM$, so that we can write
$M\cong\bigoplus_p G_pM$. Under what condition does this hold? This holds if
each $G_pM$ is a projective $R$-module, so under what conditions is this true?
\end{rmk}

\begin{defn}
A filtered chain complex is a chain complex $(C_\ast,\partial)$ together with a
filtration $\{F_pC_i\}$ of each $C_i$, such that the differential preserves the
filtration, i.e.~$\partial(F_pC_i)\subseteq F_p C_{i-1}$. 

A filtration of a chain complex is said to be bounded if it is bounded in each
dimension.
\end{defn}

Let $(F_pC_\ast,\partial)$ be a filtered chain complex. We have again our
short exact sequence
\begin{equation*}
\begin{tikzcd}
0 \arrow[r] & F_{p-1} C_\ast \arrow[r] & F_p C_\ast \arrow[r] & G_p C_\ast \arrow[r] & 0
\end{tikzcd}
\end{equation*}
of chain complexes. This also gives us the long exact sequence on homology,
which we may express conveniently as the exact couple
\begin{equation*}
\begin{tikzcd}[column sep=0em]
\bigoplus_{p,q} H_{p+q}(F_pC_\ast) \arrow[rr,"i"] & & \bigoplus_{p,q} H_{p+q}(F_pC_\ast) \arrow[dl,"j"] \\
& \bigoplus_{p,q} H_{p+q}(G_p C_\ast) \arrow[ul,"k"]
\end{tikzcd}
\end{equation*}
consisting of graded $R$-homomorphisms (of which $k$ shifts in degree).

\begin{defn}
Consider an exact couple, i.e.~a commutative triangle
\begin{equation*}
\begin{tikzcd}
A \arrow[rr,"i"] & & A \arrow[dl,"j"] \\ & E \arrow[ul,"k"]
\end{tikzcd}
\end{equation*}
of $R$-modules, which is exact at every vertex. Taking $\partial^0\defeq j\circ k$,
we see that $(\partial^0)^2=0$ by exactness. We may now form the derived exact couple
\begin{equation*}
\begin{tikzcd}[column sep=0]
\mathrm{im}(i) \arrow[rr,"i'"] & & \mathrm{im}(i) \arrow[dl,"j'"] \\
& \frac{\mathrm{ker}(\partial)}{\mathrm{im}(\partial)} \arrow[ul,"k'"]
\end{tikzcd}
\end{equation*}
where
\begin{align*}
i'(i(a)) & \defeq i(i(a)) \\
j'(i(a)) & \defeq [j(a)] \\
k'([e]) & \defeq k(e)
\end{align*}
\end{defn}

\begin{rmk}
Since quotients commute with direct sums (both are colimits), it follows that
\begin{equation*}
E'\defeq \frac{\mathrm{ker}(\partial)}{\mathrm{im}(\partial)}
  \cong
\bigoplus_{p,q} \frac{\mathrm{ker}(\partial^0_{p,q})}{\mathrm{im}(\partial^0_{p,q+1})}
\end{equation*}
is a graded $R$-module. In other words, $E'$ is a direct sum of the homology
groups of the $p$-indexed family of chain complexes
\begin{equation*}
\begin{tikzcd}
\cdots \arrow[r] & E_{p,q}^0 \arrow[r,"{\partial^0_{p,q}}"] & E_{p,q-1}^0 \arrow[r] & \cdots
\end{tikzcd}
\end{equation*}
It follows that $i'$, $j'$ and $k'$ are graded
whenever $i$, $j$ and $k$ are, where $k'$ shifts down in dimension the same way 
$k$ does.
\end{rmk}

\begin{comment}
\begin{defn}
We define
\begin{equation*}
E_{p,q}^0\defeq G_pC_{p+q}\defeq F_pC_{p+1}/F_{p-1}C_{p+q},
\end{equation*}

Since the differential preserves the filtration, we obtain from the differentials
well-defined $R$-homomorphisms functioning as the boundary maps in the chain complex

\end{defn}

\begin{defn}
The homology groups 
\begin{equation*}
E^1_{p,q}\defeq \mathrm{ker}(\partial^0_{p,q})/\mathrm{im}(\partial^0_{p,q+1})
\end{equation*}
form again a chain complex, with boundary maps $\partial^1_{p,q}:E^1_{p,q}\to
E^1_{p,q-1}$. Thus, this process may be repeated indefinitely.
\end{defn}
\end{comment}

\begin{comment}
\begin{lem}
Let $(C_\ast,\partial)$ be a filtered chain complex. Then there is a filtration
on the homology of $C_\ast$, given by
\begin{equation*}
F_pH_i(C_\ast)\defeq\{\alpha\in H_i(C_\ast)\mid \exists_{(x\in F_p C_i)}\,\alpha=[x]\}.
\end{equation*}
\end{lem}
\end{comment}

\subsection{Convergent spectral sequences}

\begin{defn}
A spectral sequence consists of
\begin{enumerate}
\item An $R$-module $E^r_{p,q}$ for each $p,q\in\Z$ and each $r\geq 0$.
\item Differentials $\partial_r:E^r_{p,q}\to E^r_{p-r,q+r-1}$ such that
$\partial_r^2=0$ and $E^{r+1}$ is the homology of $(E^r,\partial_r)$ 
\end{enumerate}
\end{defn}

\begin{defn}
A spectral sequence $\{E^r,\partial_r\}$ of $R$-modules is said to converge 
if for every $p,q\in\Z$, one has $\partial_r=0:E^r_{p,q}\to E^r_{p-r,q+r-1}$
for $r$ sufficiently large.
\end{defn}

\begin{rmk}
If a spectral sequence $\{E^r,\partial_r\}$ converges, then the $R$-module
$E^r_{p,q}$ is independent of $r$ for sufficiently large $r$. 
\end{rmk}

\begin{thm}
Let $(F_pC_\ast,\partial)$ be a filtered complex. Then we obtain a spectral
sequence $(E^r_{p,q},\partial^r)$ defined for $r\geq 0$, with
\begin{equation*}
E^1_{p,q}\defeq H_{p+q}(G_pC_\ast).
\end{equation*}
This is the spectral sequence of filtered complexes.
\end{thm}

\begin{thm}
If $(F_pC_\ast,\partial)$ is a bounded filtered complex, then the spectral
sequence converges to
\begin{equation*}
E^\infty_{p,q}\defeq G_pH_{p+q}(C_\ast).
\end{equation*}
\end{thm}

Let $X$ be a filtered space, and let our goal be to compute the $n$-th (co)homology
group $H_n(X)$. In general, this might be a complicated task. However, it might
be easier to compute the homologies of the subcomplex $C_\ast(X_p)$, and the quotient
complex $C_\ast(X)/C_\ast(X_p)$. From this, we obtain a short exact sequence
\begin{equation*}
\begin{tikzcd}
0 \arrow[r]
& \mathrm{coker}(\delta) \arrow[r]
& H_\ast(X) \arrow[r]
& \mathrm{ker}(\delta) \arrow[r]
& 0
\end{tikzcd}
\end{equation*}

\subsection{The Serre spectral sequence}

The Serre spectral sequence relates the homology of a Serre fibration to the
homology of the fibers and the base. Thus, in some cases one can compute the
homology of the fibration in terms of the homology of the fibers and the base.

Let $\pi : X\to B$ be a fibration, with $B$ a path-connected CW-complex, and we
filter $X$ by the subspaces $X_p\defeq \pi^{-1}(B_p)$, in which $B_p$ is the
$p$-skeleton of $B$. 

\begin{lem}
The spectral sequence for homology with coefficients in $G$ associated to this
filtration of $X$ converges to $H_\ast(X;G)$.
\end{lem}

\begin{thm}
Let $F\to X\to B$ be a fibration with $B$ path-connected. If $\pi_1(B)$ acts
trivially on $H_\ast(F;G)$, then there is a spectral sequence $\{E^r_{p,q},\partial_r\}$
with:
\begin{enumerate}
%\item $\partial_r : E^r_{p,q}\to E^r_{p-r,q+r-1}$ and $E^{r+1}_{p,q}=\mathrm{ker}\,d_r/\mathrm{im}\,dr$. 
\item the stable terms $E^\infty_{p,n-p}$ are isomorphic to $F^p_n/F^{p-1}_n$ in
a filtration $0\subseteq F^0_n\subseteq\cdots\subseteq F^n_n=H_n(X;G)$ of ...
\item $E^2_{p,q}\cong H_p(B;H_q(F;G))$. 
\end{enumerate}
\end{thm}

\end{document}
