\documentclass{article}

% for type setting urls
\usepackage[hyphens]{url} % This package has to be loaded *before* hyperref
\usepackage[pagebackref,colorlinks,citecolor=darkgreen,linkcolor=darkgreen,unicode]{hyperref}
\usepackage[english]{babel}

%%% Because Germans have umlauts and Slavs have even stranger ways of mangling letters
\usepackage[utf8]{inputenc}

%%% Multi-Columns for long lists of names
\usepackage{multicol}

%%% Set the fonts
\usepackage{mathpazo}
\usepackage[scaled=0.95]{helvet}
\usepackage{courier}
\linespread{1.05} % Palatino looks better with this

\usepackage{graphicx}
\usepackage{comment}

%\usepackage{wallpaper} % For the background image on the cover page
%\usepackage{geometry} % For the cover page
\usepackage{fancyhdr} % To set headers and footers

\usepackage{ifthen}
\usepackage{amssymb,amsmath,amsthm,stmaryrd,mathrsfs,wasysym}
\usepackage{enumitem,mathtools,xspace,xcolor}
\definecolor{darkgreen}{rgb}{0,0.45,0}
\usepackage{aliascnt}
\usepackage[capitalize]{cleveref}
%\usepackage[all,2cell]{xy}
%\UseAllTwocells
% \usepackage{natbib}
\usepackage{braket} % used for \setof{ ... } macro
\usepackage{tikz-cd}
\usepackage{tikz}
\usetikzlibrary{decorations.pathmorphing}
\usepackage[inference]{semantic}
\usepackage{booktabs}

%%%%%%%%%%%%%%%%%%%%%%%%%%%%%%%%%%%%%%%%%%%%%%%%%%%%%%%%%%%%%%%%%%%%%%%%%%%%%%%%
%% To include references in TOC we should use this package rather than a hack.
\usepackage{tocbibind}
%\usepackage{etoolbox}           % get \apptocmd
%\apptocmd{\thebibliography}{\addcontentsline{toc}{section}{References}}{}{} % tell bibliography to get itself into the table of contents


\begin{comment}
%%%% Header and footers
\pagestyle{fancyplain}
\setlength{\headheight}{15pt}
\renewcommand{\chaptermark}[1]{\markboth{\textsc{Chapter \thechapter. #1}}{}}
\renewcommand{\sectionmark}[1]{\markright{\textsc{\thesection\ #1}}}
\end{comment}

% TOC depth
\setcounter{tocdepth}{2}

\lhead[\fancyplain{}{{\thepage}}]%
      {\fancyplain{}{\nouppercase{\rightmark}}}
\rhead[\fancyplain{}{\nouppercase{\leftmark}}]%
      {\fancyplain{}{\thepage}}
\cfoot{\textsc{\footnotesize [Draft of \today]}}
\lfoot[]{}
\rfoot[]{}

%%%%%%%%%%%%%%%%%%%%%%%%%%%%%%%%%%%%%%%%%%%%%%%%%%%%%%%%%%%%%%%%%%%%%%%%%%%%%%%%
%%%% We mostly use the macros of the book, to keep notations
%%%% and conventions the same. Recall that when the macros file
%%%% is updated, we need to comment the lines containing the
%%%% string `[chapter]` since our article is not a book.
%%%%
%%%% Instructions for updating the macros.tex file:
%%%% - fetch the latest macros.tex file from the HoTT/book git repository.
%%%% - comment all lines containing "[chapter]" because this is not a book.
%%%% - comment the definition of pbcorner because the xypic package is not used.
%%%%
\input{macros}

\newcommand{\idsymbin}{=}

%%%%%%%%%%%%%%%%%%%%%%%%%%%%%%%%%%%%%%%%%%%%%%%%%%%%%%%%%%%%%%%%%%%%%%%%%%%%%%%%
%%%% Our commands which are not part of the macros.tex file.
%%%% We should keep these commands separate, because we will
%%%% update the macros.tex following the updates of the book.

%%%% First we redefine the \id, \eqv and \ct commands so that they accept an
%%%% arbitrary number of arguments. This is useful when writing longer strings
%%%% of equalities or equivalences.

\makeatletter

\renewcommand{\id}[3][]{
  \@ifnextchar\bgroup
    {#2 \mathbin{\idsym_{#1}} \id[#1]{#3}}
    {#2 \mathbin{\idsym_{#1}} #3}
  }

\renewcommand{\eqv}[2]{
  \@ifnextchar\bgroup
    {#1 \eqvsym \eqv{#2}}
    {#1 \eqvsym #2}
  }

\newcommand{\ctsym}{%
  \mathchoice{\mathbin{\raisebox{0.5ex}{$\displaystyle\centerdot$}}}%
             {\mathbin{\raisebox{0.5ex}{$\centerdot$}}}%
             {\mathbin{\raisebox{0.25ex}{$\scriptstyle\,\centerdot\,$}}}%
             {\mathbin{\raisebox{0.1ex}{$\scriptscriptstyle\,\centerdot\,$}}}
  }

\renewcommand{\ct}[3][]{
  \@ifnextchar\bgroup
    {#2 \mathbin{\ctsym_{#1}} \ct[#1]{#3}}
    {#2 \mathbin{\ctsym_{#1}} #3}
  }

\makeatother

%%%% We always use textstyle products and sums...
%\renewcommand{\prd}{\tprd}
%\renewcommand{\sm}{\tsm}
\makeatletter
\renewcommand{\@dprd}{\@tprd}
\renewcommand{\@dsm}{\@tsm}
\renewcommand{\@dprd@noparens}{\@tprd}
\renewcommand{\@dsm@noparens}{\@tsm}

%%%% ...with a bit more spacing
\renewcommand{\@tprd}[1]{\mathchoice{{\textstyle\prod_{(#1)}\,}}{\prod_{(#1)}\,}{\prod_{(#1)}\,}{\prod_{(#1)}\,}}
\renewcommand{\@tsm}[1]{\mathchoice{{\textstyle\sum_{(#1)}\,}}{\sum_{(#1)}\,}{\sum_{(#1)}\,}{\sum_{(#1)}\,}}

%%%%%%%%%%%%%%%%%%%%%%%%%%%%%%%%%%%%%%%%%%%%%%%%%%%%%%%%%%%%%%%%%%%%%%%%%%%%%%%%
%%%% We adjust the \prd command so that implicit arguments become possible.
%%%%
%%%% First, we have the following switch. Set it to true if implicit arguments
%%%% are desired, or to false if not. Note turning off implicit arguments
%%%% might render some parts of the text harder to comprehend, since in the
%%%% text might appear $f(x)$ where we would have $f(i,x)$ without implicit
%%%% arguments.

\newcommand{\implicitargumentson}{\boolean{true}}

%%%% If one wants to use implicit arguments in the notation for product types,
%%%% a * has to be put before the argument that has to be implicit.
%%%% For example: in $\prd{x:A}*{y:B}{u:P(y)}Q(x,y,u)$, the argument y is
%%%% implicit. Any of the arguments can be made implicit this way.

%%%% First of all, we should make the command \prd search not only for a
%%%% brace, but also for a star. We introduce an auxiliary command that
%%%% determines whether the next character is a star or brace.
\newcommand{\@ifnextchar@starorbrace}[2]
%  {\@ifnextcharamong{#1}{#2}{*}{\bgroup};}
  {\@ifnextchar*{#1}{\@ifnextchar\bgroup{#1}{#2}}}
  
%%%% When encountering the \prd command, latex should determine whether it
%%%% should print implicit argument brackets or not. So the first branching
%%%% happens right here.
\renewcommand{\prd}{\@ifnextchar*{\@iprd}{\@prd}}

\newcommand{\@prd}[1]
  {\@ifnextchar@starorbrace
    {\prd@parens{#1}}
    {\@ifnextchar\sm{\prd@parens{#1}\@eatsm}{\prd@noparens{#1}}}}
\newcommand{\@prd@parens}{\@ifnextchar*{\@iprd}{\prd@parens}}
\renewcommand{\prd@parens}[1]
  {\@ifnextchar@starorbrace
    {\@theprd{#1}\@prd@parens}
    {\@ifnextchar\sm{\@theprd{#1}\@eatsm}{\@theprd{#1}}}}
\newcommand{\@theprd}[1]
  {\mathchoice{\@dprd{#1}}{\@tprd{#1}}{\@tprd{#1}}{\@tprd{#1}}}
\renewcommand{\dprd}[1]{\@dprd{#1}\@ifnextchar@starorbrace{\dprd}{}}
\renewcommand{\tprd}[1]{\@tprd{#1}\@ifnextchar@starorbrace{\tprd}{}}

%%%% Here we tell the actual symbols to be printed.
\newcommand{\@theiprd}[1]{\mathchoice{\@diprd{#1}}{\@tiprd{#1}}{\@tiprd{#1}}{\@tiprd{#1}}}
\newcommand{\@iprd}[2]{\@ifnextchar@starorbrace%
  {\@theiprd{#2}\@prd@parens}%
  {\@ifnextchar\sm%
    {\@theiprd{#2}\@eatsm}%
    {\@theiprd{#2}}}}
\def\@tiprd#1{
  \ifthenelse{\implicitargumentson}
    {\@@tiprd{#1}\@ifnextchar\bgroup{\@tiprd}{}}
    {\@tprd{#1}}}
\def\@@tiprd#1{\mathchoice{{\textstyle\prod_{\{#1\}}\,}}{\prod_{\{#1\}}\,}{\prod_{\{#1\}}\,}{\prod_{\{#1\}}\,}}
\def\@diprd{
  \ifthenelse{\implicitargumentson}
    {\@tiprd}
    {\@tprd}}
    

%%%% And finally we need to redefine \@eatprd so that implicit arguments also
%%%% works in the scope of a dependent sum.    
\def\@eatprd\prd{\@prd@parens}

\makeatother

%%%%%%%%%%%%%%%%%%%%%%%%%%%%%%%%%%%%%%%%%%%%%%%%%%%%%%%%%%%%%%%%%%%%%%%%%%%%%%%%
%%%% Redefining the quantifiers, so that some of the longer 
%%%% formulas appear one a single line without problems

%%% Dependent products written with \forall, in the same style
\makeatletter
\def\tfall#1{\forall_{(#1)}\@ifnextchar\bgroup{\,\tfall}{\,}}
\renewcommand{\fall}{\tfall}

%%% Existential quantifier %%%
\def\texis#1{\exists_{(#1)}\@ifnextchar\bgroup{\,\texis}{\,}}
\renewcommand{\exis}{\texis}

%%% Unique existence %%%
\def\uexis#1{\exists!_{(#1)}\@ifnextchar\bgroup{\,\uexis}{\,}}
\makeatother
%%%%%%%%%%%%%%%%%%%%%%%%%%%%%%%%%%%%%%%%%%%%%%%%%%%%%%%%%%%%%%%%%%%%%%%%%%%%%%%%

%%%% Introducing logical usage of fonts.
\newcommand{\modelfont}{\mathit} % use 'mf' in command to indicate model font
\newcommand{\typefont}{\mathsf} % use 'tf' in command to indicate type font
\newcommand{\catfont}{\mathrm} % use 'cf' in command to indicate cat font

%%%%%%%%%%%%%%%%%%%%%%%%%%%%%%%%%%%%%%%%%%%%%%%%%%%%%%%%%%%%%%%%%%%%%%%%%%%%%%%%
%%%% Some macros of the book are redefined.

\renewcommand{\UU}{\typefont{U}}
\renewcommand{\isequiv}{\typefont{isEquiv}}
\renewcommand{\happly}{\typefont{hApply}}
\renewcommand{\pairr}[1]{{\mathopen{}\langle #1\rangle\mathclose{}}}
\renewcommand{\type}{\typefont{Type}}
\renewcommand{\op}[1]{{{#1}^\typefont{op}}}
\renewcommand{\susp}{\typefont{\Sigma}}

%%%%%%%%%%%%%%%%%%%%%%%%%%%%%%%%%%%%%%%%%%%%%%%%%%%%%%%%%%%%%%%%%%%%%%%%%%%%%%%%
%%%% The following is a big unorganized list of new macros that we use in the
%%%% notes. 

\newcommand{\mfM}{\modelfont{M}}
\newcommand{\mfN}{\modelfont{N}}
\newcommand{\tfctx}{\typefont{ctx}}
\newcommand{\mftypfunc}[1]{{\modelfont{typ}^{#1}}}
\newcommand{\mftyp}[2]{{\mftypfunc{#1}(#2)}}
\newcommand{\tftypfunc}[1]{{\typefont{typ}^{#1}}}
\newcommand{\tftyp}[2]{{\tftypfunc{#1}(#2)}}
\newcommand{\hfibfunc}[1]{\typefont{fib}_{#1}}
\newcommand{\mappingcone}[1]{\mathcal{C}_{#1}}
\newcommand{\equifib}{\typefont{equiFib}}
\newcommand{\tfcolim}{\typefont{colim}}
\newcommand{\tflim}{\typefont{lim}}
\newcommand{\tfdiag}{\typefont{diag}}
\newcommand{\tfGraph}{\typefont{Graph}}
\newcommand{\mfGraph}{\modelfont{Graph}}
\newcommand{\unitGraph}{\unit^\mfGraph}
\newcommand{\UUGraph}{\UU^\mfGraph}
\newcommand{\tfrGraph}{\typefont{rGraph}}
\newcommand{\mfrGraph}{\modelfont{rGraph}}
\newcommand{\isfunction}{\typefont{isFunction}}
\newcommand{\tfconst}{\typefont{const}}
\newcommand{\conemap}{\typefont{coneMap}}
\newcommand{\coconemap}{\typefont{coconeMap}}
\newcommand{\tflimits}{\typefont{limits}}
\newcommand{\tfcolimits}{\typefont{colimits}}
\newcommand{\islimiting}{\typefont{isLimiting}}
\newcommand{\iscolimiting}{\typefont{isColimiting}}
\newcommand{\islimit}{\typefont{isLimit}}
\newcommand{\iscolimit}{\typefont{iscolimit}}
\newcommand{\pbcone}{\typefont{cone_{pb}}}
\newcommand{\tfinj}{\typefont{inj}}
\newcommand{\tfsurj}{\typefont{surj}}
\newcommand{\tfepi}{\typefont{epi}}
\newcommand{\tftop}{\typefont{top}}
\newcommand{\sbrck}[1]{\Vert #1\Vert}
\newcommand{\strunc}[2]{\Vert #2\Vert_{#1}}
\newcommand{\gobjclass}{{\typefont{U}^\mfGraph}}
\newcommand{\gcharmap}{\typefont{fib}}
\newcommand{\diagclass}{\typefont{T}}
\newcommand{\opdiagclass}{\op{\diagclass}}
\newcommand{\equifibclass}{\diagclass^{\eqv{}{}}}
\newcommand{\universe}{\typefont{U}}
\newcommand{\catid}[1]{{\catfont{id}_{#1}}}
\newcommand{\isleftfib}{\typefont{isLeftFib}}
\newcommand{\isrightfib}{\typefont{isRightFib}}
\newcommand{\leftLiftings}{\typefont{leftLiftings}}
\newcommand{\rightLiftings}{\typefont{rightLiftings}}
\newcommand{\psh}{\typefont{Psh}}
\newcommand{\rgclass}{\typefont{\Omega^{RG}}}
\newcommand{\terms}[2][]{\lfloor #2 \rfloor^{#1}}
\newcommand{\grconstr}[2]
             {\mathchoice % max size is textstyle size.
             {{\textstyle \int_{#1}}#2}% 
             {\int_{#1}#2}%
             {\int_{#1}#2}%
             {\int_{#1}#2}}
\newcommand{\ctxhom}[3][]{\typefont{hom}_{#1}(#2,#3)}
\newcommand{\graphcharmapfunc}[1]{\gcharmap_{#1}}
\newcommand{\graphcharmap}[2][]{\graphcharmapfunc{#1}(#2)}
\newcommand{\tfexp}[1]{\typefont{exp}_{#1}}
\newcommand{\tffamfunc}{\typefont{fam}}
\newcommand{\tffam}[1]{\tffamfunc(#1)}
\newcommand{\tfev}{\typefont{ev}}
\newcommand{\tfcomp}{\typefont{comp}}
\newcommand{\isDec}[1]{\typefont{isDecidable}(#1)}
\newcommand{\smal}{\mathcal{S}}
\renewcommand{\modal}{{\ensuremath{\ocircle}}}
\newcommand{\eqrel}{\typefont{EqRel}}
\newcommand{\piw}{\ensuremath{\Pi\typefont{W}}} %% to be used in conjunction with -pretopos.
\renewcommand{\sslash}{/\!\!/}
\newcommand{\mprd}[2]{\Pi(#1,#2)}
\newcommand{\msm}[2]{\Sigma(#1,#2)}
\newcommand{\midt}[1]{\idvartype_#1}
\newcommand{\reflf}[1]{\typefont{refl}^{#1}}
\newcommand{\tfJ}{\typefont{J}}
\newcommand{\tftrans}{\typefont{trans}}

\newcommand{\tfT}{\typefont{T}}
\newcommand{\reflsym}{{\mathsf{refl}}}
\newcommand{\strans}[2]{\ensuremath{{#1}_{*}({#2})}}

%%%%%%%%%%%%%%%%%%%%%%%%%%%%%%%%%%%%%%%%%%%%%%%%%%%%%%%%%%%%%%%%%%%%%%%%%%%%%%%%
%%%% JUDGMENTS
%%%%
%%%% Below we define several commands for the judgments of type theory. There
%%%% are commands
%%%% * \jctx for the judgment that something is a context.
%%%% * \jctxeq for the judgment that two contexts are the same
%%%% * \jtype for the judgment that something is a type in a context
%%%% * \jtypeeq for the judgment that two types in the same context are the same
%%%% * \jterm for the judgment that something is a term of a type in a context
%%%% * \jtermeq for the judgment that two terms of the same type are the same

\makeatletter
\newcommand{\jctx}{\@ifnextchar*{\@jctxAlignTrue}{\@jctxAlignFalse}}
\newcommand{\@jctxAlignTrue}[2]{& \vdash #2~ctx}
\newcommand{\@jctxAlignFalse}[1]{\vdash #1~ctx}

\newcommand{\jtype}{\@ifnextchar*{\@jtypeAlignTrue}{\@jtypeAlignFalse}}
\newcommand{\@jtypeAlignFalse}[2]{#1\vdash #2~type}
\newcommand{\@jtypeAlignTrue}[3]{#2 & \vdash #3~type}

\newcommand{\jtermc}{\@ifnextchar*{\@jtermcAlignTrue}{\@jtermcAlignFalse}}
\newcommand{\@jtermcAlignTrue}[3]{ & \vdash #3:#2}
\newcommand{\@jtermcAlignFalse}[2]{\vdash #2:#1}

\newcommand{\jtermt}{\@ifnextchar*{\@jtermtAlignTrue}{\@jtermtAlignFalse}}
\newcommand{\@jtermtAlignTrue}[4]{#2 & \vdash #4:#3}
\newcommand{\@jtermtAlignFalse}[3]{#1 \vdash #3:#2}

\newcommand{\jctxeq}{\@ifnextchar*{\@jctxeqAlignTrue}{\@jctxeqAlignFalse}}
\newcommand{\@jctxeqAlignTrue}[3]{& \vdash #2\jdeq #3~ctx}
\newcommand{\@jctxeqAlignFalse}[2]{\vdash #1\jdeq #2~ctx}

\newcommand{\jtypeeq}{\@ifnextchar*{\@jtypeeqAlignTrue}{\@jtypeeqAlignFalse}}
\newcommand{\@jtypeeqAlignTrue}[4]{#2 & \vdash #3\jdeq #4~type}
\newcommand{\@jtypeeqAlignFalse}[3]{#1\vdash #2\jdeq #3~type}

\newcommand{\jtermceq}{\@ifnextchar*{\@jtermceqAlignTrue}{\@jtermceqAlignFalse}}
\newcommand{\@jtermceqAlignTrue}[4]{& \vdash #3\jdeq #4:#2}
\newcommand{\@jtermceqAlignFalse}[3]{\vdash #2\jdeq #3:#1}

\newcommand{\jtermteq}{\@ifnextchar*{\@jtermteqAlignTrue}{\@jtermteqAlignFalse}}
\newcommand{\@jtermteqAlignTrue}[5]{#2 & \vdash #4\jdeq #5:#3}
\newcommand{\@jtermteqAlignFalse}[4]{#1\vdash #3\jdeq #4:#2}
\makeatother

%%%%%%%%%%%%%%%%%%%%%%%%%%%%%%%%%%%%%%%%%%%%%%%%%%%%%%%%%%%%%%%%%%%%%%%%%%%%%%%%
%%%% Often we shall need to display lists of inference rules. This environment
%%%% adjusts the array environment so that there is enough vertical space
%%%% between two inference rules
%%%%
%%%% bug: there's two much space above the array.

\newenvironment{infarray}[1]{\begingroup\renewcommand*{\arraystretch}{3}
\begin{equation*}
\begin{array}{#1}}{
\end{array}
\end{equation*}
\endgroup}

%%%%%%%%%%%%%%%%%%%%%%%%%%%%%%%%%%%%%%%%%%%%%%%%%%%%%%%%%%%%%%%%%%%%%%%%%%%%%%%%
%%%% CONTEXT EXTENSION 
%%%%
%%%% explicit context extension notation which we will use only rarely

\newcommand{\tfext}{\typefont{ext}}

%%%% The context extension command.
%%%%
%%%% To get a feeling of how the command works, here are a few examples.
%%%% \ctxext{A}{B} will print A.B
%%%% \ctxext{{A}{B}}{C} will print (A.B).C
%%%% \ctxext{{{A}{B}}{C}}{{D}{E}} will print ((A.B).C).(D.E)

\makeatletter
\newcommand{\ctxext}[2]{\@ctxext@ctx #1.\@ctxext@type #2}
\newcommand{\@ctxext}{\@ifnextchar\bgroup{\@@ctxext}{}}
\newcommand{\@ctxext@ctx}{\@ifnextchar\ctxext{\@ctxext@nested}{\@ifnextchar\ctxwk{\@ctxwk@nested}{\@ctxext}}}
\newcommand{\@ctxext@type}{\@ifnextchar\ctxext{\@ctxext@nested}{\@ifnextchar\subst{\@subst@nested}{\@ctxext}}}
\newcommand{\@@ctxext}[1]{\@ifnextchar\bgroup{\@ctxext@parens{#1}}{#1}}
\newcommand{\@ctxext@parens}[2]{(\ctxext{#1}{#2})}
\newcommand{\@ctxext@nested}[3]{\@ctxext@parens{#2}{#3}}
\makeatother

%%%%%%%%%%%%%%%%%%%%%%%%%%%%%%%%%%%%%%%%%%%%%%%%%%%%%%%%%%%%%%%%%%%%%%%%%%%%%%%%
%%%% SUBSTITUTION

\newcommand{\tfsubst}{\typefont{subst}}

%%%% The substitution command will act the following way
%%%%
%%%% \subst{x}{P} will print P[x]
%%%% \subst{x}{{f}{Q}} will print Q[f][x]
%%%% \subst{{x}{f}}{{x}{Q}} will print Q[x][f[x]]

\makeatletter
\newcommand{\subst}[2]{\@subst@type #2[\@subst@term #1]}
\newcommand{\@subst}{\@ifnextchar\bgroup{\@@subst}{}}
\newcommand{\@@subst}[1]{\@ifnextchar\bgroup{\subst{#1}}{#1}}
\newcommand{\@subst@term}{\@subst}
\newcommand{\@subst@type}{\@ifnextchar\ctxext{\@ctxext@nested}{\@ifnextchar\ctxwk{\@ctxwk@nested}{\@subst}}}
\newcommand{\@subst@nested}[3]{\@subst@parens{#2}{#3}}
\newcommand{\@subst@parens}[2]{(\subst{#1}{#2})}
\makeatother

%%%%%%%%%%%%%%%%%%%%%%%%%%%%%%%%%%%%%%%%%%%%%%%%%%%%%%%%%%%%%%%%%%%%%%%%%%%%%%%%
%%%% WEAKENING

\newcommand{\tfwk}{\typefont{wk}}

%%%% The weakening command is very much like the substitution command.

\makeatletter
\newcommand{\ctxwk}[2]{\langle\@ctxwk@act #1\rangle\@ctxwk@pass #2}
\newcommand{\@ctxwk}{\@ifnextchar\bgroup{\@@ctxwk}{}}
\newcommand{\@@ctxwk}[1]{\@ifnextchar\bgroup{\ctxwk{#1}}{#1}}
\newcommand{\@ctxwk@act}{\@ctxwk}
\newcommand{\@ctxwk@pass}{\@ifnextchar\ctxext{\@ctxext@nested}{\@ifnextchar\subst{\@subst@nested}{\@ctxwk}}}
\newcommand{\@ctxwk@parens}[2]{(\ctxwk{#1}{#2})}
\newcommand{\@ctxwk@nested}[3]{\@ctxwk@parens{#2}{#3}}
\makeatother

%%%%%%%%%%%%%%%%%%%%%%%%%%%%%%%%%%%%%%%%%%%%%%%%%%%%%%%%%%%%%%%%%%%%%%%%%%%%%%%%
%%%% When investigation pointed structures we use the \pt macro.

\makeatletter
\newcommand{\pt}[1][]{*_{
  \@ifnextchar\undergraph{\@undergraph@nested}
    {\@ifnextchar\underovergraph{\@underovergraph@nested}{}}#1}}
\makeatother

%%%%%%%%%%%%%%%%%%%%%%%%%%%%%%%%%%%%%%%%%%%%%%%%%%%%%%%%%%%%%%%%%%%%%%%%%%%%%%%%
%%%% OPERATIONS ON GRAPHS
%%%%
%%%% First of all, each graph has a type of vertices and a type of edges. The
%%%% type of vertices of a graph $\Gamma$ is denoted by $\pts{\Gamma}$;
%%%% and likewise for the type of edges.

\makeatletter
\newcommand{\pts}[1]{{\@graphop@nested{#1}}_{0}}
\newcommand{\edg}[1]{{\@graphop@nested{#1}}_{1}}
\newcommand{\@graphop@nested}[1]
  {\@ifnextchar\ctxext{\@ctxext@nested}
      {\@ifnextchar\undergraph{\@undergraph@nested}
         {\@ifnextchar\underovergraph{\@underovergraph@nested}{}}}
    #1}
\makeatother

%%%% The following operations of \undergraph and \underovergraph are used to
%%%% define the free category and the free groupoid of a graph, respectively

\makeatletter
\newcommand{\@undergraphtest}[2]{\@ifnextchar({#1}{#2}}
\newcommand{\undergraph}[2]{\@undergraphtest{\@undergraph@parens{#1}{#2}}{\@undergraph{#1}{#2}}}
\newcommand{\@undergraph}[2]{{#2/#1}}
\newcommand{\@undergraph@nested}[3]{\@undergraph@parens{#2}{#3}}
\newcommand{\@undergraph@parens}[2]{(\@undergraph{#1}{#2})}
\makeatother

\makeatletter
\newcommand{\underovergraph}[2]{\@underovergraphtest{\@underovergraph@parens{#1}{#2}}{\@underovergraph{#1}{#2}}}
\newcommand{\@underovergraph}[2]{{#2}\,{\parallel}\,{#1}}
\newcommand{\@underovergraphtest}{\@undergraphtest}
\newcommand{\@underovergraph@parens}[2]{(\@underovergraph{#1}{#2})}
\newcommand{\@underovergraph@nested}[3]{\@underovergraph@parens{#2}{#3}}
\makeatother

\newcommand{\graphid}[1]{\mathrm{id}_{#1}}
\newcommand{\freecat}[1]{\mathcal{C}(#1)}
\newcommand{\freegrpd}[1]{\mathcal{G}(#1)}

%%%%%%%%%%%%%%%%%%%%%%%%%%%%%%%%%%%%%%%%%%%%%%%%%%%%%%%%%%%%%%%%%%%%%%%%%%%%%%%%
%% Some tikz macros to typeset diagrams uniformly.

\tikzset{patharrow/.style={double,double equal sign distance,-,font=\scriptsize}}
\tikzset{description/.style={fill=white,inner sep=2pt}}

%% Used for extra wide diagrams, e.g. when the label is too large otherwise.
\tikzset{commutative diagrams/column sep/Huge/.initial=18ex}

%%%%%%%%%%%%%%%%%%%%%%%%%%%%%%%%%%%%%%%%%%%%%%%%%%%%%%%%%%%%%%%%%%%%%%%%%%%%%%%%
%%%% New theorem environment for conjectures.

\defthm{conj}{Conjecture}{Conjectures}

%%%%%%%%%%%%%%%%%%%%%%%%%%%%%%%%%%%%%%%%%%%%%%%%%%%%%%%%%%%%%%%%%%%%%%%%%%%%%%%%
%%%% The following environment for desiderata should not be there. It is better
%%%% to use the issue tracker for desiderata.

\newenvironment{desiderata}{\begingroup\color{blue}\textbf{Desiderata.}}
{\endgroup}


\title{Notes on the smash product}
\author{Floris van Doorn \and Stefano Piceghello}
\date{\today}
\usepackage{fullpage}
\newcommand{\pmap}{\to}
\newcommand{\lpmap}{\xrightarrow}
\newcommand{\smsh}{\wedge}
\renewcommand{\phi}{\varphi}
\renewcommand{\epsilon}{\varepsilon}
\newcommand{\tr}{\cdot}
\renewcommand{\o}{\ensuremath{\circ}}
\newcommand{\auxl}{\mathsf{auxl}}
\newcommand{\auxr}{\mathsf{auxr}}
\newcommand{\gluel}{\mathsf{gluel}}
\newcommand{\gluer}{\mathsf{gluer}}
\newcommand{\sy}{^{-1}}
\newcommand{\const}{\ensuremath{\mathbf{0}}\xspace}
\newcommand{\alphabar}{\overline{\alpha}}
\newcommand{\rhobar}{\overline{\rho}}
\newcommand{\lambdabar}{\overline{\lambda}}
\newcommand{\gammabar}{\overline{\gamma}}
\newcommand{\pType}{\mathsf{Type}_\ast}
\newcommand{\zeroh}{\mathsf{z}}
\newcommand{\oneh}{\mathsf{u}}
\newcommand{\two}{\mathsf{t}}
\newcommand{\twist}{\mathsf{c}}
\newcommand{\mc}{\mathcal}

\begin{document}

\maketitle

\section{Pointed Types}

\begin{defn}
  We work in the $(\infty,1)$-category of pointed types.
  \begin{itemize}
  \item The objects are pointed types $A$, types together with a basepoint $a_0:A$.
\item 1-cells are pointed maps $f:A\to B$ which are maps with a chosen path $f_0:f(a_0)=b_0$. We
  write $A\pmap B$ for pointed maps and $A\pmap B\pmap C$ means $A\pmap (B\pmap C)$.
\item 2-cells are pointed homotopies. A pointed homotopy $h:f\sim g$ is a homotopy with a chosen 2-path
  $h(a_0) \tr g_0 = f_0$.
\item As 3-cells (or higher cells) we take equalities between 2-cells (or higher cells).
\end{itemize}
\end{defn}

\begin{rmk}\label{rmk:pointed-types}
  All types, maps and homotopies in these notes are pointed, unless explicitly mentioned
  otherwise. Whenever we say that a diagram of $n$-cells commutes we mean it in the sense that there
  is an $(n+1)$-cell witnessing it.
\item Pointed homotopies are equivalent to equalities of pointed types: $(f\sim g)\equiv (f=g)$. So
  we could have chosen to define our 2-cells as equalities between 1-cells. We choose not to, since
  the aforementioned equivalence requires function extensionality. In a type theory where function
  extensionality does not compute (like Lean) it is better to define the type of pointed homotopies
  manually so that the underlying homotopy of a 2-cell is definitionally equal to the homotopy we
  started with. In diagrams, we will denote pointed homotopies by equalities, but we always mean
  pointed homotopies.
\item The type $A\to B$ of pointed maps from $A$ to $B$ is itself pointed, with as basepoint the
  constant map $0\equiv0_{A,B}:A\to B$ which has as underlying function $\lam{a:A}b_0$. We have the homotopies:
  \begin{align*}
  \zeroh_g &: 0 \o g \sim 0 & \zeroh'_f &: f \o 0 \sim 0\\
  \oneh_f &: f \o \idfunc \sim f & \oneh'_g &: \idfunc \o g \sim g
  \end{align*}
  with $\zeroh_{\idfunc} = \oneh_0$ and $\zeroh'_{\idfunc} = \oneh'_0$
\item A pointed equivalence is a pointed map $f : A \to B$ whose underlying map is an
  equivalence. In this case, we can find a pointed map $f\sy:B\to A$ with pointed homotopies
  $f\o f\sy\sim0$ and $f\sy\o f\sim0$. We have the pointed equivalences:
  \[\two : (\bool \to X) \simeq X\] with underlying map defined with $\two(f) \defeq f(1_\bool)$, and
  \[\twist : (A \to B \to X) \simeq (B \to A \to X)\]
  with underlying map defined with $\twist(f) \defeq \lam{b}\lam{a}f(a)(b)$.
\end{rmk}

\begin{lem}
  Given maps $f:A'\pmap A$ and $g:B\pmap B'$. Then there are maps
  $(f\pmap C):(A\pmap C)\pmap(A'\pmap C)$ and $(C\pmap g):(C\pmap B)\pmap(C\pmap B')$ given by
  precomposition with $f$, resp. postcomposition with $g$. The map $\lam{g}C\pmap g$ preserves the basepoint, giving rise to a map $$(C\pmap ({-})):(B\pmap B')\pmap(C\pmap B)\pmap(C\pmap B').$$
  Also, the following square commutes:
\begin{center}
\begin{tikzcd}
(A\pmap B) \arrow[r,"A\pmap g"]\arrow[d,"f\pmap B"] & (A\pmap B')\arrow[d,"f\pmap B'"] \\
(A'\pmap B) \arrow[r,"A'\pmap g"] & (A'\pmap B')
\end{tikzcd}
\end{center}

\end{lem}

\section{Naturality and a version of the Yoneda lemma}

\begin{defn}\label{def:naturality}
	Let $F$, $G$ be functors of pointed types, i.e. pointed maps with a functorial action (e.g. if $f : A \to B$, then we can define $F(f) : F(A) \to F(B)$, respecting identity and composition). 
	Let $\theta : F \Rightarrow G$ be a natural transformation from $F$ to $G$, i.e. a pointed map $F(X) \to G(X)$ for all pointed types $X$. For every $f : A \to B$, there is a diagram:
	\begin{center}
	\begin{tikzcd}
		F(A)
			\arrow[r, "F(f)"]
			\arrow[d, swap, "\theta_A"]
		& F(B)
			\arrow[d, "\theta_B"]
		\\
		G(A)
			\arrow[r, swap, "G(f)"]
		& G(B)
	\end{tikzcd}
	\end{center}
	We define the following notions of naturality for $\theta$:
	\begin{itemize}
		\item \textbf{(strong) naturality} will refer to a pointed homotopy
		\[p_\theta(f) : G(f) \o \theta_A \sim \theta_B \o F(f)\]
		for every $f : A \to B$ and \textbf{weak naturality} to the underlying (non-pointed) homotopy;
		\item \textbf{pointed (strong) naturality} will refer to the same pointed homotopy, with the additional condition that $p_\theta(0) = (p_\theta)_0$, where
		\[(p_\theta)_0 : G(0) \o \theta_A \sim 0 \o \theta_A \sim 0 \sim \theta_B \o 0 \sim \theta_B \o F(0)\]
		is the canonical proof of the pointed homotopy $G(0) \o \theta_A \sim \theta_B \o F(0)$, whereas \textbf{pointed weak naturality} will refer to the corresponding non-pointed condition.
	\end{itemize}
\end{defn}

\begin{rmk}
	The relation between the four notions of naturality is as expected: strong implies weak, and pointed implies simple. Weak naturality is generally ill-behaved: for example, weak naturality of $\theta$ does not imply weak naturality of $\theta \to X$ or $X \to \theta$, whereas the implication holds for strong naturality.
\end{rmk}

\begin{rmk}
	The equivalences $\two$ and $\twist$ as defined in \autoref{rmk:pointed-types} are natural in all their arguments and pointed natural in the last argument.
\end{rmk}

\begin{lem}[Yoneda]\label{lem:yoneda}
	Let $A$, $B$ be pointed types, and assume, for all pointed types $X$, a pointed equivalence $\phi_X : (B \to X) \simeq (A \to X)$, natural in $X$, i.e. for all $f : X \to X'$ there is a homotopy \[ p_\phi(f) : (A \to f) \o \phi_X \sim \phi_X' \o (B \to f) \]
%	 making the following diagram commute for all $f : X \to X'$:
%	\begin{center}
%	\begin{tikzcd}
%		(B \to X)
%			\arrow[r, "\phi_X"]
%			\arrow[d, swap, "f \o -"]
%		& (A \to X)
%			\arrow[d, "f \o -"]
%		\\
%		(B \to X')
%			\arrow[r, swap,"\phi_{X'}"]
%		& (A \to X')
%	\end{tikzcd}
%	\end{center}
	Then there exists a pointed equivalence $\psi_\phi : A \simeq B$.
\end{lem}
\begin{proof}
	We define $\psi_\phi \defeq \phi_B(\idfunc[B]) : A \to B$ and $\psi_\phi\sy \defeq \phi_A\sy(\idfunc[A])$. The given naturality square for $X \defeq B$ and $g \defeq \psi_\phi\sy$ yields $\psi_\phi\sy \o \phi_B (\idfunc[B]) \judgeq \psi_\phi\sy \o \psi_\phi \sim \phi_A (\psi_\phi\sy \o \idfunc[B]) \judgeq \phi_A (\phi_A\sy (\idfunc[A])) \sim \idfunc[A]$, and similarly for the inverse composition.
\end{proof}

\begin{lem}\label{lem:yoneda-pointed}
	Assume $A$, $B$, $\phi_X$ and $p$ as in \autoref{lem:yoneda}, and assume moreover that $\phi_X$ is pointed natural. Then there is a pointed homotopy $(\psi_\phi \to X) \sim \phi_X$.
\end{lem}

\begin{proof}
	Let $f : B \to X$. The underlying homotopy is obtained by:
	\begin{align*}
		(\psi_\phi \to X)(f) &\judgeq f \o \psi_\phi\\
		&\sim \phi_X (f \o \idfunc) &&\text{(by $p_\phi(f)(\idfunc)$)}\\
		&\sim \phi_X (f) &&\text{(by $\mapfunc{\phi_X}(\oneh_f)$)}
	\end{align*}
	To show that this is a pointed homotopy, we need to prove that the following diagram commutes:
	\begin{center}
	\begin{tikzcd}[column sep=4em]
		(\psi_\phi \to X)(0)
			\arrow[rr, equals, "p_\phi(0)(\idfunc)\tr\mapfunc{\phi_X}(\oneh_0)"]
			\arrow[dr, equals, swap, "\zeroh_{\psi_\phi}"]
		&&\phi_X(0)
			\arrow[dl, equals, "(\phi_X)_0"]
		\\
		&0
	\end{tikzcd}
	\end{center}
	where the top-left expression is definitionally equal to $0 \o \phi_X(\idfunc)$, the horizontal path comes from the underlying homotopy and $(\phi_X)_0$ is the canonical path from $\phi_X(0)$ to $0$. Since $\phi_X$ is pointed natural, we have that
	$p_{\phi_X}(0)(\idfunc) = (p_{\phi_X})_0(\idfunc)$, which, in this case, is:
	\begin{align*}
	0\o \phi_X(\idfunc)
	&= 0 &&\text{(by $\zeroh_{q_X(\idfunc)}$)}\\
	&= \phi_X(0) &&\text{(by $(\phi_X)_0\sy$)}\\
	&= \phi_X(0\o 1) &&\text{(by $(\mapfunc{\phi_X}(\zeroh_{\idfunc}))\sy$)}
	\end{align*}
	The diagram then commutes by cancellation of inverses and using that $\zeroh_{\idfunc} = \oneh_0$.
\end{proof}

\section{Smash Product}\label{sec:smash}

\begin{defn}
  The smash of $A$ and $B$ is the HIT generated by the point constructor $(a,b)$ for $a:A$ and $b:B$
  and two auxilliary points $\auxl,\auxr:A\smsh B$ and path constructors $\gluel_a:(a,b_0)=\auxl$
  and $\gluer_b:(a_0,b)=\auxr$ (for $a:A$ and $b:B$). $A\smsh B$ is pointed with point $(a_0,b_0)$.
\end{defn}
\begin{rmk}
  This definition of $A\smsh B$ is basically the pushout of
  $\bool\leftarrow A+B\to A \times B$.  A more traditional definition of $A\smsh B$ is the pushout
  $\unit\leftarrow A\vee B\to A \times B$; here $\vee$ denotes the wedge product, which can be
  equivalently described as either the pushout $A\leftarrow \unit\to B$ or
  $\unit\leftarrow \bool\to A + B$. These two definitions of $A\smsh B$ are equivalent, because in
  the following diagram the top-left square and the top rectangle are pushout squares, hence the
  top-right square is a pushout square by applying the pushout lemma. Another application of the
  pushout lemma then states that the two definitions of $A\smsh B$ are equivalent.
\begin{center}
\begin{tikzcd}
\bool \arrow[r]\arrow[d] & A+B     \arrow[r]\arrow[d] & \bool \arrow[d] \\
\unit \arrow[r]          & A\vee B \arrow[r]\arrow[d] & \unit \arrow[d] \\
                     & A\times B        \arrow[r] & A\smsh B
\end{tikzcd}
\end{center}

\end{rmk}
\begin{lem}\mbox{}\label{lem:smash-general}
	The smash product satisfies the following properties.
  \begin{itemize}
  \item The smash product is functorial: if $f:A\pmap A'$ and $g:B\pmap B'$ then
    $f\smsh g:A\smsh B\pmap A'\smsh B'$. We write $A\smsh g$ or $f\smsh B$ if one of the
    functions is the identity function.
  \item The smash product preserves composition, which gives rise to the interchange law:
    \[i:(f' \o f)\smsh (g' \o g) \sim f' \smsh g' \o f \smsh g\]
  \item If $p:f\sim f'$ and $q:g\sim g'$ then $p\smsh q:f\smsh g\sim f'\smsh g'$. This operation
    preserves reflexivities, symmetries and transitivies.
  \item There are homotopies $f\smsh0\sim0$ and $0\smsh g\sim 0$ such that the following diagrams
    commute for given homotopies $p : f\sim f'$ and $q : g\sim g'$.
    \begin{center}
\begin{tikzcd}
f\smsh 0 \arrow[rr, equals,"p\smsh1"]\arrow[dr,equals] & &
f'\smsh 0\arrow[dl,equals] \\
& 0 &
\end{tikzcd}
\qquad
\begin{tikzcd}
0\smsh g\arrow[rr, equals,"1\smsh q"]\arrow[dr,equals] & &
0\smsh g'\arrow[dl,equals] \\
& 0 &
\end{tikzcd}
\end{center}

  \end{itemize}
\end{lem}

\begin{lem}\label{lem:smash-coh}
  Suppose that we have maps $A_1\lpmap{f_1}A_2\lpmap{f_2}A_3$ and $B_1\lpmap{g_1}B_2\lpmap{g_2}B_3$
  and suppose that either $f_1$ or $f_2$ is constant. Then there are two homotopies
  $(f_2 \o f_1)\smsh (g_2 \o g_1)\sim 0$, one which uses interchange and one which doesn't. These two
  homotopies are equal. Specifically, the following two diagrams commute:
\begin{center}
\begin{tikzcd}
(f_2 \o 0)\smsh (g_2 \o g_1) \arrow[r, equals]\arrow[dd,equals] &
(f_2 \smsh g_2)\o (0 \smsh g_1)\arrow[d,equals] \\
& (f_2 \smsh g_2)\o 0\arrow[d,equals] \\
0\smsh (g_2 \o g_1) \arrow[r,equals] &
0
\end{tikzcd}
\qquad
\begin{tikzcd}
(0 \o f_1)\smsh (g_2 \o g_1) \arrow[r, equals]\arrow[dd,equals] &
(0 \smsh g_2)\o (f_1 \smsh g_1)\arrow[d,equals] \\
& 0\o (f_1 \smsh g_1)\arrow[d,equals] \\
0\smsh (g_2 \o g_1) \arrow[r,equals] &
0
\end{tikzcd}
\end{center}

\end{lem}
\begin{proof}
  We will only do the case where $f_1\jdeq 0$, i.e. fill the diagram on the left. The other case is
  similar (and slightly easier). First apply induction on the paths that $f_2$, $g_1$ and $g_2$
  respect the basepoint. In this case $f_2\o0$ is definitionally equal to $0$, and the canonical
  proof that $f_2\o 0\sim0$ is (definitionally) equal to reflexivity. This means that the homotopy
  $(f_2 \o 0)\smsh (g_2 \o g_1)\sim0\smsh (g_2 \o g_1)$ is also equal to reflexivity, and also the
  path that $f_2 \smsh g_2$ respects the basepoint is reflexivity, hence the homotopy
  $(f_2 \smsh g_2)\o 0\sim0$ is also reflexivity. This means we need to fill the following square:
\begin{center}
\begin{tikzcd}
(f_2 \o 0)\smsh (g_2 \o g_1) \arrow[r, equals,"i"]\arrow[d,equals,"1"] &
(f_2 \smsh g_2)\o (0 \smsh g_1)\arrow[d,equals,"(f_2\smsh g_2)\o \zeroh"] \\
0\smsh (g_2 \o g_1) \arrow[r,equals,"\zeroh"] &
0
\end{tikzcd}
\end{center}

  For the underlying homotopy, take $x : A_1\smsh B_1$ and apply induction on $x$. Suppose
  $x\equiv(a,b)$ for $a:A_1$ and $b:B_1$. Then we have to fill the square (denote the basepoints of
  $A_i$ and $B_i$ by $a_i$ and $b_i$ and we suppress the arguments of $\gluer$). Now
  $\mapfunc{h\smsh k}(\gluer_z)=\gluer_{k(z)}$, so by general groupoid laws we see that the path on
  the bottom is equal to the path on the right, which means we can fill the square.
  \begin{center}
    \begin{tikzcd}
      (f_2(a_2),g_2(g_1(b)))\arrow[r, equals,"1"]
      \arrow[d,equals,"1"] &
      % \arrow[d,equals,"\gluer_{g_2(g_1(b))}\tr\gluer_{g_2(g_1(b_1))}\sy"] &
      (f_2(a_2),g_2(g_1(b)))\arrow[d,equals,"\mapfunc{f_2\smsh g_2}(\gluer\tr\gluer\sy)"] \\
      (a_3,g_2(g_1(b))) \arrow[r,equals,"\gluer\tr\gluer\sy"] &
      (a_3,b_3)
    \end{tikzcd}
  \end{center}
  If $x$ is either $\auxl$ or $\auxr$ it is similar but easier. For completeness, we will write down the square we have to fill in the case that $x$ is $\auxr$.
  \begin{center}
    \begin{tikzcd}
      \auxr \arrow[r, equals,"1"]
      \arrow[d,equals,"1"] &
      \auxr \arrow[d,equals,"\mapfunc{f_2\smsh g_2}(\gluer_{b_2}\sy)"] \\
      \auxr \arrow[r,equals,"\gluer_{g_2(b_2)}\sy"] &
      (a_3,b_3)
    \end{tikzcd}
  \end{center}

  If $x$ varies over $\gluer_b$, we need to fill the cube below. The front and the back are the
  squares we just filled, the left square is a degenerate square, and the other three squares are
  the squares in the definition of $q$ and $i$ to show that they respect $\gluer_b$ (and on the
  right we apply $f_2\smsh g_2$ to that square).
  \begin{center}
    \begin{tikzcd}
      & \auxr \arrow[rr, equals,"1"] \arrow[dd,equals,near start,"1"] & &
      \auxr \arrow[dd,equals,"\mapfunc{f_2\smsh g_2}(\gluer_{b_2}\sy)"] \\
      (f_2(a_2),g_2(g_1(b)))\arrow[rr, equals, near start, crossing over, "1"]
      \arrow[dd,equals,"1"] \arrow[ur,equals] & &
      (f_2(a_2),g_2(g_1(b))) \arrow[ur,equals] & \\
      & \auxr \arrow[rr,equals,near start, "\gluer_{g_2(b_2)}\sy"] & & (a_3,b_3) \\
      (a_3,g_2(g_1(b))) \arrow[rr,equals,"\gluer\tr\gluer\sy"] \arrow[ur,equals] & &
      (a_3,b_3) \arrow[from=uu, equals, crossing over, very near start, "\mapfunc{f_2\smsh g_2}(\gluer\tr\gluer\sy)"]       \arrow[ur,equals] &
    \end{tikzcd}
  \end{center}
  After canceling applications of
  $\mapfunc{h\smsh k}(\gluer_z)=\gluer_{k(z)}$ on various sides of the squares (TODO).


  If $x$ varies over $\gluel_a$ the proof is very similar. Only in the end we need to fill the
  following cube instead (TODO).

  To show that this homotopy is pointed, (TODO)

\end{proof}

\begin{thm}\label{thm:smash-functor-right}
Given pointed types $A$, $B$ and $C$, the functorial action of the smash product induces a map
$$({-})\smsh C:(A\pmap B)\pmap(A\smsh C\pmap B\smsh C)$$
which is natural in $A$, $B$ and dinatural in $C$.
\end{thm}
The naturality and dinaturality means that the following squares commute for $f : A' \to A$ $g:B\to B'$ and $h:C\to C'$.
\begin{center}
\begin{tikzcd}
(A\pmap B) \arrow[r,"({-})\smsh C"]\arrow[d,"f\pmap B"] &
(A\smsh C\pmap B\smsh C)\arrow[d,"f\smsh C\pmap B\smsh C"] \\
(A'\pmap B) \arrow[r,"({-})\smsh C"] &
(A'\smsh C\pmap B\smsh C)
\end{tikzcd}
\begin{tikzcd}
(A\pmap B) \arrow[r,"({-})\smsh C"]\arrow[d,"A\pmap g"] &
(A\smsh C\pmap B\smsh C)\arrow[d,"A\smsh C\pmap g\smsh C"] \\
(A\pmap B') \arrow[r,"({-})\smsh C"] &
(A\smsh C\pmap B'\smsh C)
\end{tikzcd}
\begin{tikzcd}[column sep=large]
(A\pmap B) \arrow[r,"({-})\smsh C"]\arrow[d,"({-})\smsh C'"] &
(A\smsh C\pmap B\smsh C)\arrow[d,"A\smsh C\pmap B\smsh h"] \\
(A\smsh C'\pmap B\smsh C') \arrow[r,"A\smsh h\pmap B\smsh C'"] &
(A\smsh C\pmap B\smsh C')
\end{tikzcd}
\end{center}
\begin{proof}
First note that $\lam{f}f\smsh C$ preserves the basepoint so that the map is indeed pointed.

Let $k:A\pmap B$. Then as homotopy the naturality in $A$ becomes
$(k\o f)\smsh C=k\smsh C\o f\smsh C$. To prove an equality between pointed maps, we need to give
a pointed homotopy, which is given by interchange. To show that this homotopy is pointed, we need to
fill the following square (after reducing out the applications of function extensinality), which follows from \autoref{lem:smash-coh}.
\begin{center}
\begin{tikzcd}
(0 \o f)\smsh C \arrow[r, equals]\arrow[dd,equals] &
(0 \smsh C)\o (f \smsh C)\arrow[d,equals] \\
& 0 \o (f \smsh C)\arrow[d,equals] \\
0\smsh C \arrow[r,equals] &
0
\end{tikzcd}
\end{center}
The naturality in $B$ is almost the same: for the underlying homotopy we need to show
$(g \o k)\smsh C = g\smsh C \o k\smsh C$. For the pointedness we need to fill the following
square, which follows from \autoref{lem:smash-coh}.
\begin{center}
\begin{tikzcd}
(g \o 0)\smsh C \arrow[r, equals]\arrow[dd,equals] &
(g \smsh C)\o (0 \smsh C)\arrow[d,equals] \\
& (g\smsh C) \o 0\arrow[d,equals] \\
0\smsh C \arrow[r,equals] &
0
\end{tikzcd}
\end{center}
The dinaturality in $C$ is a bit harder. For the underlying homotopy we need to show
$B\smsh h\o k\smsh C=k\smsh C'\o A\smsh h$. This follows by applying interchange twice:
$$B\smsh h\o k\smsh C\sim(\idfunc[B]\o k)\smsh(h\o\idfunc[C])\sim(k\o\idfunc[A])\smsh(\idfunc[C']\o h)\sim k\smsh C'\o A\smsh h.$$
To show that this homotopy is pointed, we need to fill the following square:
\begin{center}
  \begin{tikzcd}
    B\smsh h\o 0\smsh C \arrow[r, equals]\arrow[d,equals] &
    (\idfunc[B]\o 0)\smsh(h\o\idfunc[C]) \arrow[r, equals]\arrow[d,equals] &
    (0\o\idfunc[A])\smsh(\idfunc[C']\o h)\arrow[r, equals]\arrow[d,equals] &
    0\smsh C'\o A\smsh h\arrow[d,equals] \\
    B\smsh h\o 0 \arrow[d,equals] &
    0\smsh(h\o\idfunc[C]) \arrow[r, equals]\arrow[d,equals] &
    0\smsh(\idfunc[C']\o h) \arrow[d,equals] &
    0\o A\smsh h\arrow[d,equals] \\
    B\smsh h\o 0 \arrow[r, equals] &
    0 \arrow[r, equals] &
    0 \arrow[r, equals] &
    0
  \end{tikzcd}
\end{center}
The left and the right squares are filled by \autoref{lem:smash-coh}. The squares in the middle
are filled by (corollaries of) \autoref{lem:smash-general}.
\end{proof}

\section{Adjunction}

\begin{lem}\label{lem:unit-counit}
  There is a unit $\eta_{A,B}\equiv\eta:A\pmap B\pmap A\smsh B$ natural in $A$ and counit
  $\epsilon_{B,C}\equiv\epsilon : (B\pmap C)\smsh B \pmap C$ dinatural in $B$ and natural in $C$.
  These maps satisfy the unit-counit laws:
  $$(A\to\epsilon_{A,B})\o \eta_{A\to B,A}\sim \idfunc[A\to B]\qquad
  \epsilon_{B,B\smsh C}\o \eta_{A,B}\smsh B\sim\idfunc[A\smsh B].$$
\end{lem}
Note: $\eta$ is also dinatural in $B$, but we don't need this.
\begin{proof}
  We define $\eta (a)(b)=(a,b)$. Then $\eta (a)$ respects the basepoint because
  $(a,b_0)=(a_0,b_0)$. Also, $\eta$ itself respects the basepoint. To show this, we need to show
  that $\eta (a_0)\sim 0$. The underlying maps are homotopic, since $(a_0,b)=(a_0,b_0)$. To show that
  this homotopy is pointed, we need to show that the two given proofs of $(a_0,b_0)=(a_0,b_0)$ are
  equal, but they are both equal to reflexivity:
  $$\gluel_{a_0}\tr\gluel_{a_0}\sy=1=\gluer_{b_0}\tr\gluer_{b_0}\sy.$$
  This defines the unit. To define the counit, given $x:(B\pmap C)\smsh B$, we construct
  $\epsilon (x):C$ by induction on $x$. If $x\jdeq(f,b)$ we set $\epsilon(f,b)\defeq f(b)$. If $x$
  is either $\auxl$ or $\auxr$ then we set $\epsilon (x)\defeq c_0:C$. If $x$ varies over $\gluel_f$
  then we need to show that $f(b_0)=c_0$, which is true by $f_0$. If $x$ varies over $\gluer_b$ we
  need to show that $0(b)=c_0$ which is true by reflexivity. The map $\epsilon$ defined as such is trivially a pointed map,
  which defines the counit.

  Now we need to show that the unit and counit are (di)natural. (TODO).

  Finally, we need to show the unit-counit laws. For the underlying homotopy of the first one, let
  $f:A\to B$. We need to show that $p:\epsilon\o\eta f\sim f$. For the underlying homotopy of $p$,
  let $a:A$, and we need to show that $\epsilon(f,a)=f(a)$, which is true by reflexivity. To show
  that $p$ is a pointed homotopy, we need to show that
  $p(a_0)\tr f_0=\mapfunc{\epsilon}(\eta f)_0\tr \epsilon_0$, which reduces to
  $f_0=\mapfunc{\epsilon}(\gluel_f\tr\gluel_0\sy)$, but we can reduce the right hand side: (note:
  $0_0$ denotes the proof that $0(a_0)=b_0$, which is reflexivity)
  $$\mapfunc{\epsilon}(\gluel_f\tr\gluel_0\sy)=\mapfunc{\epsilon}(\gluel_f)\tr(\mapfunc{\epsilon}(\gluel_0))\sy=f_0\tr 0_0\sy=f_0.$$
  Now we need to show that $p$ itself respects the basepoint of $A\to B$, i.e. that the composite
  $\epsilon\o\eta(0)\sim\epsilon\o0\sim0$ is equal to $p$ for $f\equiv 0_{A,B}$. The underlying
  homotopies are the same for $a : A$; on the one side we have
  $\mapfunc{\epsilon}(\gluer_{a}\tr\gluer_{a_0}\sy)$ and on the other side we have reflexivity
  (note: this typechecks, since $0_{A,B}a\equiv0_{A,B}a_0$). These paths are equal, since
  $$\mapfunc{\epsilon}(\gluer_{a}\tr\gluer_{a_0}\sy)=\mapfunc{\epsilon}(\gluer_{a})\tr(\mapfunc\epsilon(\gluer_{a_0}))\sy=1\cdot1\sy\equiv1.$$
  Both pointed homotopies are pointed in the same way, which requires some path-algebra, and we skip
  the proof here.

  For the underlying homotopy of the second one, we need to show for $x:A\smsh B$ that
  $\epsilon(\eta\smsh B(x))=x$, which we prove by induction to $x$. (TODO).
\end{proof}

\begin{defn}
The function $e\jdeq e_{A,B,C}:(A\pmap B\pmap C)\pmap(A\smsh B\pmap C)$ is defined as the composite
$$(A\pmap B\pmap C)\lpmap{({-})\smsh B}(A\smsh B\pmap (B\pmap C)\smsh B)\lpmap{A\smsh B \pmap\epsilon}(A\smsh B\pmap C).$$
\end{defn}

\begin{lem}
  The function $e$ is invertible, hence gives a pointed equivalence $$(A\pmap B\pmap C)\simeq(A\smsh B\pmap C).$$
\end{lem}
\begin{proof}
  Define
  $$\inv{e}_{A,B,C}:(A\smsh B\pmap C)\lpmap{B\pmap({-})}((B\pmap A\smsh B)\pmap (B\pmap
  C))\lpmap{\eta\pmap(B\pmap C)}(A\pmap B\pmap C).$$ It is easy to show that $e$ and $\inv{e}$ are
  inverses as unpointed maps from the unit-counit laws (\autoref{lem:unit-counit}) and naturality of $\eta$ and $\epsilon$.
%   For $f : A\pmap B\pmap C$ we have
%   \begin{align*}
%     \inv{e}(e(f))&\equiv(\eta\pmap(B\pmap C))\o (B\pmap((A\smsh B\pmap\epsilon)\of\smsh B))\\
%                  &= (\eta\pmap(B\pmap C))\o (B\pmap(A\smsh B\pmap\epsilon))\o(B\pmapf\smsh B)\\
% %                 &= (\eta\pmap(B\pmap C))\o (B\pmap(A\smsh B\pmap\epsilon))\o(B\pmapf\smsh B)\\
%   \end{align*}
\end{proof}
\begin{lem}\label{lem:e-natural}
	The function $e$ is natural in $A$, $B$ and $C$.
\end{lem}
\begin{proof}
	\textbf{Naturality of $e$ in $A$}. Suppose that $f:A'\pmap A$. Then the following diagram commutes. The left square commutes by naturality of $({-})\smsh B$ in the first argument and the right square commutes because composition on the left commutes with composition on the right.
	\begin{center}
	\begin{tikzcd}
		(A\pmap B\pmap C) \arrow[r,"({-})\smsh B"]\arrow[d,"f\pmap B\pmap C"] &
		(A\smsh B\pmap (B\pmap C)\smsh B) \arrow[r,"A\smsh B\pmap\epsilon"]\arrow[d,"f\smsh B\pmap\cdots"]  &
		(A\smsh B\pmap C)\arrow[d,"f\smsh B\pmap C"] \\
		(A'\pmap B\pmap C) \arrow[r,"({-})\smsh B"] &
		(A'\smsh B\pmap (B\pmap C)\smsh B) \arrow[r,"A\smsh B\pmap\epsilon"] &
		(A'\smsh B\pmap C)
	\end{tikzcd}
	\end{center}

	\textbf{Naturality of $e$ in $C$}. Suppose that $f:C\pmap C'$. Then in the following diagram the left square commutes by naturality of $({-})\smsh B$ in the second argument (applied to $B\pmap f$) and the right square commutes by applying the functor $A\smsh B \pmap({-})$ to the naturality of $\epsilon$ in the second argument.
	\begin{center}
	\begin{tikzcd}
		(A\pmap B\pmap C) \arrow[r]\arrow[d] &
		(A\smsh B\pmap (B\pmap C)\smsh B) \arrow[r]\arrow[d] &
		(A\smsh B\pmap C)\arrow[d] \\
		(A\pmap B\pmap C') \arrow[r] &
		(A\smsh B\pmap (B\pmap C')\smsh B) \arrow[r] &
		(A\smsh B\pmap C')
	\end{tikzcd}
	\end{center}

	\textbf{Naturality of $e$ in $B$}. Suppose that $f:B'\pmap B$. Here the diagram is a bit more
complicated, since $({-})\smsh B$ is dinatural (instead of natural) in $B$. Then we get the
following diagram. The front square commutes by naturality of $({-})\smsh B$ in the second argument
	(applied to $f\pmap C$). The top square commutes by naturality of $({-})\smsh B$ in the third
argument, the back square commutes because composition on the left commutes with composition on the
	right, and finally the right square commutes by applying the functor $A\smsh B' \pmap({-})$ to the
	naturality of $\epsilon$ in the first argument.
	\begin{center}
	\begin{tikzcd}[row sep=scriptsize, column sep=-4em]
		& (A\smsh B\pmap (B\pmap C)\smsh B) \arrow[rr] \arrow[dd] & & (A\smsh B'\pmap (B\pmap C)\smsh B)\arrow[dd] \\
		(A\pmap B\pmap C) \arrow[ur] \arrow[rr, crossing over] \arrow[dd] & & (A\smsh B'\pmap (B\pmap C)\smsh B') \arrow[ur] \\
		& (A\smsh B\pmap C)\arrow[rr] &  & (A\smsh B'\pmap C) \\
		(A\pmap B'\pmap C) \arrow[rr] & & (A\smsh B'\pmap (B'\pmap C)\smsh B') \arrow[ur] \arrow[from=uu, crossing over]
	\end{tikzcd}
	\end{center}

\end{proof}
\begin{rmk}
  Instead of showing that $e$ is natural, we could instead show that $e^{-1}$ is natural. In
  that case we need to show that the map $A\to({-}):(B\to C)\to(A\to B)\to(A\to C)$ is natural in
  $A$, $B$ and $C$. This might actually be easier, since we don't need to work with any higher
  inductive type to prove that.
\end{rmk}

\begin{lem}\label{lem:e-pointed-natural}
	The function $e$ is pointed natural in $C$.
\end{lem}

\begin{proof}
	(TODO)
\end{proof}

\section{Symmetric monoidality}
We aim to prove that the smash product is a (1-coherent) symmetric monoidal product [REF: Brunerie] for pointed types, i.e., that
\[(\pType,\, \bool,\, \smsh,\, \alpha,\, \lambda,\, \rho,\, \gamma)\]
is a symmetric monoidal category, with the type of booleans $\bool$ (pointed in $0_\bool$) as unit, and for suitable instances of $\alpha$, $\lambda$, $\rho$ and $\gamma$ witnessing associativity, left- and right unitality and the braiding for $\smsh$ and satisfying appropriate coherence relations (associativity pentagon; unitors triangle; braiding-unitors triangle; associativity-braiding hexagon; double braiding).

Using \autoref{lem:yoneda} (Yoneda) we can prove associativity, left- and right unitality and braiding equivalences for the smash product, in the following way.

\begin{defn}\label{def:equiv-precursors}
	The following pointed equivalences are defined for $A$, $B$, $C$ and $X$ pointed types:
	\begin{itemize}
		\item $\alphabar_X : (A \smsh (B \smsh C) \to X) \simeq ((A \smsh B) \smsh C \to X)$ as the composition of the equivalences:
			\begin{align*}
			    A \smsh (B \smsh C)\to X&\simeq A \to B\smsh C\to X && (e\sy)\\
			    &\simeq A \to B\to C\to X && (A \to e\sy)\\
			    &\simeq A \smsh B\to C\to X && (e)\\
		    	&\simeq (A \smsh B)\smsh C\to X. && (e)
			\end{align*}
		\item $\lambdabar_X : (B \to X) \simeq (\bool \smsh B \to X)$, where $\bool$ is the type of booleans, as the composition of the equivalences:
			\begin{align*}
				B \to X &\simeq \bool \to B \to X && (\two\sy)\\
				&\simeq \bool \smsh B \to X && (e)
			\end{align*}
		\item $\rhobar_X : (A \to X) \simeq (A \smsh \bool \to X)$ as the composition of the equivalences:
			\begin{align*}
				A \to X &\simeq A \to \bool \to X && (A \to \two\sy)\\
				&\simeq A \smsh \bool \to X && (e)
			\end{align*}
		\item $\gammabar_X : (B \smsh A \to X) \simeq (A \smsh B \to X)$ as the composition of the equivalences:
			\begin{align*}
				B \smsh A \to X &\simeq B \to A \to X && (e\sy)\\
				&\simeq A \to B \to X && (\twist)\\
				&\simeq A \smsh B \to X && (e)
			\end{align*}
	\end{itemize}
\end{defn}

\begin{rmk}\label{rmk:alrg-pointed-natural}
	The equivalences in \autoref{def:equiv-precursors} are natural in all their arguments and pointed natural in $X$, by (pointed) naturality of $e$ (\autoref{lem:e-natural} and \autoref{lem:e-pointed-natural}), $c$ and $t$. In particular, we will use:
	\begin{align*}
	f \o \alphabar(g) &\sim \alphabar(f \o g) & f \o \lambdabar(g) &\sim \lambdabar(f \o g)\\
	f \o \rhobar(g) &\sim \rhobar(f \o g) & f \o \gammabar(g) &\sim \gammabar(f \o g)
	\end{align*}
\end{rmk}

\begin{defn}\label{def:smash-alrg}
	We define the following equivalences, natural in all their arguments, with inverses provided as in \autoref{lem:yoneda}:
	\begin{itemize}
		\item $\alpha\defeq\alphabar_{A \smsh (B \smsh C)}(\idfunc) : (A \smsh B) \smsh C \simeq A \smsh (B \smsh C)$ (associativity of the smash product), with inverse $\alpha\sy\defeq\alphabar\sy_{(A \smsh B) \smsh C}(\idfunc)$;
		\item $\lambda \defeq \lambdabar_B(\idfunc) : \bool \smsh B \simeq B$ and $\rho \defeq \rhobar_A(\idfunc) : A \smsh \bool \simeq A$ (left- and right unitors for the smash product), with inverses $\lambda\sy\defeq \lambdabar_{\bool\smsh B}\sy(\idfunc)$ and $\rho\sy\defeq \rhobar_{A\smsh \bool}\sy(\idfunc)$, respectively;
		\item $\gamma \defeq \gammabar_{B\smsh A} (\idfunc) : A \smsh B \simeq B \smsh A$ (braiding for the smash product), with inverse $\gamma\sy \defeq \gammabar_{A \smsh B}\sy (\idfunc)$.
	\end{itemize}
	$\alpha$, $\lambda$, $\rho$ and $\gamma$ are natural in all their arguments, as $\alphabar$, $\lambdabar$, $\rhobar$ and $\gammabar$ are.
\end{defn}

\begin{lem}\label{lem:bar-homotopy}
	There are pointed homotopies
	\begin{align*}
	\alphabar_X &\sim \alpha \to X
		& \lambdabar_X &\sim \lambda \to X
	\\
	\rhobar_X &\sim \rho \to X
		& \gammabar_X &\sim \gamma \to X
	\end{align*}
\end{lem}

\begin{proof}
	This follows directly from \autoref{lem:yoneda-pointed} and \autoref{rmk:alrg-pointed-natural}.
\end{proof}

\begin{thm}[Associativity pentagon]\label{thm:smash-associativity-pentagon}
	For $A$, $B$, $C$ and $D$ pointed types, there is a homotopy
	\[\alpha \o \alpha \sim (A \smsh \alpha) \o \alpha \o (\alpha \smsh D)\]
	corresponding to the commutativity of the following diagram:
	\begin{center}
	\begin{tikzcd}
		&((A \smsh B) \smsh (C \smsh D))
			\arrow[dr, "\alpha"]
		\\
		(((A \smsh B) \smsh C) \smsh D)
			\arrow[ru, "\alpha"]
			\arrow[d, swap, "\alpha \smsh D"]
		&& (A \smsh (B \smsh (C \smsh D)))
		\\
		((A \smsh (B \smsh C)) \smsh D)
			\arrow[rr, swap, "\alpha"]
		&& (A \smsh ((B \smsh C) \smsh D))
			\arrow[u, swap, "A \smsh \alpha"]
	\end{tikzcd}
%	\begin{tikzcd}
%		(((A \smsh B) \smsh C) \smsh D)
%			\arrow[rr, "\alpha"]
%			\arrow[d, swap, "\alpha \smsh D"]
%		&& ((A \smsh B) \smsh (C \smsh D))
%			\arrow[d, "\alpha"]
%		\\
%		((A \smsh (B \smsh C)) \smsh D)
%			\arrow[r, swap, "\alpha"]
%		& (A \smsh ((B \smsh C) \smsh D))
%			\arrow[r, swap, "A \smsh \alpha"]
%		& (A \smsh (B \smsh (C \smsh D)))
%	\end{tikzcd}
	\end{center}
\end{thm}
\begin{proof}
	We articulate the proof in several steps. A map homotopic to both sides of the sought homotopy will be constructed via the equivalence
	\begin{align*}	
		\alphabar^4 : (A \smsh (B \smsh (C \smsh D)) \to X) &\simeq (((A \smsh B) \smsh C) \smsh D \to X)
		\intertext{(natural in all its arguments), defined as the composite:}
		A \smsh (B \smsh (C \smsh D)) \to X
		&\simeq A \to B \smsh (C \smsh D) \to X && \text{($e\sy$)}\\
		&\simeq A \to B \to C \smsh D \to X &&\text{($A \to e\sy$)}\\
		&\simeq A \to B \to C \to D \to X &&\text{($A \to B \to e\sy$)}\\
		&\simeq A \smsh B \to C \to D \to X &&\text{($e$)}\\
		&\simeq (A \smsh B) \smsh C \to D \to X &&\text{($e$)}\\
		&\simeq ((A \smsh B) \smsh C) \smsh D \to X && \text{($e$)}
		\intertext{giving $\alphabar^4(\idfunc) : ((A \smsh B) \smsh C) \smsh D) \simeq A \smsh (B \smsh (C \smsh D))$. Moreover, in order to simplify the expressions of $\alpha \smsh D$ and $A \smsh \alpha$, we also define:}
		\alphabar^R : ((A \smsh (B \smsh C)) \smsh D \to X) &\simeq (((A \smsh B) \smsh C) \smsh D \to X)
		\intertext{as the composite:}
		(A \smsh (B \smsh C)) \smsh D \to X
		&\simeq A \smsh (B \smsh C) \to D \to X &&\text{($e\sy$)}\\
		&\simeq (A \smsh B) \smsh C \to D \to X &&\text{($\alphabar$)}\\
		&\simeq ((A \smsh B) \smsh C) \smsh D \to X &&\text{($e$)}
		\intertext{and}
		\alphabar^L : (A \smsh (B \smsh (C \smsh D)) \to X) &\simeq (A \smsh ((B \smsh C) \smsh D) \to X)
		\intertext{as the composite:}
		A \smsh (B \smsh (C \smsh D)) \to X
		&\simeq A \to B \smsh (C \smsh D) \to X &&\text{($e\sy$)}\\
		&\simeq A \to (B \smsh C) \smsh D \to X &&\text{($A \to \alphabar$)}\\
		&\simeq A \smsh ((B \smsh C) \smsh D) \to X &&\text{($e$)}
	\end{align*}
	also natural in their arguments. Evaluating these equivalences to the identity function, we get new arrows that fit in the original diagram:
	\begin{center}
	\begin{tikzcd}
		&((A \smsh B) \smsh (C \smsh D))
			\arrow[dr, "\alpha"]
		\\
		(((A \smsh B) \smsh C) \smsh D)
			\arrow[ru, "\alpha"]
			\arrow[d, swap, "\alpha \smsh D"]
			\arrow[d, bend left=40, "\alphabar^R(\idfunc)"]
			\arrow[rr, "\alphabar^4(\idfunc)"]
		&& (A \smsh (B \smsh (C \smsh D)))
		\\
		((A \smsh (B \smsh C)) \smsh D)
			\arrow[rr, swap, "\alpha"]
		&& (A \smsh ((B \smsh C) \smsh D))
			\arrow[u, swap, "A \smsh \alpha"]
			\arrow[u, bend left=40, "\alphabar^L(\idfunc)"]
	\end{tikzcd}
	\end{center}
	
	The theorem is then proved once we show the chain of homotopies:
	\begin{equation}\label{eq:alphafour}
	\alpha \o \alpha
	\sim \alphabar^4(\idfunc)
	\sim \alphabar^L(\idfunc) \o \alpha \o \alphabar^R(\idfunc)
	\sim (A \smsh \alpha) \o \alpha \o (\alpha \smsh D)
	\end{equation}
	
	To verify the first homotopy in (\ref{eq:alphafour}), we see that:
	\begin{align*}
		\alpha \o \alpha
		&\judgeq \alphabar(\idfunc) \o \alphabar(\idfunc)\\
		&\sim (\alphabar \o \alphabar) (\idfunc) &&\text{(naturality of $\alphabar$)}\\
		&\judgeq (e \o e \o (A \to e\sy) \o e\sy \o e \o e \o (A \to e\sy) \o e\sy)(\idfunc)\\
		&\sim (e \o e \o (A \to e\sy) \o e \o (A \to e\sy) \o e\sy)(\idfunc) &&\text{(cancelling)}\\
		&\sim (e \o e \o e \o (B \to A \to e\sy) \o (A \to e\sy) \o e\sy)(\idfunc) &&\text{(naturality of $e$)}\\
		&\judgeq \alphabar^4(\idfunc)
	\end{align*}		

	The second homotopy in (\ref{eq:alphafour}) is verified by (right-to-left):
	\begin{align*}
		\alphabar^L(\idfunc) \o \alpha \o \alphabar^R(\idfunc)
		&\judgeq \alphabar^L(\idfunc) \o \alphabar(\idfunc) \o \alphabar^R(\idfunc)\\
		&\sim (\alphabar^R \o \alphabar \o \alphabar^L)(\idfunc) &&\text{(nat. of $\alphabar$ and $\alphabar^R$)}\\
		&\judgeq (e \o \alphabar \o e\sy \o e \o e \o (A \to e\sy) \o e\sy \o e \o (A \to \alphabar) \o e\sy)(\idfunc)\\
		&\sim (e \o \alphabar \o e \o (A \to e\sy) \o (A \to \alphabar) \o e\sy)(\idfunc) &&\text{(cancelling)}\\
		&\sim (e \o \alphabar \o e \o (A \to (e\sy \o \alphabar)) \o e\sy)(\idfunc) &&\text{(funct. of $A\to -$)}\\
		&\judgeq (e \o e \o e \o (A \to e\sy) \o e\sy \o e \\
		&\hspace{3em}\o (A \to (e\sy \o e \o e \o (B \to e\sy) \o e\sy)) \o e\sy)(\idfunc)\\
		&\sim (e \o e \o e \o (A \to ((B \to e\sy) \o e\sy)) \o e\sy)(\idfunc) &&\text{(cancelling)}\\
		&\sim (e \o e \o e \o (B \to A \to e\sy) \o (A \to e\sy) \o e\sy)(\idfunc) &&\text{(funct. of $A \to -$)}\\
		&\judgeq \alphabar^4(\idfunc)
	\end{align*}
	
	In order to prove the last homotopy in (\ref{eq:alphafour}), it is sufficient to show that $\alphabar^R(\idfunc) \sim \alpha \smsh D$ and that $\alphabar^L(\idfunc) \sim A \smsh \alpha$. We have:
	\begin{align*}
		\alphabar^R(\idfunc)
		&\judgeq e(\alphabar (e\sy(\idfunc)))\\
		&\sim e (\alphabar (\eta))\\
		&\sim e (\eta \o \alphabar(\idfunc)) &&\text{(naturality of $\alphabar$)}\\
		&\judgeq \epsilon \o (\eta \o \alpha) \smsh D\\
		&\sim \epsilon \o (\eta \smsh D) \o (\alpha \smsh D) &&\text{(distrib. of $\smsh$)}\\
		&\sim \alpha \smsh D &&\text{(\autoref{lem:unit-counit})}
	\end{align*}
	and, lastly,
	\begin{align*}
		\alphabar^L(\idfunc)
		&\judgeq e(\alphabar \o e\sy(\idfunc))\\
		&\sim e(\alphabar \o \eta)\\
		&\sim e((\alpha \to A \smsh (B \smsh (C \smsh D))) \o \eta) &&\text{(\autoref{lem:bar-homotopy})}\\
		&\sim e((B \smsh (C \smsh D) \to A \smsh \alpha) \o \eta) &&\text{(dinaturality of $\eta$)}\\
		&\sim (A \smsh \alpha) \o e(\eta) &&\text{(naturality of $e$)}\\
		&\sim A \smsh \alpha &&\text{(\autoref{lem:unit-counit})}
	\end{align*}
	thus proving the desired homotopy.
\end{proof}

\begin{thm}[Unitors triangle]\label{thm:smash-unitors-triangle}
	For $A$ and $B$ pointed types, there is a homotopy
	\[(A \smsh \lambda) \o \alpha \sim (\rho \smsh B)\]
	corresponding to the commutativity of the following diagram:
	\begin{center}
	\begin{tikzcd}
	((A \smsh \bool) \smsh B)
		\arrow[rr, "\alpha"]
		\arrow[dr, swap, "\rho \smsh B"]
	&& (A \smsh (\bool \smsh B))
		\arrow[dl, "A \smsh \lambda"]
	\\
	& (A \smsh B)
	\end{tikzcd}
	\end{center}
\end{thm}
\begin{proof}
	By an argument similar to the one for $\alphabar^L$ and $\alphabar^R$ in \autoref{thm:smash-associativity-pentagon}, one can verify the homotopies $A \smsh \lambda \sim (e \o (A \to \lambdabar) \o e\sy)(\idfunc)$ and $\rho \smsh B \sim (e \o \rhobar \o e)(\idfunc)$, simplifying the expressions in the sought homotopy. Then:
	\begin{align*}
		(A \smsh \lambda) \o \alpha
		&\sim e(\lambdabar \o e\sy(\idfunc)) \o \alphabar(\idfunc) &&\text{(simplification)}\\
		&\sim \alphabar(e(\lambdabar \o e\sy(\idfunc)) &&\text{(naturality of $\alphabar$)}\\
		&\judgeq e(e(e\sy \o e\sy (e(\lambdabar \o e\sy(\idfunc)))))\\
		&\sim e(e(e\sy \o \lambdabar \o e\sy(\idfunc))) &&\text{(cancelling)}\\
		&\judgeq e(e(e\sy \o e \o \two\sy \o e\sy(\idfunc)))\\
		&\sim e(e(\two\sy \o e\sy(\idfunc))) &&\text{(cancelling)}\\
		&\judgeq (e \o \rhobar \o e\sy)(\idfunc)\\
		&\sim \rho \smsh B &&\text{(simplification)}
	\end{align*}
	gives the desired homotopy.
\end{proof}

\begin{thm}[Braiding-unitors triangle]\label{thm:smash-braiding-unitors}
	For a pointed type $A$, there is a homotopy
	\[\lambda \o \gamma \sim \rho\]
	corresponding to the commutativity of the following diagram:
	\begin{center}
	\begin{tikzcd}
	(A \smsh \bool)
		\arrow[rr, "\gamma"]
		\arrow[dr, swap, "\rho"]
	&& (\bool \smsh A)
		\arrow[dl, "\lambda"]
	\\
	& A
	\end{tikzcd}
	\end{center}
\end{thm}
\begin{proof}
	We have:
	\begin{align*}
		\lambda \o \gamma
		&\judgeq \lambdabar(\idfunc) \o \gammabar(\idfunc)\\
		&\sim (\gammabar \o \lambdabar)(\idfunc) &&\text{(naturality of $\gammabar$)}\\
		&\judgeq (e \o \twist \o e\sy \o e \o \two\sy)(\idfunc)\\
		&\sim (e \o \twist \o \two\sy)(\idfunc) &&\text{(cancelling)}\\
		&\sim (e \o (A \to \two\sy))(\idfunc)\\
		&\judgeq \rhobar(\idfunc) \judgeq \rho
	\end{align*}		
	where the last homotopy is given by $(A \to c) \o \two \sim \twist : (\bool \to A \to X) \to (A \to X)$.
\end{proof}

\begin{lem}\label{lem:pentagon-c}
	The following diagram commutes, for $A$, $B$, $C$ and $X$ pointed types:
	\begin{center}
	\begin{tikzcd}[column sep=7em]
		(B \smsh C \to A \to X)
			\arrow[r, "e\sy"]
			\arrow[dd, swap, "\twist"]
		& (B \to C \to A \to X)
			\arrow[d, "B \to \twist"]
		\\
		& (B \to A \to C \to X)
			\arrow[d, "\twist"]
		\\
		(A \to B \smsh C \to X)
			\arrow[r, swap, "A \to e\sy"]
		& (A \to B \to C \to X)
	\end{tikzcd}
	\end{center}
\end{lem}
\begin{proof}
	Unfolding the definition of $e\sy$, we get the diagram:
	\begin{center}
	\begin{tikzcd}[column sep=6em]
		(B \smsh C \to A \to X)
			\arrow[rr, bend left=10, "e\sy"]
			\arrow[r, swap, "C\to -"]
			\arrow[dd, swap, "\twist"]
		& ((C \to B \smsh C) \to C \to A \to X)
			\arrow[r, swap, "\eta \to C \to A \to X"]
			\arrow[d, swap, "(C \to B \smsh C) \to \twist"]
		& (B \to C \to A \to X)
			\arrow[d, "B \to \twist"]
		\\
		& ((C \to B \smsh C) \to A \to C \to X)
			\arrow[r, swap, "\eta \to A \to C \to X"]
			\arrow[d, swap, "\twist"]
		& (B \to A \to C \to X)
			\arrow[d, "\twist"]
		\\
		(A \to B \smsh C \to X)
			\arrow[rr, swap, bend right=10, "A \to e\sy"]
			\arrow[r, "A \to (C \to -)"]
		& (A \to (C \to B \smsh C) \to C \to X)
			\arrow[r, "A \to (\eta \to C \to X)"]
		& (A \to B \to C \to X)
	\end{tikzcd}
	\end{center}
	where the squares on the right are instances of naturality of $\twist$, while the commutativity of the pentagon on the left follows easily from the definition of $\twist$.
\end{proof}


\begin{thm}[Associativity-braiding hexagon]\label{thm:smash-associativity-braiding}
	For pointed types $A$, $B$ and $C$, there is a homotopy
	\[\alpha \o \gamma \o \alpha \sim (B \smsh \gamma) \o \alpha \o (\gamma \smsh C)\]
	corresponding to the commutativity of the following diagram:
	\begin{center}
	\begin{tikzcd}
		((A \smsh B) \smsh C)
			\arrow[r, "\alpha"]
			\arrow[d, swap, "\gamma \smsh C"]
		&(A \smsh (B \smsh C))
			\arrow[r, "\gamma"]
		& ((B \smsh C) \smsh A)
			\arrow[d, "\alpha"]
		\\
		((B \smsh A) \smsh C))
			\arrow[r, swap, "\alpha"]
		& (B \smsh (A \smsh C))
			\arrow[r, swap, "B \smsh \gamma"]
		& (B \smsh (C \smsh A))
	\end{tikzcd}
	\end{center}
\end{thm}
\begin{proof}
	The proof is structured similarly to the one for \autoref{thm:smash-associativity-pentagon}: the homotopies
	\begin{align*}
	B \smsh \gamma &\sim \gammabar^L(\idfunc) &\text{with\ \ } \gammabar^L &\defeq e \o (B \to \gammabar) \o e\sy\\
	\gamma \smsh C &\sim \gammabar^R(\idfunc) &\text{with\ \ } \gammabar^R &\defeq e \o \gammabar \o e\sy
	\end{align*}		
	can be proven in exactly the same way and, using these simplifications, we will show that both sides of the sought homotopy are homotopic to the same equivalence. Indeed we have:
	\begin{align*}
		\alpha \o \gamma \o \alpha
		&\judgeq \alphabar(\idfunc) \o \gammabar(\idfunc) \o \alphabar(\idfunc)\\
		&\sim (\alphabar \o \gammabar \o \alphabar)(\idfunc) &&\text{(naturality of $\gammabar$ and $\alphabar$)}\\
		&\judgeq (e \o e \o (A \to e\sy) \o e\sy \o e \o \twist \o e\sy \o e \o e \o (B \to e\sy) \o e\sy)(\idfunc)\\
		&\sim (e \o e \o (A \to e\sy) \o \twist \o e \o (B \to e\sy) \o e\sy)(\idfunc) &&\text{(cancelling)}\\
		&\sim (e \o e \o \twist \o (B \to \twist) \o e\sy \o e \o (B \to e\sy) \o e\sy)(\idfunc) &&\text{(\autoref{lem:pentagon-c})}\\
		&\sim (e \o e \o \twist \o (B \to \twist) \o (B \to e\sy) \o e\sy)(\idfunc) &&\text{(cancelling)}
	\end{align*}
	and
	\begin{align*}
		(B \smsh \gamma) \o \alpha \o (\gamma \smsh C)
		&\sim \gammabar^L(\idfunc) \o \alphabar \o \gammabar^R(\idfunc) &&\text{(simplification)}\\
		&\sim (\gammabar^R \o \alphabar \o \gammabar^L)(\idfunc) &&\text{(naturality of $\alphabar$ and $\gammabar^R$)}\\
		&\judgeq (e \o \gammabar \o e\sy \o e \o e \o (B \to e\sy) \o e\sy \o e \o (B \to \gammabar) \o e\sy)(\idfunc)\\
		&\sim (e \o \gammabar \o e \o (B \to e\sy) \o (B \to \gammabar) \o e\sy)(\idfunc) &&\text{(cancelling)}\\
		&\sim (e \o \gammabar \o e \o (B \to (e\sy \o \gammabar)) \o e\sy)(\idfunc) &&\text{(funct. of $B \to -$)}\\
		&\judgeq (e \o e \o \twist \o e\sy \o e \o (B \to (e\sy \o e \o \twist \o e\sy)) \o e\sy)(\idfunc)\\
		&\sim (e \o e \o \twist \o (B \to \twist) \o (B \to e\sy) \o e\sy)(\idfunc) &&\text{(cancelling)}
	\end{align*}
	proving the commutativity of the diagram.
\end{proof}

\begin{thm}[Double braiding]\label{thm:smash-double-braiding}
	For $A$ and $B$ pointed types, there is a homotopy
	\[\gamma \o \gamma \sim \idfunc\]
	corresponding to the commutativity of the following diagram:
	\begin{center}
	\begin{tikzcd}
	(A \smsh B)
		\arrow[r, "\gamma"]
		\arrow[dr, equals]
	& (B \smsh A)
		\arrow[d, "\gamma"]
	\\
	& (A \smsh B)
	\end{tikzcd}
	\end{center}
\end{thm}
\begin{proof}
	Using that $\twist \o \twist \sim \idfunc$, we get:
	\begin{align*}
		\gamma \o \gamma
		&\judgeq \gammabar(\idfunc) \o \gammabar(\idfunc)\\
		&\sim (\gammabar \o \gammabar)(\idfunc) &&\text{(naturality of $\gammabar$)}\\
		&\judgeq (e \o \twist \o e\sy \o e \o \twist \o e\sy)(\idfunc)\\
		&\sim \idfunc &&\text{(cancelling)}
	\end{align*}
	as desired.
\end{proof}

\begin{cor}
	$(\pType,\, \bool,\, \smsh,\, \alpha,\, \lambda,\, \rho,\, \gamma)$ has a structure of a symmetric monoidal category, for $\alpha$, $\lambda$, $\rho$ and $\gamma$ as in \autoref{def:smash-alrg}.
\end{cor}
\begin{proof}
	Follows by the theorems in this section.
\end{proof}

\section{Discussion}

Many results in \autoref{sec:smash} about $({-}) \smsh B$ can be generalized to arbitrary pointed $\infty$-functors $\type^*\to\type^*$. Internally in HoTT we cannot talk about $\infty$-categories and $\infty$-functors (as far as we know), and therefore we cannot express this internally. But we only need finitely many coherences, so that we can formulate these results internally for arbitrary functors with enough coherences. However, when we tried this, we run into the problem that we needed all (or at least, some) coherences in the definition of a tricategory and functor for tricategories, which are harder to check than the result we wanted to prove. Here we sketch this argument.

$\type^*$ is a pointed $(\infty,1)$-category meaning it has an zero object $\unit$, which is an object that is both (homotopy-)initial and (homotopy-)terminal. Given two pointed $(\infty,1)$-categories $\mc{A}$ and $\mc{B}$, we call a map $F:\mc{A}\to\mc{B}$ a 1-coherent functor if it satisfies all the coherences of a functor (i.e. an action on morphisms and a 2-cell stating that $F$ respects compositions and identities), a 2-coherent functor if it satisfies all the coherences of a bifunctor (a functor between bicategories) and a 3-coherent functor if it satisfies all the coherences of a trifunctor (a functor between tricategories).
% Add reference:
% @article{gurski2006algebraic,
% title={An algebraic theory of tricategories},
% author={Gurski, Nick},
% year={2007},
% url={http://gauss.math.yale.edu/~mg622/tricats.pdf}
% }

We call $F:\mc{A}\to\mc{B}$ pointed if it preserves the zero object, i.e. if $F\unit=\unit$. If $F$ is a pointed 1-coherent functor, we can then show that $F(0_{A,B})=0_{FA,FB}$, where $0_{A,B}$ is the zero morphism (the canonical morphism $A\to\unit\to B$).

If $F$ is a pointed 2-coherent functor (or more precisely a 1-coherent functor which a coherence for the associativity 2-cell), then we can show that \autoref{lem:smash-coh} holds for $F$. We can show that the two pentagons have the same filler if $F$ is a 3-coherent functor. 

Now we can also show \autoref{thm:smash-functor-right} more generally, but we will only formulate that for functors between pointed types. If $F:\type^*\to\type^*$ is a 2-coherent pointed functor, then it induces a map
$(A\to B)\to(FA\to FB)$ which is natural in $A$ and $B$. Moreover, if $F$ is 3-coherent then this is a pointed natural transformation in $B$.

\end{document}


%%%%%%%%%%%%%%%%%%%%%%%%%%%%%%%%%%%%%%%%%%%%%%%%%%%%%%%%%%%%%%%%%
%%%%%%%%%%%%%%%%%%%%%%%%%%%%%%%%%%%%%%%%%%%%%%%%%%%%%%%%%%%%%%%%%
%%%%%%%%%%%%%%%%%%%%%%%%%%%%%%%%%%%%%%%%%%%%%%%%%%%%%%%%%%%%%%%%%
\begin{thm}[Associativity]\label{thm:smash-associativity}
  The smash product is associative: there is an equivalence $\alpha : (A \smsh B) \smsh C \simeq A \smsh (B \smsh C)$ which is natural in $A$, $B$ and $C$.
\end{thm}
\begin{proof}
  For a pointed type $X$, let $\alphabar_X$ be the composite of the following equivalences:
  \begin{align*}
    A \smsh (B \smsh C)\to X&\simeq A \to B\smsh C\to X && (e\sy)\\
    &\simeq A \to B\to C\to X && (A \to e\sy)\\
    &\simeq A \smsh B\to C\to X && (e)\\
    &\simeq (A \smsh B)\smsh C\to X. && (e)
  \end{align*}
  $\alphabar_X$ is natural in $A,B,C,X$ by repeatedly applying \autoref{lem:e-natural}. Let
  $\alpha\defeq\alphabar_{A \smsh (B \smsh C)}(\idfunc)$ and
  $\alpha\sy\defeq\alphabar\sy_{(A \smsh B) \smsh C}(\idfunc)$. These maps are inverses by \autoref{lem:yoneda}; lastly, $\alpha$ is natural in $A$, $B$
  and $C$, since $\alphabar_X$ is.
\end{proof}

\begin{thm}[Unitors]\label{thm:smash-unitors}
	The type of booleans $\bool$ is a left- and right unit for the smash product: there are equivalences
	$\lambda : (\bool \smsh B) \simeq B$ and $\rho : (A \smsh \bool) \simeq A$, respectively natural in $B$ and in $A$.
\end{thm}
\begin{proof}
	We define $\lambdabar_X$ as the following composition of equivalences:
	\begin{align*}
		B \to X &\simeq \bool \to B \to X && (t\sy)\\
			&\simeq \bool \smsh B \to X && (e)
		\end{align*}
	and, similarly, $\rhobar_X$ as the composition:
	\begin{align*}
		A \to X &\simeq A \to \bool \to X && (A \to t\sy)\\
			&\simeq A \smsh \bool \to X && (e)
	\end{align*}
	for any pointed type $X$, where $t : (\bool \to X) \simeq X$ is the pointed equivalence with underlying function sending $f : \bool \to X$ to $f(1_\bool) : X$. Both $\lambdabar_X$ and $\rhobar_X$ are natural in their arguments by multiple applications of \autoref{lem:e-natural}. By an argument similar to \autoref{thm:smash-associativity}, we can define $\lambda \defeq \lambdabar_B(\idfunc)$ and $\rho \defeq \rhobar_A(\idfunc)$ together with the corresponding inverses, yielding the sought natural equivalences.
\end{proof}

\begin{thm}[Braiding]\label{thm:smash-braiding}
	The smash product is symmetric, i.e. there are equivalences $\gamma : A \smsh B \simeq B \smsh A$, natural in $A$ and $B$.
\end{thm}
\begin{proof}
	We define $\gammabar_X$, for $X$ a pointed type, as the composition of the following equivalences:
	\begin{align*}
		B \smsh A \to X &\simeq B \to A \to X && (e\sy)\\
			&\simeq A \to B \to X && (c)\\
			&\simeq A \smsh B \to X && (e)
	\end{align*}
	where $c : (A \to B \to X) \simeq (B \to A \to X)$ is the obvious pointed equivalence, natural in $A$, $B$ and $X$. $\gamma_X$ is then also natural in all its arguments, by myltiple applications of \autoref{lem:e-natural}; then, as in \autoref{thm:smash-associativity}, we can define $\gamma \defeq \gammabar_{B\smsh A} (\idfunc)$ and $\gamma\sy \defeq \gammabar_{A \smsh B}\sy (\idfunc)$
\end{proof}
