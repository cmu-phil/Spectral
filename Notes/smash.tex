\documentclass{article}

% for type setting urls
\usepackage[hyphens]{url} % This package has to be loaded *before* hyperref
\usepackage[pagebackref,colorlinks,citecolor=darkgreen,linkcolor=darkgreen,unicode]{hyperref}
\usepackage[english]{babel}

%%% Because Germans have umlauts and Slavs have even stranger ways of mangling letters
\usepackage[utf8]{inputenc}

%%% Multi-Columns for long lists of names
\usepackage{multicol}

%%% Set the fonts
\usepackage{mathpazo}
\usepackage[scaled=0.95]{helvet}
\usepackage{courier}
\linespread{1.05} % Palatino looks better with this

\usepackage{graphicx}
\usepackage{comment}

%\usepackage{wallpaper} % For the background image on the cover page
%\usepackage{geometry} % For the cover page
\usepackage{fancyhdr} % To set headers and footers

\usepackage{ifthen}
\usepackage{amssymb,amsmath,amsthm,stmaryrd,mathrsfs,wasysym}
\usepackage{enumitem,mathtools,xspace,xcolor}
\definecolor{darkgreen}{rgb}{0,0.45,0}
\usepackage{aliascnt}
\usepackage[capitalize]{cleveref}
%\usepackage[all,2cell]{xy}
%\UseAllTwocells
% \usepackage{natbib}
\usepackage{braket} % used for \setof{ ... } macro
\usepackage{tikz-cd}
\usepackage{tikz}
\usetikzlibrary{decorations.pathmorphing}
\usepackage[inference]{semantic}
\usepackage{booktabs}

%%%%%%%%%%%%%%%%%%%%%%%%%%%%%%%%%%%%%%%%%%%%%%%%%%%%%%%%%%%%%%%%%%%%%%%%%%%%%%%%
%% To include references in TOC we should use this package rather than a hack.
\usepackage{tocbibind}
%\usepackage{etoolbox}           % get \apptocmd
%\apptocmd{\thebibliography}{\addcontentsline{toc}{section}{References}}{}{} % tell bibliography to get itself into the table of contents


\begin{comment}
%%%% Header and footers
\pagestyle{fancyplain}
\setlength{\headheight}{15pt}
\renewcommand{\chaptermark}[1]{\markboth{\textsc{Chapter \thechapter. #1}}{}}
\renewcommand{\sectionmark}[1]{\markright{\textsc{\thesection\ #1}}}
\end{comment}

% TOC depth
\setcounter{tocdepth}{2}

\lhead[\fancyplain{}{{\thepage}}]%
      {\fancyplain{}{\nouppercase{\rightmark}}}
\rhead[\fancyplain{}{\nouppercase{\leftmark}}]%
      {\fancyplain{}{\thepage}}
\cfoot{\textsc{\footnotesize [Draft of \today]}}
\lfoot[]{}
\rfoot[]{}

%%%%%%%%%%%%%%%%%%%%%%%%%%%%%%%%%%%%%%%%%%%%%%%%%%%%%%%%%%%%%%%%%%%%%%%%%%%%%%%%
%%%% We mostly use the macros of the book, to keep notations
%%%% and conventions the same. Recall that when the macros file
%%%% is updated, we need to comment the lines containing the
%%%% string `[chapter]` since our article is not a book.
%%%%
%%%% Instructions for updating the macros.tex file:
%%%% - fetch the latest macros.tex file from the HoTT/book git repository.
%%%% - comment all lines containing "[chapter]" because this is not a book.
%%%% - comment the definition of pbcorner because the xypic package is not used.
%%%%
\input{macros}

\newcommand{\idsymbin}{=}

%%%%%%%%%%%%%%%%%%%%%%%%%%%%%%%%%%%%%%%%%%%%%%%%%%%%%%%%%%%%%%%%%%%%%%%%%%%%%%%%
%%%% Our commands which are not part of the macros.tex file.
%%%% We should keep these commands separate, because we will
%%%% update the macros.tex following the updates of the book.

%%%% First we redefine the \id, \eqv and \ct commands so that they accept an
%%%% arbitrary number of arguments. This is useful when writing longer strings
%%%% of equalities or equivalences.

\makeatletter

\renewcommand{\id}[3][]{
  \@ifnextchar\bgroup
    {#2 \mathbin{\idsym_{#1}} \id[#1]{#3}}
    {#2 \mathbin{\idsym_{#1}} #3}
  }

\renewcommand{\eqv}[2]{
  \@ifnextchar\bgroup
    {#1 \eqvsym \eqv{#2}}
    {#1 \eqvsym #2}
  }

\newcommand{\ctsym}{%
  \mathchoice{\mathbin{\raisebox{0.5ex}{$\displaystyle\centerdot$}}}%
             {\mathbin{\raisebox{0.5ex}{$\centerdot$}}}%
             {\mathbin{\raisebox{0.25ex}{$\scriptstyle\,\centerdot\,$}}}%
             {\mathbin{\raisebox{0.1ex}{$\scriptscriptstyle\,\centerdot\,$}}}
  }

\renewcommand{\ct}[3][]{
  \@ifnextchar\bgroup
    {#2 \mathbin{\ctsym_{#1}} \ct[#1]{#3}}
    {#2 \mathbin{\ctsym_{#1}} #3}
  }

\makeatother

%%%% We always use textstyle products and sums...
%\renewcommand{\prd}{\tprd}
%\renewcommand{\sm}{\tsm}
\makeatletter
\renewcommand{\@dprd}{\@tprd}
\renewcommand{\@dsm}{\@tsm}
\renewcommand{\@dprd@noparens}{\@tprd}
\renewcommand{\@dsm@noparens}{\@tsm}

%%%% ...with a bit more spacing
\renewcommand{\@tprd}[1]{\mathchoice{{\textstyle\prod_{(#1)}\,}}{\prod_{(#1)}\,}{\prod_{(#1)}\,}{\prod_{(#1)}\,}}
\renewcommand{\@tsm}[1]{\mathchoice{{\textstyle\sum_{(#1)}\,}}{\sum_{(#1)}\,}{\sum_{(#1)}\,}{\sum_{(#1)}\,}}

%%%%%%%%%%%%%%%%%%%%%%%%%%%%%%%%%%%%%%%%%%%%%%%%%%%%%%%%%%%%%%%%%%%%%%%%%%%%%%%%
%%%% We adjust the \prd command so that implicit arguments become possible.
%%%%
%%%% First, we have the following switch. Set it to true if implicit arguments
%%%% are desired, or to false if not. Note turning off implicit arguments
%%%% might render some parts of the text harder to comprehend, since in the
%%%% text might appear $f(x)$ where we would have $f(i,x)$ without implicit
%%%% arguments.

\newcommand{\implicitargumentson}{\boolean{true}}

%%%% If one wants to use implicit arguments in the notation for product types,
%%%% a * has to be put before the argument that has to be implicit.
%%%% For example: in $\prd{x:A}*{y:B}{u:P(y)}Q(x,y,u)$, the argument y is
%%%% implicit. Any of the arguments can be made implicit this way.

%%%% First of all, we should make the command \prd search not only for a
%%%% brace, but also for a star. We introduce an auxiliary command that
%%%% determines whether the next character is a star or brace.
\newcommand{\@ifnextchar@starorbrace}[2]
%  {\@ifnextcharamong{#1}{#2}{*}{\bgroup};}
  {\@ifnextchar*{#1}{\@ifnextchar\bgroup{#1}{#2}}}
  
%%%% When encountering the \prd command, latex should determine whether it
%%%% should print implicit argument brackets or not. So the first branching
%%%% happens right here.
\renewcommand{\prd}{\@ifnextchar*{\@iprd}{\@prd}}

\newcommand{\@prd}[1]
  {\@ifnextchar@starorbrace
    {\prd@parens{#1}}
    {\@ifnextchar\sm{\prd@parens{#1}\@eatsm}{\prd@noparens{#1}}}}
\newcommand{\@prd@parens}{\@ifnextchar*{\@iprd}{\prd@parens}}
\renewcommand{\prd@parens}[1]
  {\@ifnextchar@starorbrace
    {\@theprd{#1}\@prd@parens}
    {\@ifnextchar\sm{\@theprd{#1}\@eatsm}{\@theprd{#1}}}}
\newcommand{\@theprd}[1]
  {\mathchoice{\@dprd{#1}}{\@tprd{#1}}{\@tprd{#1}}{\@tprd{#1}}}
\renewcommand{\dprd}[1]{\@dprd{#1}\@ifnextchar@starorbrace{\dprd}{}}
\renewcommand{\tprd}[1]{\@tprd{#1}\@ifnextchar@starorbrace{\tprd}{}}

%%%% Here we tell the actual symbols to be printed.
\newcommand{\@theiprd}[1]{\mathchoice{\@diprd{#1}}{\@tiprd{#1}}{\@tiprd{#1}}{\@tiprd{#1}}}
\newcommand{\@iprd}[2]{\@ifnextchar@starorbrace%
  {\@theiprd{#2}\@prd@parens}%
  {\@ifnextchar\sm%
    {\@theiprd{#2}\@eatsm}%
    {\@theiprd{#2}}}}
\def\@tiprd#1{
  \ifthenelse{\implicitargumentson}
    {\@@tiprd{#1}\@ifnextchar\bgroup{\@tiprd}{}}
    {\@tprd{#1}}}
\def\@@tiprd#1{\mathchoice{{\textstyle\prod_{\{#1\}}\,}}{\prod_{\{#1\}}\,}{\prod_{\{#1\}}\,}{\prod_{\{#1\}}\,}}
\def\@diprd{
  \ifthenelse{\implicitargumentson}
    {\@tiprd}
    {\@tprd}}
    

%%%% And finally we need to redefine \@eatprd so that implicit arguments also
%%%% works in the scope of a dependent sum.    
\def\@eatprd\prd{\@prd@parens}

\makeatother

%%%%%%%%%%%%%%%%%%%%%%%%%%%%%%%%%%%%%%%%%%%%%%%%%%%%%%%%%%%%%%%%%%%%%%%%%%%%%%%%
%%%% Redefining the quantifiers, so that some of the longer 
%%%% formulas appear one a single line without problems

%%% Dependent products written with \forall, in the same style
\makeatletter
\def\tfall#1{\forall_{(#1)}\@ifnextchar\bgroup{\,\tfall}{\,}}
\renewcommand{\fall}{\tfall}

%%% Existential quantifier %%%
\def\texis#1{\exists_{(#1)}\@ifnextchar\bgroup{\,\texis}{\,}}
\renewcommand{\exis}{\texis}

%%% Unique existence %%%
\def\uexis#1{\exists!_{(#1)}\@ifnextchar\bgroup{\,\uexis}{\,}}
\makeatother
%%%%%%%%%%%%%%%%%%%%%%%%%%%%%%%%%%%%%%%%%%%%%%%%%%%%%%%%%%%%%%%%%%%%%%%%%%%%%%%%

%%%% Introducing logical usage of fonts.
\newcommand{\modelfont}{\mathit} % use 'mf' in command to indicate model font
\newcommand{\typefont}{\mathsf} % use 'tf' in command to indicate type font
\newcommand{\catfont}{\mathrm} % use 'cf' in command to indicate cat font

%%%%%%%%%%%%%%%%%%%%%%%%%%%%%%%%%%%%%%%%%%%%%%%%%%%%%%%%%%%%%%%%%%%%%%%%%%%%%%%%
%%%% Some macros of the book are redefined.

\renewcommand{\UU}{\typefont{U}}
\renewcommand{\isequiv}{\typefont{isEquiv}}
\renewcommand{\happly}{\typefont{hApply}}
\renewcommand{\pairr}[1]{{\mathopen{}\langle #1\rangle\mathclose{}}}
\renewcommand{\type}{\typefont{Type}}
\renewcommand{\op}[1]{{{#1}^\typefont{op}}}
\renewcommand{\susp}{\typefont{\Sigma}}

%%%%%%%%%%%%%%%%%%%%%%%%%%%%%%%%%%%%%%%%%%%%%%%%%%%%%%%%%%%%%%%%%%%%%%%%%%%%%%%%
%%%% The following is a big unorganized list of new macros that we use in the
%%%% notes. 

\newcommand{\mfM}{\modelfont{M}}
\newcommand{\mfN}{\modelfont{N}}
\newcommand{\tfctx}{\typefont{ctx}}
\newcommand{\mftypfunc}[1]{{\modelfont{typ}^{#1}}}
\newcommand{\mftyp}[2]{{\mftypfunc{#1}(#2)}}
\newcommand{\tftypfunc}[1]{{\typefont{typ}^{#1}}}
\newcommand{\tftyp}[2]{{\tftypfunc{#1}(#2)}}
\newcommand{\hfibfunc}[1]{\typefont{fib}_{#1}}
\newcommand{\mappingcone}[1]{\mathcal{C}_{#1}}
\newcommand{\equifib}{\typefont{equiFib}}
\newcommand{\tfcolim}{\typefont{colim}}
\newcommand{\tflim}{\typefont{lim}}
\newcommand{\tfdiag}{\typefont{diag}}
\newcommand{\tfGraph}{\typefont{Graph}}
\newcommand{\mfGraph}{\modelfont{Graph}}
\newcommand{\unitGraph}{\unit^\mfGraph}
\newcommand{\UUGraph}{\UU^\mfGraph}
\newcommand{\tfrGraph}{\typefont{rGraph}}
\newcommand{\mfrGraph}{\modelfont{rGraph}}
\newcommand{\isfunction}{\typefont{isFunction}}
\newcommand{\tfconst}{\typefont{const}}
\newcommand{\conemap}{\typefont{coneMap}}
\newcommand{\coconemap}{\typefont{coconeMap}}
\newcommand{\tflimits}{\typefont{limits}}
\newcommand{\tfcolimits}{\typefont{colimits}}
\newcommand{\islimiting}{\typefont{isLimiting}}
\newcommand{\iscolimiting}{\typefont{isColimiting}}
\newcommand{\islimit}{\typefont{isLimit}}
\newcommand{\iscolimit}{\typefont{iscolimit}}
\newcommand{\pbcone}{\typefont{cone_{pb}}}
\newcommand{\tfinj}{\typefont{inj}}
\newcommand{\tfsurj}{\typefont{surj}}
\newcommand{\tfepi}{\typefont{epi}}
\newcommand{\tftop}{\typefont{top}}
\newcommand{\sbrck}[1]{\Vert #1\Vert}
\newcommand{\strunc}[2]{\Vert #2\Vert_{#1}}
\newcommand{\gobjclass}{{\typefont{U}^\mfGraph}}
\newcommand{\gcharmap}{\typefont{fib}}
\newcommand{\diagclass}{\typefont{T}}
\newcommand{\opdiagclass}{\op{\diagclass}}
\newcommand{\equifibclass}{\diagclass^{\eqv{}{}}}
\newcommand{\universe}{\typefont{U}}
\newcommand{\catid}[1]{{\catfont{id}_{#1}}}
\newcommand{\isleftfib}{\typefont{isLeftFib}}
\newcommand{\isrightfib}{\typefont{isRightFib}}
\newcommand{\leftLiftings}{\typefont{leftLiftings}}
\newcommand{\rightLiftings}{\typefont{rightLiftings}}
\newcommand{\psh}{\typefont{Psh}}
\newcommand{\rgclass}{\typefont{\Omega^{RG}}}
\newcommand{\terms}[2][]{\lfloor #2 \rfloor^{#1}}
\newcommand{\grconstr}[2]
             {\mathchoice % max size is textstyle size.
             {{\textstyle \int_{#1}}#2}% 
             {\int_{#1}#2}%
             {\int_{#1}#2}%
             {\int_{#1}#2}}
\newcommand{\ctxhom}[3][]{\typefont{hom}_{#1}(#2,#3)}
\newcommand{\graphcharmapfunc}[1]{\gcharmap_{#1}}
\newcommand{\graphcharmap}[2][]{\graphcharmapfunc{#1}(#2)}
\newcommand{\tfexp}[1]{\typefont{exp}_{#1}}
\newcommand{\tffamfunc}{\typefont{fam}}
\newcommand{\tffam}[1]{\tffamfunc(#1)}
\newcommand{\tfev}{\typefont{ev}}
\newcommand{\tfcomp}{\typefont{comp}}
\newcommand{\isDec}[1]{\typefont{isDecidable}(#1)}
\newcommand{\smal}{\mathcal{S}}
\renewcommand{\modal}{{\ensuremath{\ocircle}}}
\newcommand{\eqrel}{\typefont{EqRel}}
\newcommand{\piw}{\ensuremath{\Pi\typefont{W}}} %% to be used in conjunction with -pretopos.
\renewcommand{\sslash}{/\!\!/}
\newcommand{\mprd}[2]{\Pi(#1,#2)}
\newcommand{\msm}[2]{\Sigma(#1,#2)}
\newcommand{\midt}[1]{\idvartype_#1}
\newcommand{\reflf}[1]{\typefont{refl}^{#1}}
\newcommand{\tfJ}{\typefont{J}}
\newcommand{\tftrans}{\typefont{trans}}

\newcommand{\tfT}{\typefont{T}}
\newcommand{\reflsym}{{\mathsf{refl}}}
\newcommand{\strans}[2]{\ensuremath{{#1}_{*}({#2})}}

%%%%%%%%%%%%%%%%%%%%%%%%%%%%%%%%%%%%%%%%%%%%%%%%%%%%%%%%%%%%%%%%%%%%%%%%%%%%%%%%
%%%% JUDGMENTS
%%%%
%%%% Below we define several commands for the judgments of type theory. There
%%%% are commands
%%%% * \jctx for the judgment that something is a context.
%%%% * \jctxeq for the judgment that two contexts are the same
%%%% * \jtype for the judgment that something is a type in a context
%%%% * \jtypeeq for the judgment that two types in the same context are the same
%%%% * \jterm for the judgment that something is a term of a type in a context
%%%% * \jtermeq for the judgment that two terms of the same type are the same

\makeatletter
\newcommand{\jctx}{\@ifnextchar*{\@jctxAlignTrue}{\@jctxAlignFalse}}
\newcommand{\@jctxAlignTrue}[2]{& \vdash #2~ctx}
\newcommand{\@jctxAlignFalse}[1]{\vdash #1~ctx}

\newcommand{\jtype}{\@ifnextchar*{\@jtypeAlignTrue}{\@jtypeAlignFalse}}
\newcommand{\@jtypeAlignFalse}[2]{#1\vdash #2~type}
\newcommand{\@jtypeAlignTrue}[3]{#2 & \vdash #3~type}

\newcommand{\jtermc}{\@ifnextchar*{\@jtermcAlignTrue}{\@jtermcAlignFalse}}
\newcommand{\@jtermcAlignTrue}[3]{ & \vdash #3:#2}
\newcommand{\@jtermcAlignFalse}[2]{\vdash #2:#1}

\newcommand{\jtermt}{\@ifnextchar*{\@jtermtAlignTrue}{\@jtermtAlignFalse}}
\newcommand{\@jtermtAlignTrue}[4]{#2 & \vdash #4:#3}
\newcommand{\@jtermtAlignFalse}[3]{#1 \vdash #3:#2}

\newcommand{\jctxeq}{\@ifnextchar*{\@jctxeqAlignTrue}{\@jctxeqAlignFalse}}
\newcommand{\@jctxeqAlignTrue}[3]{& \vdash #2\jdeq #3~ctx}
\newcommand{\@jctxeqAlignFalse}[2]{\vdash #1\jdeq #2~ctx}

\newcommand{\jtypeeq}{\@ifnextchar*{\@jtypeeqAlignTrue}{\@jtypeeqAlignFalse}}
\newcommand{\@jtypeeqAlignTrue}[4]{#2 & \vdash #3\jdeq #4~type}
\newcommand{\@jtypeeqAlignFalse}[3]{#1\vdash #2\jdeq #3~type}

\newcommand{\jtermceq}{\@ifnextchar*{\@jtermceqAlignTrue}{\@jtermceqAlignFalse}}
\newcommand{\@jtermceqAlignTrue}[4]{& \vdash #3\jdeq #4:#2}
\newcommand{\@jtermceqAlignFalse}[3]{\vdash #2\jdeq #3:#1}

\newcommand{\jtermteq}{\@ifnextchar*{\@jtermteqAlignTrue}{\@jtermteqAlignFalse}}
\newcommand{\@jtermteqAlignTrue}[5]{#2 & \vdash #4\jdeq #5:#3}
\newcommand{\@jtermteqAlignFalse}[4]{#1\vdash #3\jdeq #4:#2}
\makeatother

%%%%%%%%%%%%%%%%%%%%%%%%%%%%%%%%%%%%%%%%%%%%%%%%%%%%%%%%%%%%%%%%%%%%%%%%%%%%%%%%
%%%% Often we shall need to display lists of inference rules. This environment
%%%% adjusts the array environment so that there is enough vertical space
%%%% between two inference rules
%%%%
%%%% bug: there's two much space above the array.

\newenvironment{infarray}[1]{\begingroup\renewcommand*{\arraystretch}{3}
\begin{equation*}
\begin{array}{#1}}{
\end{array}
\end{equation*}
\endgroup}

%%%%%%%%%%%%%%%%%%%%%%%%%%%%%%%%%%%%%%%%%%%%%%%%%%%%%%%%%%%%%%%%%%%%%%%%%%%%%%%%
%%%% CONTEXT EXTENSION 
%%%%
%%%% explicit context extension notation which we will use only rarely

\newcommand{\tfext}{\typefont{ext}}

%%%% The context extension command.
%%%%
%%%% To get a feeling of how the command works, here are a few examples.
%%%% \ctxext{A}{B} will print A.B
%%%% \ctxext{{A}{B}}{C} will print (A.B).C
%%%% \ctxext{{{A}{B}}{C}}{{D}{E}} will print ((A.B).C).(D.E)

\makeatletter
\newcommand{\ctxext}[2]{\@ctxext@ctx #1.\@ctxext@type #2}
\newcommand{\@ctxext}{\@ifnextchar\bgroup{\@@ctxext}{}}
\newcommand{\@ctxext@ctx}{\@ifnextchar\ctxext{\@ctxext@nested}{\@ifnextchar\ctxwk{\@ctxwk@nested}{\@ctxext}}}
\newcommand{\@ctxext@type}{\@ifnextchar\ctxext{\@ctxext@nested}{\@ifnextchar\subst{\@subst@nested}{\@ctxext}}}
\newcommand{\@@ctxext}[1]{\@ifnextchar\bgroup{\@ctxext@parens{#1}}{#1}}
\newcommand{\@ctxext@parens}[2]{(\ctxext{#1}{#2})}
\newcommand{\@ctxext@nested}[3]{\@ctxext@parens{#2}{#3}}
\makeatother

%%%%%%%%%%%%%%%%%%%%%%%%%%%%%%%%%%%%%%%%%%%%%%%%%%%%%%%%%%%%%%%%%%%%%%%%%%%%%%%%
%%%% SUBSTITUTION

\newcommand{\tfsubst}{\typefont{subst}}

%%%% The substitution command will act the following way
%%%%
%%%% \subst{x}{P} will print P[x]
%%%% \subst{x}{{f}{Q}} will print Q[f][x]
%%%% \subst{{x}{f}}{{x}{Q}} will print Q[x][f[x]]

\makeatletter
\newcommand{\subst}[2]{\@subst@type #2[\@subst@term #1]}
\newcommand{\@subst}{\@ifnextchar\bgroup{\@@subst}{}}
\newcommand{\@@subst}[1]{\@ifnextchar\bgroup{\subst{#1}}{#1}}
\newcommand{\@subst@term}{\@subst}
\newcommand{\@subst@type}{\@ifnextchar\ctxext{\@ctxext@nested}{\@ifnextchar\ctxwk{\@ctxwk@nested}{\@subst}}}
\newcommand{\@subst@nested}[3]{\@subst@parens{#2}{#3}}
\newcommand{\@subst@parens}[2]{(\subst{#1}{#2})}
\makeatother

%%%%%%%%%%%%%%%%%%%%%%%%%%%%%%%%%%%%%%%%%%%%%%%%%%%%%%%%%%%%%%%%%%%%%%%%%%%%%%%%
%%%% WEAKENING

\newcommand{\tfwk}{\typefont{wk}}

%%%% The weakening command is very much like the substitution command.

\makeatletter
\newcommand{\ctxwk}[2]{\langle\@ctxwk@act #1\rangle\@ctxwk@pass #2}
\newcommand{\@ctxwk}{\@ifnextchar\bgroup{\@@ctxwk}{}}
\newcommand{\@@ctxwk}[1]{\@ifnextchar\bgroup{\ctxwk{#1}}{#1}}
\newcommand{\@ctxwk@act}{\@ctxwk}
\newcommand{\@ctxwk@pass}{\@ifnextchar\ctxext{\@ctxext@nested}{\@ifnextchar\subst{\@subst@nested}{\@ctxwk}}}
\newcommand{\@ctxwk@parens}[2]{(\ctxwk{#1}{#2})}
\newcommand{\@ctxwk@nested}[3]{\@ctxwk@parens{#2}{#3}}
\makeatother

%%%%%%%%%%%%%%%%%%%%%%%%%%%%%%%%%%%%%%%%%%%%%%%%%%%%%%%%%%%%%%%%%%%%%%%%%%%%%%%%
%%%% When investigation pointed structures we use the \pt macro.

\makeatletter
\newcommand{\pt}[1][]{*_{
  \@ifnextchar\undergraph{\@undergraph@nested}
    {\@ifnextchar\underovergraph{\@underovergraph@nested}{}}#1}}
\makeatother

%%%%%%%%%%%%%%%%%%%%%%%%%%%%%%%%%%%%%%%%%%%%%%%%%%%%%%%%%%%%%%%%%%%%%%%%%%%%%%%%
%%%% OPERATIONS ON GRAPHS
%%%%
%%%% First of all, each graph has a type of vertices and a type of edges. The
%%%% type of vertices of a graph $\Gamma$ is denoted by $\pts{\Gamma}$;
%%%% and likewise for the type of edges.

\makeatletter
\newcommand{\pts}[1]{{\@graphop@nested{#1}}_{0}}
\newcommand{\edg}[1]{{\@graphop@nested{#1}}_{1}}
\newcommand{\@graphop@nested}[1]
  {\@ifnextchar\ctxext{\@ctxext@nested}
      {\@ifnextchar\undergraph{\@undergraph@nested}
         {\@ifnextchar\underovergraph{\@underovergraph@nested}{}}}
    #1}
\makeatother

%%%% The following operations of \undergraph and \underovergraph are used to
%%%% define the free category and the free groupoid of a graph, respectively

\makeatletter
\newcommand{\@undergraphtest}[2]{\@ifnextchar({#1}{#2}}
\newcommand{\undergraph}[2]{\@undergraphtest{\@undergraph@parens{#1}{#2}}{\@undergraph{#1}{#2}}}
\newcommand{\@undergraph}[2]{{#2/#1}}
\newcommand{\@undergraph@nested}[3]{\@undergraph@parens{#2}{#3}}
\newcommand{\@undergraph@parens}[2]{(\@undergraph{#1}{#2})}
\makeatother

\makeatletter
\newcommand{\underovergraph}[2]{\@underovergraphtest{\@underovergraph@parens{#1}{#2}}{\@underovergraph{#1}{#2}}}
\newcommand{\@underovergraph}[2]{{#2}\,{\parallel}\,{#1}}
\newcommand{\@underovergraphtest}{\@undergraphtest}
\newcommand{\@underovergraph@parens}[2]{(\@underovergraph{#1}{#2})}
\newcommand{\@underovergraph@nested}[3]{\@underovergraph@parens{#2}{#3}}
\makeatother

\newcommand{\graphid}[1]{\mathrm{id}_{#1}}
\newcommand{\freecat}[1]{\mathcal{C}(#1)}
\newcommand{\freegrpd}[1]{\mathcal{G}(#1)}

%%%%%%%%%%%%%%%%%%%%%%%%%%%%%%%%%%%%%%%%%%%%%%%%%%%%%%%%%%%%%%%%%%%%%%%%%%%%%%%%
%% Some tikz macros to typeset diagrams uniformly.

\tikzset{patharrow/.style={double,double equal sign distance,-,font=\scriptsize}}
\tikzset{description/.style={fill=white,inner sep=2pt}}

%% Used for extra wide diagrams, e.g. when the label is too large otherwise.
\tikzset{commutative diagrams/column sep/Huge/.initial=18ex}

%%%%%%%%%%%%%%%%%%%%%%%%%%%%%%%%%%%%%%%%%%%%%%%%%%%%%%%%%%%%%%%%%%%%%%%%%%%%%%%%
%%%% New theorem environment for conjectures.

\defthm{conj}{Conjecture}{Conjectures}

%%%%%%%%%%%%%%%%%%%%%%%%%%%%%%%%%%%%%%%%%%%%%%%%%%%%%%%%%%%%%%%%%%%%%%%%%%%%%%%%
%%%% The following environment for desiderata should not be there. It is better
%%%% to use the issue tracker for desiderata.

\newenvironment{desiderata}{\begingroup\color{blue}\textbf{Desiderata.}}
{\endgroup}


\title{Notes on the smash product}
\date{\today}
\usepackage{fullpage}
\newcommand{\pmap}{\to}
\newcommand{\lpmap}{\xrightarrow}
\renewcommand{\smash}{\wedge}
\renewcommand{\phi}{\varphi}
\renewcommand{\epsilon}{\varepsilon}
\newcommand{\tr}{\cdot}
\renewcommand{\o}{\ensuremath{\circ}}

\begin{document}

\maketitle

\section{Pointed Types}

\begin{defn}
  We work in the $(\infty,1)$-category of pointed types.
  \begin{itemize}
  \item The objects are pointed types $A$, types together with a basepoint $a_0:A$.
\item 1-cells are pointed maps $f:A\to B$ which are maps with a chosen path $f_0:f(a_0)=b_0$. We
  write $A\pmap B$ for pointed maps and $A\pmap B\pmap C$ means $A\pmap (B\pmap C)$.
\item 2-cells pointed homotopies. A pointed homotopy $h:f\sim g$ is a homotopy with a chosen 2-path
  $h(a_0) \tr g_0 = f_0$.
\item As 3-cells (or higher cells) we take equalities between 2-cells (or higher cells).
\end{itemize}
\end{defn}

\begin{rmk}
\item All types, maps and homotopies in these notes are pointed, unless explicitly mentioned
  otherwise. Whenever we say that a diagram of $n$-cells commutes we mean it in the sense that there
  is an $(n+1)$-cell witnessing it.
\item Pointed homotopies are equivalent to equalities of pointed types: $(f\sim g)\equiv (f=g)$. So
  we could have chosen to define our 2-cells as equalities between 1-cells. We choose not to, since
  the aforementioned equivalence requires function extensionality. In a type theory where function
  extensionality doesn't compute (like Lean) it is better to define the type of pointed homotopies
  manually so that the underlying homotopy of a 2-cell is definitionally equal to the homotopy we
  started with. In diagrams, we will denote pointed homotopies by equalities, but we always mean
  pointed homotopies.
\item The type $A\to B$ of pointed maps from $A$ to $B$ is itself pointed, with as basepoint the
  constant map $0\equiv0_{A,B}:A\to B$ which has as underlying function $\lam{a:A}b_0$. In these
  notes we will not use $0$ for the empty type (since that is not pointed, we will not use the empty
  type).
\end{rmk}

\begin{lem}
  Given maps $f:A'\pmap A$ and $g:B\pmap B'$. Then there are maps
  $(f\pmap C):(A\pmap C)\pmap(A'\pmap C)$ and $(C\pmap g):(C\pmap B)\pmap(C\pmap B')$ given by
  precomposition with $f$, resp. postcomposition with $g$. The map $\lam{g}C\pmap g$ preserves the basepoint, giving rise to a map $$(C\pmap ({-})):(B\pmap B')\pmap(C\pmap B)\pmap(C\pmap B').$$
  Also, the following square commutes
\begin{center}
\begin{tikzcd}
(A\pmap B) \arrow[r,"A\pmap g"]\arrow[d,"f\pmap B"] & (A\pmap B')\arrow[d,"f\pmap B'"] \\
(A'\pmap B) \arrow[r,"A'\pmap g"] & (A'\pmap B')
\end{tikzcd}
\end{center}

\end{lem}


\section{Smash Product}


\begin{lem}\mbox{}
  \begin{itemize}
  \item The smash is functorial: if $f:A\pmap A'$ and $g:B\pmap B'$ then
    $f\smash g:A\smash B\pmap A'\smash B'$. We write $A\smash g$ or $f\smash B$ if one of the
    functions is the identity function.
  \item Smash preserves composition, which gives rise to the interchange law:
    $(f' \o f)\smash (g' \o g) \sim f' \smash g' \o f \smash g$
  \item If $p:f\sim f'$ and $q:g\sim g'$ then $p\smash q:f\smash g\sim f'\smash g'$. This operation
    preserves reflexivities, symmetries and transitivies.
  \item $f\smash0\sim0$ and $0\smash g\sim 0$.
  \end{itemize}
\end{lem}

\begin{lem}\label{lem:smash-coh}
  Suppose that we have maps $A_1\lpmap{f_1}A_2\lpmap{f_2}A_3$ and $B_1\lpmap{g_1}B_2\lpmap{g_2}B_3$
  and suppose that either $g_1$ or $g_2$ is constant. Then there are two homotopies
  $(f_2 \o f_1)\smash (g_2 \o g_1)\sim 0$, one which uses interchange and one which doesn't. These two
  homotopies are equal. Specifically, the following two diagrams commute: %% TODO: reformulate
\begin{center}
\begin{tikzcd}
(f_2 \o f_1)\smash (g_2 \o 0) \arrow[r, equals]\arrow[dd,equals] &
(f_2 \smash g_1)\o (f_1 \smash 0)\arrow[d,equals] \\
& (f_2 \smash g_1)\o 0\arrow[d,equals] \\
(f_2 \o f_1)\smash 0 \arrow[r,equals] &
0
\end{tikzcd}
\qquad
\begin{tikzcd}
(f_2 \o f_1)\smash (0 \o g_1) \arrow[r, equals]\arrow[dd,equals] &
(f_2 \smash 0)\o (f_1 \smash g_1)\arrow[d,equals] \\
& 0\o (f_1 \smash g_1)\arrow[d,equals] \\
(f_2 \o f_1)\smash 0 \arrow[r,equals] &
0
\end{tikzcd}
\end{center}

\end{lem}
\begin{proof}
We will only do the case where $g_1\jdeq 0$, the other case is similar (but slightly easier). In this case, we need to show that the following diagram commutes.
Proof to do.
\end{proof}

\begin{thm}\label{thm:smash-functor-right}
Given pointed types $A$, $B$ and $C$, the functorial action of the smash product induces a map
$$({-})\smash C:(A\pmap B)\pmap(A\smash C\pmap B\smash C)$$
which is natural in $A$, $B$ and $C$. (note: $(A\smash C\pmap B\smash C)$ is both covariant and contravariant in $C$).
\end{thm}
\begin{proof}
  First note that $\lam{f}f\smash C$ preserves the basepoint so that the map is indeed pointed. We
  show that this map is natural in each of its arguments individually, which means we need to fill
  the following squares for $f : A' \to A$ $g:B\to B'$ and $h:C\to C'$.
\begin{center}
\begin{tikzcd}
(A\pmap B) \arrow[r,"({-})\smash C"]\arrow[d,"f\pmap B"] &
(A\smash C\pmap B\smash C)\arrow[d,"f\smash C\pmap B\smash C"] \\
(A'\pmap B) \arrow[r,"({-})\smash C"] &
(A'\smash C\pmap B\smash C)
\end{tikzcd}
\begin{tikzcd}
(A\pmap B) \arrow[r,"({-})\smash C"]\arrow[d,"A\pmap g"] &
(A\smash C\pmap B\smash C)\arrow[d,"A\smash C\pmap g\smash C"] \\
(A\pmap B') \arrow[r,"({-})\smash C"] &
(A\smash C\pmap B'\smash C)
\end{tikzcd}
\begin{tikzcd}[column sep=large]
(A\pmap B) \arrow[r,"({-})\smash C"]\arrow[d,"({-})\smash C'"] &
(A\smash C\pmap B\smash C)\arrow[d,"A\smash C\pmap B\smash h"] \\
(A\smash C'\pmap B\smash C') \arrow[r,"A\smash h\pmap B\smash C'"] &
(A\smash C\pmap B\smash C')
\end{tikzcd}
\end{center}
Let $k:A\pmap B$. Then as homotopy the naturality in $A$ becomes
$(k\o f)\smash C=k\smash C\o f\smash C$. To prove an equality between pointed maps, we need to give a pointed homotopy, which is given by interchange. To show that this homotopy
is pointed, we need to fill the following square, which follows from \autoref{lem:smash-coh}.
\begin{center}
\begin{tikzcd}
(0 \o f)\smash C \arrow[r, equals]\arrow[dd,equals] &
(0 \smash C)\o (f \smash C)\arrow[d,equals] \\
& 0 \o (f \smash C)\arrow[d,equals] \\
0\smash C \arrow[r,equals] &
0
\end{tikzcd}
\end{center}
The naturality in $B$ is almost the same: for the underlying homotopy we need to show
$(g \o k)\smash C = g\smash C \o k\smash C$. For the pointedness we need to fill the following
square, which follows from \autoref{lem:smash-coh}.
\begin{center}
\begin{tikzcd}
(g \o 0)\smash C \arrow[r, equals]\arrow[dd,equals] &
(g \smash C)\o (0 \smash C)\arrow[d,equals] \\
& (g\smash C) \o 0\arrow[d,equals] \\
0\smash C \arrow[r,equals] &
0
\end{tikzcd}
\end{center}
The naturality in $C$ is harder. For the underlying homotopy we need to show
$B\smash h\o g\smash C=g\smash C'\o A\smash h$. This follows by applying interchange twice:
$$B\smash h\o g\smash C\sim(\idfunc[B]\o g)\smash(h\o\idfunc[C])\sim(g\o\idfunc[A])\smash(\idfunc[C']\o h)\sim g\smash C'\o A\smash h.$$
To show that this homotopy is pointed, we need to fill the following square:
% \begin{center}
% \begin{tikzcd}
% (B\smash h\o g\smash C) \arrow[r, equals]\arrow[dd,equals] &
% (g \smash C)\o (0 \smash C)\arrow[d,equals] \\
% & (g\smash C) \o 0\arrow[d,equals] \\
% 0\smash C \arrow[r,equals] &
% 0
% \end{tikzcd}
% \end{center}

\end{proof}

\section{Adjunction}

\begin{defn}
There is a unit $\eta:A\pmap B\pmap A\smash B$ and counit $\epsilon : (B\pmap C)\smash B \pmap C$
\end{defn}


\begin{defn}
The function $e\jdeq e_{A,B,C}:(A\pmap B\pmap C)\pmap(A\smash B\pmap C)$ is defined as the composite
$$(A\pmap B\pmap C)\lpmap{({-})\smash B}(A\smash B\pmap (B\pmap C)\smash B)\lpmap{A\smash B \pmap\epsilon}(A\smash B\pmap C)).$$

\begin{lem}
$e$ has an inverse $\inv e\jdeq \inv{e}_{A,B,C}:(A\smash B\pmap C)\pmap(A\pmap B\pmap C)$ which is defined as
$$(A\smash B\pmap C)\lpmap{B\pmap({-})}((B\pmap A\smash B)\pmap (B\pmap C))\lpmap{\eta\pmap(B\pmap C)}(A\pmap B\pmap C).$$
\begin{proof}
We do not actually use that $\inv{e}_{A,B,C}$ has this form in later proofs, we only use that $e$ is invertible (the former fact is also not formalized). Proof to do.
\end{proof}
\end{lem}
\end{defn}
\begin{lem}
$e$ is natural in $A$, $B$ and $C$.
\end{lem}
\begin{proof}
\textbf{$e$ is natural in $A$}. Suppose that $f:A'\pmap A$. Then the following diagram commutes. The left square commutes by naturality of $({-})\smash B$ in the first argument and the right square commutes because composition on the left commutes with composition on the right.
\begin{center}
\begin{tikzcd}
(A\pmap B\pmap C) \arrow[r,"({-})\smash B"]\arrow[d,"f\pmap B\pmap C"] &
(A\smash B\pmap (B\pmap C)\smash B) \arrow[r,"A\smash B\pmap\epsilon"]\arrow[d,"f\smash B\pmap\cdots"]  &
(A\smash B\pmap C)\arrow[d,"f\smash B\pmap C"] \\
(A'\pmap B\pmap C) \arrow[r,"({-})\smash B"] &
(A'\smash B\pmap (B\pmap C)\smash B) \arrow[r,"A\smash B\pmap\epsilon"] &
(A'\smash B\pmap C)
\end{tikzcd}
\end{center}

\textbf{$e$ is natural in $C$}. Suppose that $f:C\pmap C'$. Then in the following diagram the left square commutes by naturality of $({-})\smash B$ in the second argument (applied to $B\pmap f$) and the right square commutes by applying the functor $A\smash B \pmap({-})$ to the naturality of $\epsilon$ in the second argument.
\begin{center}
\begin{tikzcd}
(A\pmap B\pmap C) \arrow[r]\arrow[d] &
(A\smash B\pmap (B\pmap C)\smash B) \arrow[r]\arrow[d] &
(A\smash B\pmap C)\arrow[d] \\
(A\pmap B\pmap C') \arrow[r] &
(A\smash B\pmap (B\pmap C')\smash B) \arrow[r] &
(A\smash B\pmap C')
\end{tikzcd}
\end{center}

\textbf{$e$ is natural in $B$}. Suppose that $f:B'\pmap B$. The diagram looks weird since $({-})\smash B$ is both contravariant and covariant in $B$. Then we get the following diagram. The front square commutes by naturality of $({-})\smash B$ in the second argument (applied to $f\pmap C$). The top square commutes by naturality of $({-})\smash B$ in the third argument, the back square commutes because composition on the left commutes with composition on the right, and finally the right square commutes by applying the functor $A\smash B' \pmap({-})$ to the naturality of $\epsilon$ in the first argument.
\begin{center}
\begin{tikzcd}[row sep=scriptsize, column sep=-4em]
& (A\smash B\pmap (B\pmap C)\smash B) \arrow[rr] \arrow[dd] & & (A\smash B'\pmap (B\pmap C)\smash B)\arrow[dd] \\
(A\pmap B\pmap C) \arrow[ur] \arrow[rr, crossing over] \arrow[dd] & & (A\smash B'\pmap (B\pmap C)\smash B') \arrow[ur] \\
& (A\smash B\pmap C)\arrow[rr] &  & (A\smash B'\pmap C) \\
(A\pmap B'\pmap C) \arrow[rr] & & (A\smash B'\pmap (B'\pmap C)\smash B') \arrow[ur] \arrow[from=uu, crossing over]\\
\end{tikzcd}
\end{center}

\end{proof}


\section{Notes on the formalization}

The order of arguments are different in the formalization here and there.
Also, some smashes are commuted. This is because some unfortunate choices have been made in the formalization. Reversing these choices is possible, but probably more work than it's worth (the final result is exactly the same).

\end{document}
